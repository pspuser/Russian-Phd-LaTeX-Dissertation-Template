
{\actuality} 
%\todo{поменять с 1 режим черновика на 0 в setup.tex для соблюдения ГОСТ}

	Квантовая рассылка ключа (КРК) является одной из наиболее развитых на сегодняшний день направлений квантовой информатики - области знаний на стыке фотоники, теории информации и квантовой физики, которая использует квантовые биты (кубиты), то есть квантовые системы, способные находиться в двух состояниях, для осуществления передачи, хранения и обработки информации [1]. Системы квантовой коммуникации позволяют их пользователям осуществлять рассылку симметричных кодирующих последовательностей, или ключей, таким образом, что любые попытки вторжения нелегитимного пользователя в канал связи принципиально обнаруживаются. К другим перспективным приложениям квантовой информатики относятся квантовый компьютер - гипотетическое вычислительное устройство, работающее на принципах квантовой механики [2] и квантовая телепортация - передача квантового состояния на расстояние [3].
	
	Отличия физических реализаций систем КРК от идеальных моделей, используемых в теоретических обоснованиях, могут быть основанием для проведения различных типов атак на используемое в составе систем оборудование. Ранее было показано, что стробируемые детекторы одиночных фотонов (ДОФ) нескольких коммерчески доступных систем КРК были уязвимы к воздействию злоумышленника. Примечательно, что воздействие может осуществляться только оптическими методами из квантового канала, которым соединяются легитимные пользователи. На практике чаще всего применяют оптические волокно, как среду для передачи квантовых состояний, в качестве которых обычно используют одиночные фотоны и их свойства. 
	
	Такой тип атак получил название <<атаки с навязыванием ключа>> или <<атаки с поддельными состояниями>>. Он основывается на том факте, что для детектирования фотонов применяют детектор на базе лавинных фотодиодов (ЛФД), работающих в режиме Гейгера, или счета фотонов. При таком подходе в отсутствие принятых дополнительно защитных мер злоумышленник имеет возможность при помощи мощного постоянного оптического излучения выводить детектор из режима Гейгера в линейный режим и провоцировать срабатывания, благодаря использованию мощных коротких импульсов. Таким образом, практически осуществимой является возможность полного контроля узла регистрации одиночных фотонов, критического для протоколов КРК, и как следствие незаметное получение полного ключа, коррелирующего с таковым у легитимных пользователей. 
	
	Известны несколько способов противостояния такому типу атак, однако на практике большая часть из них не проходила тестирование в лабораториях, специализирующихся на квантовом взломе, а к тем контрмерам, которые были проанализированы и исследованы, обязательно находились контратаки и методы взлома, не учтенные разработчиками контрмер. 
	
	Наиболее эффективной контрмерой против любого тип атак на детекторы является реализация принципиально новой архитектуры КРК, устойчивой к атакам на измерительное оборудование (Measurement-Device-Independent, или MDI). При таком подходе изначально предполагается, что узел регистрации одиночных фотонов вынесен за пределы блоков отправителя и получателя и полностью доступен злоумышленнику. Однако, такой подход характеризуется более высоким уровнем сложности реализации и более низкими характеристиками по отношению к системам КРК с топологией точка-точка с узлом регистрации в одном из блоков. 
	
	Одним из перспективных подходов является метод квантовой коммуникации на боковых частотах. Его отличительной особенностью является вынесение квантового канала на генерируемые в результате модуляции боковые частоты. Благодаря этому обеспечивается высокая спектральную эффективность и хорошие показатели по соотношению скорость формирования ключа к дистанции между блоками отправителя и получателя. Однако, в составе такого типа систем для малых и средних дистанций (до 100 км) также применяются детекторы одиночных фотонов на базе ЛФД, и устойчивость к атакам на измерительное оборудование систем квантовой коммуникации на боковых частотах исследована не была. 

% Обзор, введение в тему, обозначение места данной работы в
% мировых исследованиях и~т.\:п., можно использовать ссылки на~другие
% работы\ifnumequal{\value{bibliosel}}{1}{~\autocite{Gosele1999161}}{}
% (если их~нет, то~в~автореферате
% автоматически пропадёт раздел <<Список литературы>>). Внимание! Ссылки
% на~другие работы в разделе общей характеристики работы можно
% использовать только при использовании \verb!biblatex! (из-за технических
% ограничений \verb!bibtex8!. Это связано с тем, что одна
% и~та~же~характеристика используются и~в~тексте диссертации, и в
% автореферате. В~последнем, согласно ГОСТ, должен присутствовать список
% работ автора по~теме диссертации, а~\verb!bibtex8! не~умеет выводить в одном
% файле два списка литературы).
% При использовании \verb!biblatex! возможно использование исключительно
% в~автореферате подстрочных ссылок
% для других работ командой \verb!\autocite!, а~также цитирование
% собственных работ командой \verb!\cite!. Для этого в~файле
% \verb!Synopsis/setup.tex! необходимо присвоить положительное значение
% счётчику \verb!\setcounter{usefootcite}{1}!.
% 
% Для генерации содержимого титульного листа автореферата, диссертации
% и~презентации используются данные из файла \verb!common/data.tex!. Если,
% например, вы меняете название диссертации, то оно автоматически
% появится в~итоговых файлах после очередного запуска \LaTeX. Согласно
% ГОСТ 7.0.11-2011 <<5.1.1 Титульный лист является первой страницей
% диссертации, служит источником информации, необходимой для обработки и
% поиска документа>>. Наличие логотипа организации на~титульном листе
% упрощает обработку и~поиск, для этого разметите логотип вашей
% организации в папке images в~формате PDF (лучше найти его в векторном
% варианте, чтобы он хорошо смотрелся при печати) под именем
% \verb!logo.pdf!. Настроить размер изображения с логотипом можно
% в~соответствующих местах файлов \verb!title.tex!  отдельно для
% диссертации и автореферата. Если вам логотип не~нужен, то просто
% удалите файл с~логотипом.

% \ifsynopsis
% Этот абзац появляется только в~автореферате.
% Для формирования блоков, которые будут обрабатываться только в~автореферате,
% заведена проверка условия \verb!\!\verb!ifsynopsis!.
% Значение условия задаётся в~основном файле документа (\verb!synopsis.tex! для
% автореферата).
% \else
% Этот абзац появляется только в~диссертации.
% Через проверку условия \verb!\!\verb!ifsynopsis!, задаваемого в~основном файле
% документа (\verb!dissertation.tex! для диссертации), можно сделать новую
% команду, обеспечивающую появление цитаты в~диссертации, но~не~в~автореферате.
% \fi

% {\progress}
% Этот раздел должен быть отдельным структурным элементом по
% ГОСТ, но он, как правило, включается в описание актуальности
% темы. Нужен он отдельным структурынм элемементом или нет ---
% смотрите другие диссертации вашего совета, скорее всего не нужен.

{\aim} данной работы является исследование возможностей злоумышленника по получению секретного ключа с использованием атак на измерительное оборудование систем квантовой коммуникации на боковых частотах и разработка методов противодействия атакам.


Для~достижения поставленной цели необходимо было решить следующие {\tasks}:
\begin{enumerate}
  \item Исследование устойчивости детектора одиночных фотонов, применяемого в системах квантовой коммуникации на боковых частотах, к атакам с выведением из режима Гейгера (<<ослеплением>>). 

  \item Оценка возможностей злоумышленника при атаке с выведением из режима Гейгера для систем квантовой коммуникации на боковых частотах. 

  \item Разработка оптической схемы системы квантовой коммуникации, устойчивой к атакам на измерительное оборудование. 

  \item Разработка протокола квантовой рассылки ключа, устойчивого к атаке на измерительное оборудование. 

\end{enumerate}


{\novelty}
\begin{enumerate}
  \item Впервые исследована устойчивость системы квантовой коммуникации на боковых частотах к атакам злоумышленника на измерительное оборудование приёмного блока. 
  \item Впервые предложена и применена контрмера против атаки злоумышленника с выведением детектора из режима Гейгера и навязыванием легитимным пользователям ключа. 
  \item Было выполнено оригинальное исследование и разработана оптическая схема с вынесением устройства детектирования одиночных фотонов из приемного блока в <<недоверенный>> узел, подконтрольный злоумышленнику 
\end{enumerate}

{\influence} 

Разработанные методы и подход позволили однозначно определить уязвимость коммерчески доступных детекторов компании id Quantique модели id210 к выведению из режима Гейгера оптическими средствами. В связи с чем доказана необходимость применения дополнительных мер защиты от атак на измерительный узел. Предложена схема, позволяющая производить активный мониторинг попыток <<ослепить>> детектор, благодаря использованию особенностей систем квантовой коммуникации на боковых частотах. Результаты внедрены в производство ООО "Кванттелеком". 

%  {\methods} \todo{TO DO} \ldots

{\defpositions}
\begin{enumerate}
  \item Использование коммерческих детекторов одиночных фотонов на основе лавинных фотодиодов в режиме Гейгера модели id210 с частотой стробирования 100 МГц  требует применения дополнительных средств защиты от атаки с выведением из режима Гейгера.   
  \item \VCc{ КОНТРМЕРА }
  \item Метод квантовой коммуникации на боковых частотах позволяет реализовывать протокол, устойчивый к контролю нелегитимным пользователем измерительного оборудования. 
  \item В результате интерференции квантового фазомодулированного сигнала на боковых частотах на симметричном светоделителе в схеме квантовой рассылки ключа с узлом регистрации, независящим от легитимного пользователя, происходит спектральное разделение квантового сигнала и сигнала на центральной длине волны с их независимой регистрацией в разных плечах светоделителя. 
  \item Показана возможность двукратного увеличения дальности квантовой рассылки ключа на боковых частотах посредством применения недоверенной системы регистрации квантовых состояний. 
\end{enumerate}
% В папке Documents можно ознакомиться в решением совета из Томского ГУ
% в~файле \verb+Def_positions.pdf+, где обоснованно даются рекомендации
% по~формулировкам защищаемых положений.

{\reliability} полученных результатов обеспечивается применением утверждённых методик проведений экспериментальных исследований и аттестованного оборудование. Математическое моделирование и обработка данных, полученных в результате экспериментов, осуществлялось с использованием пакетов прикладных программ MathCad и Excel. Результаты находятся в соответствии с результатами, полученными другими авторами.


{\probation}
Основные результаты работы докладывались~на:
перечисление основных конференций, симпозиумов и~т.\:п. \todo{TO DO}

{\contribution} Автор принимал активное участие \todo{TO DO} \ldots

%\publications\ Основные результаты по теме диссертации изложены в ХХ печатных изданиях~\cite{Sokolov,Gaidaenko,Lermontov,Management},
%Х из которых изданы в журналах, рекомендованных ВАК~\cite{Sokolov,Gaidaenko},
%ХХ --- в тезисах докладов~\cite{Lermontov,Management}.

\ifnumequal{\value{bibliosel}}{0}{% Встроенная реализация с загрузкой файла через движок bibtex8
    \publications\ Основные результаты по теме диссертации изложены в XX печатных изданиях,
    X из которых изданы в журналах, рекомендованных ВАК,
    X "--- в тезисах докладов.%
}{% Реализация пакетом biblatex через движок biber
%Сделана отдельная секция, чтобы не отображались в списке цитированных материалов
    \begin{refsection}[vak,wos,scopus,papers,conf]% Подсчет и нумерация авторских работ. Засчитываются только те, которые были прописаны внутри \nocite{}.
        %Чтобы сменить порядок разделов в сгрупированном списке литературы необходимо перетасовать следующие три строчки, а также команды в разделе \newcommand*{\insertbiblioauthorgrouped} в файле biblio/biblatex.tex
        \printbibliography[heading=countauthorvak, env=countauthorvak, keyword=biblioauthorvak, section=1]%
        \printbibliography[heading=countauthorwos,env=countauthorwos, keyword=biblioauthorwos, section=1]%
        \printbibliography[heading=countauthorscopus,env=countauthorscopus, keyword=biblioauthorscopus, section=1]%
	\printbibliography[heading=countauthorconf, env=countauthorconf, keyword=biblioauthorconf, section=1]%
        \printbibliography[heading=countauthorothers, env=countauthorothers, keyword=biblioauthorothers, section=1]%
        \printbibliography[heading=countauthor, env=countauthor, keyword=biblioauthor, section=1]%
        \nocite{%Порядок перечисления в этом блоке определяет порядок вывода в списке публикаций автора
                vakbib1,vakbib2,%
		wosbib1,%
		scbib1,%
                confbib1,confbib2,%
                bib1,bib2,%
        }%
	\publications\ Основные результаты по теме диссертации изложены
	\setcounter{citeauthorscwostot}{\value{citeauthorscopus}} % вместе setcounter и addtocounter добавляют пробел между словами. По-этому они так раскиданы.
        в~\arabic{citeauthor}~печатных изданиях,
	\addtocounter{citeauthorscwostot}{\value{citeauthorwos}}
	\arabic{citeauthorvak} из которых изданы в журналах, рекомендованных ВАК\sloppy
	\ifnum \value{citeauthorscwostot}>0
	, \arabic{citeauthorscwostot} "--- в~периодических научных журналах, индексируемых Web of Science и Scopus\sloppy
	\fi
	\ifnum \value{citeauthorconf}>0
	, \arabic{citeauthorconf} "--- в~тезисах докладов.
	\else
	.
	\fi
    \end{refsection}
    \begin{refsection}[vak,wos,scopus,papers,conf]%Блок, позволяющий отобрать из всех работ автора наиболее значимые, и только их вывести в автореферате, но считать в блоке выше общее число работ
        \printbibliography[heading=countauthorvak, env=countauthorvak, keyword=biblioauthorvak, section=2]%
        \printbibliography[heading=countauthorwos, env=countauthorwos, keyword=biblioauthorwos, section=2]%
        \printbibliography[heading=countauthorscopus, env=countauthorscopus, keyword=biblioauthorscopus, section=2]%
        \printbibliography[heading=countauthorothers, env=countauthorothers, keyword=biblioauthorothers, section=2]%
        \printbibliography[heading=countauthorconf, env=countauthorconf, keyword=biblioauthorconf, section=2]%
        \printbibliography[heading=countauthor, env=countauthor, keyword=biblioauthor, section=2]%
        \nocite{vakbib2}%vak
        \nocite{bib1}%other
        \nocite{confbib1}%conf
    \end{refsection}
}
При использовании пакета \verb!biblatex! для автоматического подсчёта
количества публикаций автора по теме диссертации, необходимо
их~здесь перечислить с использованием команды \verb!\nocite!.
