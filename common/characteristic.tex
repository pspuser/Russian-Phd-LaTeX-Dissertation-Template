
{\actuality} 
%\todo{поменять с 1 режим черновика на 0 в setup.tex для соблюдения ГОСТ}

	Квантовая информатика -- область знаний, сочетающая в себе элементы теории информации и фотоники. В качестве базовой единицы в рамках квантовой информатики используются квантовые биты, или кубиты, -- системы, которые могут находиться в состоянии суперпозиции и применяться для хранения, вычисления и передачи информации. Одним из перспективных направлений являются квантовый вычислитель и квантовая память - основные составляющие квантового компьютера -- принципиально нового типа вычислительных устройств, работающего на фундаментальных принципах квантовой механики. Передача квантового состояния на различные расстояния, или квантовая телепортация, -- другое перспективное приложение квантовой информатики. Генерация симметричной битовой последовательности квантовым методом, или квантовая рассылка ключа (КРК), сформировалось как научное направление в 80-ых годах 20-го века. Данное направление является передовым с точки зрения прикладного применения. В системах квантовой коммуникации (СКК) за счет использования квантовых состояний два и более легитимных абонентов могут осуществлять рассылку симметричных битовых последовательностей, которые в последствии могут быть использованы в качестве ключей для кодирования информации, таким образом, что любые попытки несанкционированного доступа нелегитимного пользователя в канал связи будут обнаружены по возросшему уровню ошибок.
	
	Отличия физических реализаций систем КРК от идеальных моделей, используемых в теоретических обоснованиях, могут быть основанием для проведения различных типов атак на используемое в составе систем оборудование. Ранее было показано, что стробируемые детекторы одиночных фотонов (ДОФ) нескольких коммерчески доступных систем КРК были уязвимы к воздействию злоумышленника. Примечательно, что воздействие может осуществляться оптическими методами, в том числе и  непосредственно из соединяющего легитимных пользователей квантового канала, который доступен злоумышленнику. На практике чаще всего применяют оптическое волокно, как среду для передачи квантовых состояний, в качестве которых обычно используют одиночные фотоны и их свойства. 
	
	Такой тип атак получил название <<атаки с навязыванием ключа>> или <<атаки с поддельными состояниями>>. Он основывается на том факте, что для детектирования фотонов применяют детектор на базе лавинных фотодиодов (ЛФД), работающих в режиме Гейгера, или счета фотонов. При таком подходе в отсутствие принятых дополнительно защитных мер злоумышленник имеет возможность при помощи мощного постоянного оптического излучения выводить детектор из режима Гейгера в линейный режим и провоцировать срабатывания, благодаря использованию мощных коротких импульсов. Таким образом, практически осуществимой является возможность полного контроля узла регистрации одиночных фотонов, критического для протоколов КРК, и как следствие незаметное получение полного ключа, коррелирующего с таковым у легитимных пользователей. 
	
	Известны несколько способов противостояния такому типу атак, однако на практике большая часть из них не проходила тестирование в лабораториях, специализирующихся на квантовом взломе, а к тем контрмерам, которые были проанализированы и исследованы, обязательно находились контратаки и методы взлома, не учтенные разработчиками контрмер. 
	
	Наиболее эффективной контрмерой против основных известных типов атак на детекторы является реализация принципиально новой архитектуры КРК, устойчивой к атакам на измерительное оборудование (Measurement-Device-Independent, или MDI). При таком подходе изначально предполагается, что узел регистрации одиночных фотонов вынесен за пределы блоков отправителя и получателя и полностью доступен злоумышленнику. Однако, такой подход характеризуется более высоким уровнем сложности реализации оптической схемы и более низкими характеристиками по отношению к системам КРК с топологией точка-точка с узлом регистрации в одном из блоков. 
	
	Одним из перспективных подходов является метод квантовой коммуникации на боковых частотах (ККБЧ). Его отличительной особенностью является вынесение квантового канала на генерируемые в результате модуляции боковые частоты. Благодаря этому обеспечивается высокая устойчивость к воздействию внешних факторов и спектральная эффективность, а также хорошие показатели по соотношению скорости формирования ключа к дистанции между блоками отправителя и получателя. Однако, в составе такого типа систем для малых и средних дистанций (до 100 км) также применяются детекторы одиночных фотонов на базе ЛФД, и устойчивость к атакам на измерительное оборудование систем ККБЧ исследована не была. 

% Обзор, введение в тему, обозначение места данной работы в
% мировых исследованиях и~т.\:п., можно использовать ссылки на~другие
% работы\ifnumequal{\value{bibliosel}}{1}{~\autocite{Gosele1999161}}{}
% (если их~нет, то~в~автореферате
% автоматически пропадёт раздел <<Список литературы>>). Внимание! Ссылки
% на~другие работы в разделе общей характеристики работы можно
% использовать только при использовании \verb!biblatex! (из-за технических
% ограничений \verb!bibtex8!. Это связано с тем, что одна
% и~та~же~характеристика используются и~в~тексте диссертации, и в
% автореферате. В~последнем, согласно ГОСТ, должен присутствовать список
% работ автора по~теме диссертации, а~\verb!bibtex8! не~умеет выводить в одном
% файле два списка литературы).
% При использовании \verb!biblatex! возможно использование исключительно
% в~автореферате подстрочных ссылок
% для других работ командой \verb!\autocite!, а~также цитирование
% собственных работ командой \verb!\cite!. Для этого в~файле
% \verb!Synopsis/setup.tex! необходимо присвоить положительное значение
% счётчику \verb!\setcounter{usefootcite}{1}!.
% 
% Для генерации содержимого титульного листа автореферата, диссертации
% и~презентации используются данные из файла \verb!common/data.tex!. Если,
% например, вы меняете название диссертации, то оно автоматически
% появится в~итоговых файлах после очередного запуска \LaTeX. Согласно
% ГОСТ 7.0.11-2011 <<5.1.1 Титульный лист является первой страницей
% диссертации, служит источником информации, необходимой для обработки и
% поиска документа>>. Наличие логотипа организации на~титульном листе
% упрощает обработку и~поиск, для этого разметите логотип вашей
% организации в папке images в~формате PDF (лучше найти его в векторном
% варианте, чтобы он хорошо смотрелся при печати) под именем
% \verb!logo.pdf!. Настроить размер изображения с логотипом можно
% в~соответствующих местах файлов \verb!title.tex!  отдельно для
% диссертации и автореферата. Если вам логотип не~нужен, то просто
% удалите файл с~логотипом.

% \ifsynopsis
% Этот абзац появляется только в~автореферате.
% Для формирования блоков, которые будут обрабатываться только в~автореферате,
% заведена проверка условия \verb!\!\verb!ifsynopsis!.
% Значение условия задаётся в~основном файле документа (\verb!synopsis.tex! для
% автореферата).
% \else
% Этот абзац появляется только в~диссертации.
% Через проверку условия \verb!\!\verb!ifsynopsis!, задаваемого в~основном файле
% документа (\verb!dissertation.tex! для диссертации), можно сделать новую
% команду, обеспечивающую появление цитаты в~диссертации, но~не~в~автореферате.
% \fi

% {\progress}
% Этот раздел должен быть отдельным структурным элементом по
% ГОСТ, но он, как правило, включается в описание актуальности
% темы. Нужен он отдельным структурынм элемементом или нет ---
% смотрите другие диссертации вашего совета, скорее всего не нужен.

{\aim} данной работы является исследование возможностей злоумышленника по получению секретного ключа с использованием атак на измерительное оборудование систем квантовой коммуникации на боковых частотах и разработка методов противодействия атакам.


Для~достижения поставленной цели необходимо было решить следующие {\tasks}:
\begin{enumerate}
  \item Исследование устойчивости детектора одиночных фотонов, применяемого в системах квантовой коммуникации на боковых частотах, к атакам с выведением из режима Гейгера (<<ослеплением>>). 

  \item Оценка возможностей злоумышленника при атаке с выведением из режима Гейгера для систем квантовой коммуникации на боковых частотах. 

  \item Разработка оптической схемы системы квантовой коммуникации, устойчивой к атакам на измерительное оборудование. 

  \item Разработка протокола квантовой рассылки ключа, устойчивого к атаке на измерительное оборудование. 

\end{enumerate}


{\novelty}
\begin{enumerate}
  \item Впервые исследована устойчивость системы квантовой коммуникации на боковых частотах к атакам злоумышленника на измерительное оборудование приёмного блока. 
  \item Впервые предложена и применена контрмера против атаки злоумышленника с выведением детектора из режима Гейгера и навязыванием легитимным пользователям ключа. 
  \item Было выполнено оригинальное исследование и разработана оптическая схема системы квантовой коммуникации на боковых частотах с вынесением устройства детектирования одиночных фотонов из приемного блока в <<недоверенный>> узел, подконтрольный злоумышленнику 
\end{enumerate}

{\influence} 

Разработанные методы и подход позволили однозначно определить уязвимость коммерчески доступных детекторов компании id Quantique модели id210 к выведению из режима Гейгера оптическими средствами. В связи с чем доказана необходимость применения дополнительных мер защиты от атак на измерительный узел. Предложена схема, позволяющая производить активный мониторинг попыток <<ослепить>> детектор, благодаря использованию особенностей систем квантовой коммуникации на боковых частотах. Результаты внедрены в производство ООО "Кванттелеком". 

%  {\methods} \todo{TO DO} \ldots

%%%%%	{\defpositions}
%%%%%	\begin{enumerate}
%%%%%  \item Использование коммерческих детекторов одиночных фотонов на основе лавинных фотодиодов в режиме Гейгера модели id210 с частотой стробирования 100 МГц  требует применения дополнительных средств защиты от атаки с выведением из режима Гейгера при помощи коротких оптических импульсов с энергией не менее 15,4 фДж и при постоянном уровне оптической засветки средним уровнем мощности излучения не менее 35 нВт.  
%%%%%  \item Измерение величины оптического излучения на несущей частоте, отраженного от оптического фильтра, при помощи мониторного фотодиода в приемном блоке системы квантовой коммуникации на боковых частотах в диапазоне от 7 нВт до 2,93 мкВт с применением дополнительных мер в виде пассивного оптического аттенюатора номиналом 10 дБ для его защиты позволяет противостоять атаке с выведением детектора одиночных фотонов из режима Гейгера и навязыванием ключа нелегитимным пользователем. 
%%%%%  \item Метод квантовой коммуникации на боковых частотах позволяет реализовывать протокол, устойчивый к контролю нелегитимным пользователем измерительного оборудования. 
%%%%%  \item В результате интерференции квантового фазомодулированного сигнала на боковых частотах на симметричном светоделителе в схеме квантовой рассылки ключа с узлом регистрации, независящим от легитимного пользователя, происходит спектральное разделение квантового сигнала и сигнала на центральной длине волны с их независимой регистрацией в разных плечах светоделителя. 
%%%%%%%  \item Показана возможность двукратного увеличения дальности квантовой рассылки ключа на боковых частотах посредством применения недоверенной системы регистрации квантовых состояний. 
%%%%%%%%\end{enumerate}
% В папке Documents можно ознакомиться в решением совета из Томского ГУ
% в~файле \verb+Def_positions.pdf+, где обоснованно даются рекомендации
% по~формулировкам защищаемых положений.


{\defpositions}
\begin{enumerate}

  \item Детектор одиночных фотонов на основе лавинного фотодиода с частотой стробирования 100 МГц  может быть выведен из режима Гейгера при постоянном уровне оптической засветки средним уровнем мощности излучения не менее 35 нВт и регистрировать срабатывания посредством воздействия на него оптических импульсов с энергией более 15,4 фДж независимо от частоты следования этих импульсов
  \item Преобразование частоты контролирующего сигнала посредством фазовой модуляции на радиочастотной моде позволяет обнаружить атаку с поддельными состояниями и навязыванием срабатываний модулю получателя системы квантовой коммуникации.
  \item Применение слабых когерентных многомодовых состояний с фазовым кодированием позволяет реализовывать протокол, устойчивый к контролю нелегитимным пользователем измерительного оборудования, в том числе с вынесением за пределы приемного модуля. 
  \item В результате интерференции квантовых сигналов на боковых частотах, сформированных в результате  фазовой модуляции двух независимых абонентов системы квантовой коммуникации на симметричном светоделителе в составе узла регистрации, независимого от легитимных пользователей, происходит спектральное разделение квантового сигнала и сигнала на центральной длине волны с их независимой регистрацией в разных плечах светоделителя. 
  %  \item Показана возможность двукратного увеличения дальности квантовой рассылки ключа на боковых частотах посредством применения недоверенной системы регистрации квантовых состояний. 
 
\end{enumerate}

{\reliability} полученных результатов обеспечивается применением утверждённых методик проведений экспериментальных исследований и аттестованного оборудование. Математическое моделирование и обработка данных, полученных в результате экспериментов, осуществлялось с использованием пакетов прикладных программ MathCad и Excel. Результаты находятся в соответствии с результатами, полученными другими авторами.


{\probation}
Основные результаты работы докладывались~на:
%перечисление основных конференций, симпозиумов и~т.\:п. 
\begin{enumerate}
	\item ICQOQI 2019, Минск, Беларусь, 13 - 17 мая 2019
	\item XLVIII научная и учебно-методическая конференция Университета~ИТМО, Санкт-Петербург, Россия, 29 января - 1 февраля 2019
	\item QCrypt 2018, Шанхай, Китай, 27 - 31 августа 2018
	\item 18th International Conference on Laser Optics ICLO 2018, Санкт-Петербург, Россия, 4 - 8 июня 2018
	\item VII Всероссийский конгресс молодых ученых, Санкт-Петербург, Россия, 17 - 20 апреля 2018
	\item XLVII научная и учебно-методическая конференция Университета ИТМО, Санкт-Петербург, Россия, 30 января - 2 февраля 2018
\end{enumerate}

% {\contribution} Автор принимал активное участие \todo{TO DO} \ldots

%\publications\ Основные результаты по теме диссертации изложены в ХХ печатных изданиях~\cite{Sokolov,Gaidaenko,Lermontov,Management},
%Х из которых изданы в журналах, рекомендованных ВАК~\cite{Sokolov,Gaidaenko},
%ХХ --- в тезисах докладов~\cite{Lermontov,Management}.

\ifnumequal{\value{bibliosel}}{0}{% Встроенная реализация с загрузкой файла через движок bibtex8
    \publications\ Основные результаты по теме диссертации изложены в XX печатных изданиях,
    X из которых изданы в журналах, рекомендованных ВАК,
    X "--- в тезисах докладов.%
}{% Реализация пакетом biblatex через движок biber
%Сделана отдельная секция, чтобы не отображались в списке цитированных материалов
    \begin{refsection}[vak,wos,scopus,papers,conf]% Подсчет и нумерация авторских работ. Засчитываются только те, которые были прописаны внутри \nocite{}.
        %Чтобы сменить порядок разделов в сгрупированном списке литературы необходимо перетасовать следующие три строчки, а также команды в разделе \newcommand*{\insertbiblioauthorgrouped} в файле biblio/biblatex.tex
        \printbibliography[heading=countauthorvak, env=countauthorvak, keyword=biblioauthorvak, section=1]%
        \printbibliography[heading=countauthorwos,env=countauthorwos, keyword=biblioauthorwos, section=1]%
        \printbibliography[heading=countauthorscopus,env=countauthorscopus, keyword=biblioauthorscopus, section=1]%
	\printbibliography[heading=countauthorconf, env=countauthorconf, keyword=biblioauthorconf, section=1]%
        \printbibliography[heading=countauthorothers, env=countauthorothers, keyword=biblioauthorothers, section=1]%
        \printbibliography[heading=countauthor, env=countauthor, keyword=biblioauthor, section=1]%
        \nocite{%Порядок перечисления в этом блоке определяет порядок вывода в списке публикаций автора
                JOT,Chistyakov_2016,%
		Chistiakov:19,%
		scbib1,%
                confbib1,confbib2,%
                bib1,bib2,%
        }%
	\publications\ Основные результаты по теме диссертации изложены в 13 печатных изданиях. 10 из которых изданы в журналах, рекомендованных ВАК, 3 в тезисах докладов. 
	
	\section*{Работы автора по теме диссертации}
{Статьи в журналах, рекомендованных ВАК: }
\begin{enumerate}\addtolength{\itemsep}{-0.5\baselineskip}
\renewcommand{\labelenumi}{[\theenumi]}
\item Vladimir Chistiakov, Anqi Huang, Vladimir Egorov, and Vadim Makarov, Controlling single-photon detector ID210 with bright light, Opt. Express 27, 32253-32262 (2019)
\\
\item Чистяков В.В., Гайдаш А.А., Козубов А.В., Глейм А.В. Исследование интерференции слабых когерентных многомодовых состояний для задач квантовой коммуникации с недоверенным приемным узлом // Научно-технический вестник информационных технологий, механики и оптики. 2019. Т. 19. № 6. doi: 10.17586/2226-1494-2019-19-6
\\
\item    Gleim A.V., Egorov V.I., Nazarov Y.V., Smirnov S.V., Chistyakov V.V., Bannik O.I., Anisimov A.A., Kynev S.M., Ivanova A.E., Collins R.J., Kozlov S.A., Buller G. Secure polarization-independent subcarrier quantum key distribution in optical fiber channel using BB84 protocol with a strong reference//Optics express, IET - 2016, Vol. 24, No. 3, pp. 2619-2633
\\
\item  Глейм А.В., Егоров В.И., Чистяков В.В., Смирнов С.В., Банник О.И., Булдаков Н.В., Гайдаш А.А., Козубов А.В., Васильев А.Б., Кынев С.М., Хоружников С.Э., Козлов С.А., Васильев В.Н. Квантовая коммуникация на боковых частотах со скоростью 1 Мбит/с в городской сети // Оптический журнал -2017. - Т. 84. - № 6. - С. 3-9
\\
\item  Chistyakov V.V., Kynev S.M, Smirnov S.V., Nazarov Y.V., Gleim A.V. Achieving high visibility in subcarrier wave quantum key distribution system // Journal of Physics: Conference Series, IET - 2016, Vol. 735, No. 1, pp. 012085
\\
\item V. V. Chistyakov, A. V. Gleim, V. I. Egorov, Yu. V. Nazarov. Implementation of multiplexing in a subcarrier-wave quantum cryptography system // Journal of Physics: Conference Series - 2014  vol. 541,  pp. 012078
\\
\item   Kynev S.M., Chistyakov V.V., Smirnov S.V., Volkova K.P., Egorov V.I., Gleim A.V. Free-space subcarrier wave quantum communication // Journal of Physics: Conference Series - 2017, Vol. 917, No. 5, pp. 052003
\\

\item    Gleim A.V., Nazarov Y.V., Egorov V.I., Smirnov S.V., Bannik O.I., Chistyakov V.V., Kynev S.M., Anisimov A.A., Kozlov S.A., Vasil'ev V.N. Subcarrier Wave Quantum Key Distribution in Telecommunication Network with Bitrate 800 kbit/s//EPJ Web of Conferences, IET - 2015, Vol. 103, pp. 10005
\\
\item    Gleim A.V., Egorov V.I., Nazarov Y.V., Smirnov S.V., Chistyakov V.V., Bannik O.I., Anisimov A.A., Kynev S.M., Collins R.J., Kozlov S.A., Buller G.S. Polarization insensitive 100 MHz clock subcarrier quantum key distribution over a 45 dB loss optical fiber channel // Conference on Lasers and Electro-Optics, CLEO 2015, IET - 2015, pp. 7182997
\\
\item Gaidash A.A., Kozubov A.V., Chistyakov V.V., Miroshnichenko G.P., Egorov V.I., Gleim A.V. Security conditions for sub-carrier wave quantum key distribution protocol in errorless channel // Journal of Physics: Conference Series - 2017, Vol. 917, No. 6, pp. 062014
\\
%\item  Gleim A.V., Chistyakov V.V., Bannik O.I., Egorov V.I., Buldakov N.V., Vasilev A.B., Gaidash A.A., Kozubov A.V., Smirnov S.V., Kynev S.M., Khoruzhnikov S.E., Kozlov S.A., Vasil'ev V.N. Sideband quantum communication at 1 Mbit/s on a metropolitan area network // Journal of Optical Technology - 2017, Vol. 84, No. 6, pp. 362-36
\\

\end{enumerate}
\noindent{ Другие публикации: }
\begin{enumerate}\addtolength{\itemsep}{-0.5\baselineskip}
\renewcommand{\labelenumi}{[\theenumi]}
\setcounter{enumi}{9}
\item   Чистяков В.В., Кынев С.М., Смирнов С.В., Назаров Ю.В., Глейм А.В. Обеспечение высокой видности в системе квантовой криптографии на боковых частотах // Сборник трудов IX международной конференции молодых ученых и специалистов «Оптика – 2015», с. 658-660
\\
\item Глейм А.В., Назаров Ю.В., Егоров В.И., Чистяков В.В, Смирнов С.В., Банник О.И., Кынев С.М., Иванова А.Е., Дубровская В.Д., Тарасов М.Г., Булдаков Н.В., Кузьмина Т.Б., Чивилихин С.А., Анисимов А.А., Рощупкин С.В., Рогачёв К.С., Хоружников С.Э., Козлов С.А., Васильев В.Н. Создание квантовой сети университета ИТМО //Сборник трудов VIII международной конференции «Фундаментальные проблемы оптики – 2014». Санкт-Петербург, 20-24 октября ,2014, С.3-4,  541 с. 
\\
\item А.В. Глейм, В.И.Егоров, А.А. Анисимов, Ю.В. Назаров, С.М. Кынев, А.В. Рупасов, В.В. Чистяков, А.А.Гайдаш, М.А. Смирнов, С.А. Чивилихин, С.А. Козлов Квантовая рассылка криптографического ключа по оптическому волокну телекоммуникационного стандарта на расстояние 200 км со скоростью 0.18 кбит/с // Cборник трудов III Всероссийская конференция по фотонике и информационной оптике Москва, НИЯУ МИФИ, 2014 с. 17-19
\\


\end{enumerate}

	
%	\setcounter{citeauthorscwostot}{\value{citeauthorscopus}} % вместе setcounter и addtocounter добавляют пробел между словами. По-этому они так раскиданы.
%        в~\arabic{citeauthor}~печатных изданиях,
%	\addtocounter{citeauthorscwostot}{\value{citeauthorwos}}
%	\arabic{citeauthorvak} из которых изданы в журналах, рекомендованных ВАК\sloppy
%	\ifnum \value{citeauthorscwostot}>0
%	, \arabic{citeauthorscwostot} "--- в~периодических научных журналах, индексируемых Web of Science и Scopus\sloppy
%	\fi
%	\ifnum \value{citeauthorconf}>0
%	, \arabic{citeauthorconf} "--- в~тезисах докладов.
%	\else
%	.
%	\fi
%    \end{refsection}
%    \begin{refsection}[vak,wos,scopus,papers,conf]%Блок, позволяющий отобрать из всех работ автора наиболее значимые, и только их вывести в автореферате, но считать в блоке выше общее число работ
%        \printbibliography[heading=countauthorvak, env=countauthorvak, keyword=biblioauthorvak, section=2]%
%        \printbibliography[heading=countauthorwos, env=countauthorwos, keyword=biblioauthorwos, section=2]%
 %       \printbibliography[heading=countauthorscopus, env=countauthorscopus, keyword=biblioauthorscopus, section=2]%
  %      \printbibliography[heading=countauthorothers, env=countauthorothers, keyword=biblioauthorothers, section=2]%
  %      \printbibliography[heading=countauthorconf, env=countauthorconf, keyword=biblioauthorconf, section=2]%
   %     \printbibliography[heading=countauthor, env=countauthor, keyword=biblioauthor, section=2]%
        
    %    \nocite{Chistyakov_2016}%vak
    %    \nocite{Gleim:15}
    %    \nocite{Gleim:16}
    %    \nocite{Gleĭm:17}
        
     %   \nocite{bib1}%other
     %   \nocite{confbib1}%conf
   \end{refsection}
}
% При использовании пакета \verb!biblatex! для автоматического подсчёта
% количества публикаций автора по теме диссертации, необходимо
% их~здесь перечислить с использованием команды \verb!\nocite!.
