% Новые переменные, которые могут использоваться во всём проекте
% ГОСТ 7.0.11-2011
% 9.2 Оформление текста автореферата диссертации
% 9.2.1 Общая характеристика работы включает в себя следующие основные структурные
% элементы:
% актуальность темы исследования;
\newcommand{\actualityTXT}{Актуальность темы.}
% \newcommand{\relevance}{\underline{\textbf{\relevanceTXT}}}
\newcommand{\relevanceTXT}{Relevance.}
% степень ее разработанности;
\newcommand{\progressTXT}{Степень разработанности темы.}
% цели и задачи;
\newcommand{\aimTXT}{Целью}
% \newcommand{\the_goalTXT}{The goal.}
\newcommand{\tasksTXT}{задачи}
% \newcommand{\scientific_tasksTXT}{Scientific tasks.}
% научную новизну;
\newcommand{\noveltyTXT}{Научная новизна:}
% \newcommand{\scientific_noveltyTXT}{Scientific novelty}
% теоретическую и практическую значимость работы;
% \newcommand{\influenceTXT}{Теоретическая и практическая значимость}
% или чаще используют просто
\newcommand{\influenceTXT}{Практическая значимость}
% \newcommand{\practical_importanceTXT}{Practical importance}
% методологию и методы исследования;
\newcommand{\methodsTXT}{Методология и методы исследования.}
% положения, выносимые на защиту;
\newcommand{\defpositionsTXT}{Основные положения, выносимые на~защиту:}
% степень достоверности и апробацию результатов.
\newcommand{\reliabilityTXT}{Достоверность}
\newcommand{\probationTXT}{Апробация работы.}

\newcommand{\contributionTXT}{Личный вклад.}
\newcommand{\publicationsTXT}{Публикации.}


\newcommand{\authorbibtitle}{Публикации автора по теме диссертации}
\newcommand{\vakbibtitle}{В изданиях из списка ВАК РФ}
\newcommand{\notvakbibtitle}{В прочих изданиях}
\newcommand{\confbibtitle}{В сборниках трудов конференций}
\newcommand{\fullbibtitle}{Список литературы} % (ГОСТ Р 7.0.11-2011, 4)

% выделение цветом персональных комментарий
\definecolor{vcc}{RGB}{0,0,179}
\newcommand{\VCc}[1]{\textcolor{vcc}{[VC: #1]}} % Владимир Чистяков

