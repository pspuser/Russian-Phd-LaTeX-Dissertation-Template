
  {\relevance} 
  Quantum informatics is a field of knowledge that combines elements of the photonics and the theory of information. Quantum bits, or qubits, are used as the basic units of quantum information science. Qubit is a system that can be in a superposition state and used to store, calculate and transmit information. One of the promising areas of quantum informatics are a quantum calculations and a quantum memory -- the main components of a quantum computer -- a totally new type of computing device that works on the fundamental principles of quantum mechanics. The transmission of a quantum state at various distances, or quantum teleportation, is another promising application of quantum informatics. The generation of a symmetric bit sequence by the quantum methods, or quantum key distribution (QKD), was formed as a scientific direction in the 80's of the 20th century. This direction is advanced in terms of application. In quantum communication systems (QCS), through the use of quantum states, two or more legitimate users can distribute symmetric bit sequences, which can subsequently be used as keys for encoding information, so that any eavesdropping attempts of the communication channel will be detected by the increased error rate.
  
Differences between physical implementations of QCS from ideal models used in theoretical approaches can be the basis for conducting various types of attacks on equipment used in this systems. It was previously shown that gated single-photon detectors (SPD) of several commercially available QKD systems were vulnerable to attack by the eavesdropper. It is noteworthy that exposure can be carried out even by optical methods. Eavesdropping can be made directly from a quantum channel that connects legitimate users. In practice, optical fiber is most often used as a medium for the transferring of quantum states. Single photons and their properties are usually used for that purposes.
	
This type of attack is called a <<faked-states attack>> (FSA). It is based on the fact that for the detection of photons, a detector based on avalanche photodiodes (APDs) operating in the Geiger mode, or photon counting mode, is used. With this approach, in the absence of additional protective measures, the attacker has the ability to use the powerful constant optical radiation to carry the detector out from the Geiger mode to the linear mode (so-called <<blinding>>) and provoke triggering due to the use of powerful short pulses. Thus, the feasibility of complete control of the single-photon registration node, became critical for QKD protocols, and as a result, leads the eavesdropper to imperceptible obtaining of the full key correlating with that of legitimate users.
	

There are several ways to counter this type of attack, but in practice most of them have not been tested in quantum hacking laboratories. Several countermeasures that were analyzed and investigated have included counterattacks and hacking methods that were not taken into account by the developers of countermeasures.
	
The most effective countermeasure against the well-known types of attacks on detector side is the implementation of a  quite new QKD architecture that is resistant to attacks on measuring equipment (Measurement-Device-Independent, or MDI). With this approach, it is initially assumed that the single-photon registration unit is moved outside the sender and receiver blocks and is completely accessible to the eavesdropper. However, such an approach is characterized by a higher level of complexity of implementing the optical scheme and lower characteristics with respect to QKD systems with a point-to-point topology with a registration node in one of the blocks.
	
One of the promising approaches is the method of quantum communication at subcarrier-wave quantum key distribution (SCW~QKD). Its distinctive feature is the carrying of the quantum channel out to the sidebands generated as a result of modulation. This ensures high resistance to external factors and spectral efficiency, as well as good performance in the ratio of the key formation speed to the distance between the sender and receiver units. However, as part of this type of system for small and medium distances (up to 100~km), single-photon detectors based on the APD are also used, and the resistance to attacks on the measuring equipment of SCW~QKD systems has not been investigated.
	
	
% Обзор, введение в тему, обозначение места данной работы в
% мировых исследованиях и~т.\:п., можно использовать ссылки на~другие
% работы\ifnumequal{\value{bibliosel}}{1}{~\autocite{Gosele1999161}}{}
% (если их~нет, то~в~автореферате
% автоматически пропадёт раздел <<Список литературы>>). Внимание! Ссылки
% на~другие работы в разделе общей характеристики работы можно
% использовать только при использовании \verb!biblatex! (из-за технических
% ограничений \verb!bibtex8!. Это связано с тем, что одна
% и~та~же~характеристика используются и~в~тексте диссертации, и в
% автореферате. В~последнем, согласно ГОСТ, должен присутствовать список
% работ автора по~теме диссертации, а~\verb!bibtex8! не~умеет выводить в одном
% файле два списка литературы).
% При использовании \verb!biblatex! возможно использование исключительно
% в~автореферате подстрочных ссылок
% для других работ командой \verb!\autocite!, а~также цитирование
% собственных работ командой \verb!\cite!. Для этого в~файле
% \verb!Synopsis/setup.tex! необходимо присвоить положительное значение
% счётчику \verb!\setcounter{usefootcite}{1}!.
% 
% Для генерации содержимого титульного листа автореферата, диссертации
% и~презентации используются данные из файла \verb!common/data.tex!. Если,
% например, вы меняете название диссертации, то оно автоматически
% появится в~итоговых файлах после очередного запуска \LaTeX. Согласно
% ГОСТ 7.0.11-2011 <<5.1.1 Титульный лист является первой страницей
% диссертации, служит источником информации, необходимой для обработки и
% поиска документа>>. Наличие логотипа организации на~титульном листе
% упрощает обработку и~поиск, для этого разметите логотип вашей
% организации в папке images в~формате PDF (лучше найти его в векторном
% варианте, чтобы он хорошо смотрелся при печати) под именем
% \verb!logo.pdf!. Настроить размер изображения с логотипом можно
% в~соответствующих местах файлов \verb!title.tex!  отдельно для
% диссертации и автореферата. Если вам логотип не~нужен, то просто
% удалите файл с~логотипом.

% \ifsynopsis
% Этот абзац появляется только в~автореферате.
% Для формирования блоков, которые будут обрабатываться только в~автореферате,
% заведена проверка условия \verb!\!\verb!ifsynopsis!.
% Значение условия задаётся в~основном файле документа (\verb!synopsis.tex! для
% автореферата).
% \else
% Этот абзац появляется только в~диссертации.    ----- условие выполняется как else в обоих случаях (для диссертации и реферата)
% Через проверку условия \verb!\!\verb!ifsynopsis!, задаваемого в~основном файле
% документа (\verb!dissertation.tex! для диссертации), можно сделать новую
% команду, обеспечивающую появление цитаты в~диссертации, но~не~в~автореферате.
% \fi

% {\progress}
% Этот раздел должен быть отдельным структурным элементом по
% ГОСТ, но он, как правило, включается в описание актуальности
% темы. Нужен он отдельным структурынм элемементом или нет ---
% смотрите другие диссертации вашего совета, скорее всего не нужен.

{\goal} of this dissertation is to investigate a feasibility of the secret key eavesdropping by detector side channel attacks of SCW~QKD systems and to develop countermeasure methods.


{\scientifictasks}:
\begin{enumerate}
  \item Investigate vulnerability to blinding of the SPD used in SCW~QKD.  

  \item Estimate the eavesdropper limits in case of successfully controlling detector side. 

  \item Develop an optical scheme for monitoring possible eavesdropping.

  \item Investigate a measurement-device independent QKD protocol based on multi-mode weak coherent states used in SCW~QKD. 
\end{enumerate}


{\scientificnovelty} is specified by the following new results:
\begin{enumerate}
  \item It is experimentally demonstrated the vulnerability of SCW~QKD to faked-state attacks.
  \item For the first time it is shown that the SCW~QKD basic optical scheme provides an advantage and could be updated for monitoring the bright light.
  \item It is proposed and experimentally studied the feasibility of twin-field quantum key distribution based on multi-mode coherent phase-coded states. The detection node is moved away to untrusted relay. The nontrivial interference is obtained. Key rate estimation shows that SCW approach can beat well-known fundamental limits of repeaterless quantum communications (linear bound).
\end{enumerate}

{\statements}
\begin{enumerate}

  \item Single-photon detector based on avalanche photodiode with 100~MHz external gating could be moved out from photon-counting mode by bright light illumination no less than 35 nW. It could be triggered by optical pulses with energy more than 15,4 fJ regardless of their repetition rate.
  \item The frequency conversion of the control signal by phase modulation on the radio frequency mode allows to detect a faked-state attack by monitoring a carrier's frequency.
  \item The use of weak coherent multi-mode states with phase coding makes it possible to implement a protocol that is resistant to control by an illegitimate user of measuring device.
  \item In result of interference of quantum states at sidebands on a symmetrical beam splitter placed in untrusted detection node, the spectral separation of the quantum signal and the signal at the central wavelength occurs with their independent registration in different beamsplitter outputs. 
  %  \item Double distance increasing is performed by using twin-field approach for multi-mode weak coherent phase-coded states. 
  %  \item Показана возможность двукратного увеличения дальности квантовой рассылки ключа на боковых частотах посредством применения недоверенной системы регистрации квантовых состояний. 
 
\end{enumerate}

{\importance} of the dissertation lies in a possibility to implement a next generation of QKD system using subcarrier-wave approach overcoming the linear bound of key rate generation and to achieve the benefits of MDI protocol invulnerable to detector side attacks.

%  {\methods} \todo{TO DO} \ldots

%%%%%	{\defpositions}
%%%%%	\begin{enumerate}
%%%%%  \item Использование коммерческих детекторов одиночных фотонов на основе лавинных фотодиодов в режиме Гейгера модели id210 с частотой стробирования 100 МГц  требует применения дополнительных средств защиты от атаки с выведением из режима Гейгера при помощи коротких оптических импульсов с энергией не менее 15,4 фДж и при постоянном уровне оптической засветки средним уровнем мощности излучения не менее 35 нВт.  
%%%%%  \item Измерение величины оптического излучения на несущей частоте, отраженного от оптического фильтра, при помощи мониторного фотодиода в приемном блоке системы квантовой коммуникации на боковых частотах в диапазоне от 7 нВт до 2,93 мкВт с применением дополнительных мер в виде пассивного оптического аттенюатора номиналом 10 дБ для его защиты позволяет противостоять атаке с выведением детектора одиночных фотонов из режима Гейгера и навязыванием ключа нелегитимным пользователем. 
%%%%%  \item Метод квантовой коммуникации на боковых частотах позволяет реализовывать протокол, устойчивый к контролю нелегитимным пользователем измерительного оборудования. 
%%%%%  \item В результате интерференции квантового фазомодулированного сигнала на боковых частотах на симметричном светоделителе в схеме квантовой рассылки ключа с узлом регистрации, независящим от легитимного пользователя, происходит спектральное разделение квантового сигнала и сигнала на центральной длине волны с их независимой регистрацией в разных плечах светоделителя. 
%%%%%%%  \item Показана возможность двукратного увеличения дальности квантовой рассылки ключа на боковых частотах посредством применения недоверенной системы регистрации квантовых состояний. 
%%%%%%%%\end{enumerate}
% В папке Documents можно ознакомиться в решением совета из Томского ГУ
% в~файле \verb+Def_positions.pdf+, где обоснованно даются рекомендации
% по~формулировкам защищаемых положений.

{\validity} of the results obtained are provided by the use of modern methods of experimental research and confirming by measurements on certified equipment. Mathematical modeling and processing of data obtained as a result of experiments was carried out using application packages MathCad and Excel. The experimental data are in agreement with theoretical results.

{\implementation} A scheme is proposed that allows for active monitoring of attempts to <<blind>> the detector, using the features of SCW QKD system. The results are implemented in the production of LLC "Quanttelecom".

{\approbation} The main results of this work were presented and discussed at the following conferences:
%перечисление основных конференций, симпозиумов и~т.\:п. 
\begin{enumerate}
	\item ICQOQI 2019, Minsk, Belarus, 2019
	\item XLVIII scientific and educational conference of ITMO University, St. Petersburg, Russia, 2019
	\item QCrypt 2018, Shanghai, China, 2018
	\item 18th International Conference on Laser Optics ICLO 2018, St. Petersburg, Russia, 2018
	\item VII All-Russian Congress of Young Scientists, St. Petersburg, Russia, 2018
	\item XLVII scientific and educational conference of ITMO University, St. Petersburg, Russia, 2018
\end{enumerate}

% {\contribution} Автор принимал активное участие \todo{TO DO} \ldots

%\publications\ Основные результаты по теме диссертации изложены в ХХ печатных изданиях~\cite{Sokolov,Gaidaenko,Lermontov,Management},
%Х из которых изданы в журналах, рекомендованных ВАК~\cite{Sokolov,Gaidaenko},
%ХХ --- в тезисах докладов~\cite{Lermontov,Management}.

\ifnumequal{\value{bibliosel}}{0}{% Встроенная реализация с загрузкой файла через движок bibtex8
    \publications\ Основные результаты по теме диссертации изложены в XX печатных изданиях,
    X из которых изданы в журналах, рекомендованных ВАК,
    X "--- в тезисах докладов.%
}{% Реализация пакетом biblatex через движок biber
%Сделана отдельная секция, чтобы не отображались в списке цитированных материалов
    \begin{refsection}[vak,wos,scopus,papers,conf]% Подсчет и нумерация авторских работ. Засчитываются только те, которые были прописаны внутри \nocite{}.
        %Чтобы сменить порядок разделов в сгрупированном списке литературы необходимо перетасовать следующие три строчки, а также команды в разделе \newcommand*{\insertbiblioauthorgrouped} в файле biblio/biblatex.tex
        \printbibliography[heading=countauthorvak, env=countauthorvak, keyword=biblioauthorvak, section=1]%
        \printbibliography[heading=countauthorwos,env=countauthorwos, keyword=biblioauthorwos, section=1]%
        \printbibliography[heading=countauthorscopus,env=countauthorscopus, keyword=biblioauthorscopus, section=1]%
	\printbibliography[heading=countauthorconf, env=countauthorconf, keyword=biblioauthorconf, section=1]%
        \printbibliography[heading=countauthorothers, env=countauthorothers, keyword=biblioauthorothers, section=1]%
        \printbibliography[heading=countauthor, env=countauthor, keyword=biblioauthor, section=1]%
        \nocite{%Порядок перечисления в этом блоке определяет порядок вывода в списке публикаций автора
                JOT,Chistyakov_2016,%
		Chistiakov:19,%
		scbib1,%
                confbib1,confbib2,%
                bib1,bib2,%
        }%
	{\refs} The main content of the dissertation is published in 13 articles. 10 of them are indexed in Web of Science and/or Scopus. 3 of them are conference theses.
	
%	\section*{Работы автора по теме диссертации}
	
%	\setcounter{citeauthorscwostot}{\value{citeauthorscopus}} % вместе setcounter и addtocounter добавляют пробел между словами. По-этому они так раскиданы.
%        в~\arabic{citeauthor}~печатных изданиях,
%	\addtocounter{citeauthorscwostot}{\value{citeauthorwos}}
%	\arabic{citeauthorvak} из которых изданы в журналах, рекомендованных ВАК\sloppy
%	\ifnum \value{citeauthorscwostot}>0
%	, \arabic{citeauthorscwostot} "--- в~периодических научных журналах, индексируемых Web of Science и Scopus\sloppy
%	\fi
%	\ifnum \value{citeauthorconf}>0
%	, \arabic{citeauthorconf} "--- в~тезисах докладов.
%	\else
%	.
%	\fi
%    \end{refsection}
%    \begin{refsection}[vak,wos,scopus,papers,conf]%Блок, позволяющий отобрать из всех работ автора наиболее значимые, и только их вывести в автореферате, но считать в блоке выше общее число работ
%        \printbibliography[heading=countauthorvak, env=countauthorvak, keyword=biblioauthorvak, section=2]%
%        \printbibliography[heading=countauthorwos, env=countauthorwos, keyword=biblioauthorwos, section=2]%
 %       \printbibliography[heading=countauthorscopus, env=countauthorscopus, keyword=biblioauthorscopus, section=2]%
  %      \printbibliography[heading=countauthorothers, env=countauthorothers, keyword=biblioauthorothers, section=2]%
  %      \printbibliography[heading=countauthorconf, env=countauthorconf, keyword=biblioauthorconf, section=2]%
   %     \printbibliography[heading=countauthor, env=countauthor, keyword=biblioauthor, section=2]%
        
    %    \nocite{Chistyakov_2016}%vak
    %    \nocite{Gleim:15}
    %    \nocite{Gleim:16}
    %    \nocite{Gleĭm:17}
        
     %   \nocite{bib1}%other
     %   \nocite{confbib1}%conf
   \end{refsection}
}
% При использовании пакета \verb!biblatex! для автоматического подсчёта
% количества публикаций автора по теме диссертации, необходимо
% их~здесь перечислить с использованием команды \verb!\nocite!.
