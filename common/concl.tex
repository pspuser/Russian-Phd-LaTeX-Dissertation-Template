%% Согласно ГОСТ Р 7.0.11-2011:
%% 5.3.3 В заключении диссертации излагают итоги выполненного исследования, рекомендации, перспективы дальнейшей разработки темы.
%% 9.2.3 В заключении автореферата диссертации излагают итоги данного исследования, рекомендации и перспективы дальнейшей разработки темы.
\begin{enumerate}
  \item На основе экспериментального анализа детектора, работающего в режиме Гейгера, показано, что требуются дополнительные средства защиты от атаки с выведением из режима Гейгера при помощи коротких оптических импульсов с энергией не менее 15,4 фДж и при постоянном уровне оптической засветки средним уровнем мощности излучения не менее 35 нВт. 
  \item Численные исследования показали, что измерение величины оптического излучения на несущей частоте, отраженного от оптического фильтра, при помощи мониторного фотодиода в приемном блоке системы квантовой коммуникации на боковых частотах в диапазоне от 7 нВт до 2,93 мкВт с применением дополнительных мер в виде пассивного оптического аттенюатора номиналом 10 дБ для его защиты позволяет противостоять атаке с выведением детектора одиночных фотонов из режима Гейгера и навязыванием ключа нелегитимным пользователем. 
  \item Метод квантовой коммуникации на боковых частотах позволяет реализовывать протокол, устойчивый к контролю нелегитимным пользователем измерительного оборудования.
  \item Для выполнения поставленных задач был создан экспериментальный стенд и в результате интерференции квантового фазомодулированного сигнала на боковых частотах на симметричном светоделителе в схеме квантовой рассылки ключа с узлом регистрации, независящим от легитимного пользователя, наблюдается спектральное разделение квантового сигнала и сигнала на центральной длине волны с их независимой регистрацией в разных плечах светоделителя. 
\end{enumerate}