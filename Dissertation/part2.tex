\chapter{Исследование устойчивости системы квантовой коммуникации на боковых частотах к атакам на измерительное оборудование}  \label{ch:ch2}


\section{Детектор в составе системы} \label{sec:ch2/sec1}

В зависимости от поставленных целей и решаемых задач в состав систем квантовой коммуникации включают в основном два типа детекторов одиночных фотонов, сравнительная характеристика которых дана в главе \ref{sec:ch1/sec5}. Для магистральных дистанций от 100~км, либо для обеспечения высокой скорости формирования квантовых кодирующих последовательностей применяют сверхпроводниковые ДОФ (SNSPD). Для внутригородских дистанций до 100~км с потерями в линиях связи менее 15~дБ обычно применяют более компактный, простой в использовании ДОФ на базе лавинного фотодиода (SPAD), однако его типовые характеристики на порядок уступают характеристикам сверхпроводникового ДОФ. 


В СКК на боковых частотах применяется коммерчески доступный детектор модели ID210, разработанный компанией idQuantique. Его отличительными особенностями являются:
\begin{enumerate}
	\item Поддержка высокой частоты стробирующих импульсов - до 100~МГц
	\item Возможность подачи стробирующих импульсов от внешнего устройства (External gating mode)
	\item Широкий диапазон настройки ширины окна срабатывания (gate) - от 0,5~нс до 25~нс
	\item Выставление задержки открытия окна срабатывания относительно стробирующего импульса (Trigger delay) в диапазоне до 10~нс с высоким разрешением во времени - 10~пс 
	\item Возможность выставления <<мертвого времени>> в широком диапазоне - от 0,1~мкс до 100~мкс
	\item Возможность регулировки квантовой эффективности с шагом 2,5~\% в диапазоне от 5~\% до 25~\%
	\item Полупроводниковая структура ЛФД - InGaAs/InP
	\item Относительно низкий уровень темнового счета при заданных параметрах квантовой эффективности
\end{enumerate}


В ходе исследования для обеспечения реалистичных условий атаки злоумышленника на измерительное оборудование в составе СКК устройство рассматривалось, как <<черный ящик>>, то есть оно не вскрывалось и не производились манипуляции с внутренними платами и микросхемами. Все настройки детектора выставлялись в соответствии со штатным режимом для систем квантовой коммуникации на боковых частотах модулированного излучения. Основные настройки вынесены в таблицу \ref{tab:ID210_setups}.  




\begin{table} 
	\centering
	\caption{Типовые настройки детектора ID210 в составе СКК}
	\label{tab:ID210_setups}
		\begin{tabular}{|c|c|c|}
			\hline
				№  				& Параметр    				 & Значение     \\
			\hline
				1 				& Квантовая эффективность,~\% 	 & 10 		 \\
			\hline 

				2 				& Частота стробирования,~Гц 		 & $10^8$   \\
			\hline

				3 				& Ширина окна,~нс & 3 	     \\
			\hline

				4 				& <<Мертвое>> время,~нс  & 100 		  \\
			\hline

				5 				& Частота темновых срабатываний,~Гц & 200 		  \\

			\hline
		\end{tabular}
\end{table}





%%%%%%%%%%%%%%%%%%%%%%%%%%%%%%%%%%%%%%%%%%%%%%%%%%%%%%%%%%%%%%%%%%%%%%%%%%%%%%%%%%%%%%%%%%%%%%%%%%%%%%%%%%%%%%%%%

\section{Оптическая схема выведения детектора из режима Гейгера} \label{sec:ch2/sec2}

Для успешного осуществления атаки с навязыванием ключа злоумышленнику требуется манипулировать детектором, то есть форсировать срабатывания и их отсутствие в нужные моменты времени, при этом предполагается, что тип применяемого измерительного оборудования известен, но непосредственный доступ к нему отсутствует. В таких рамках модель атаки ограничивается возможностью воздействия на детектор только оптическими методами непосредственно из квантового канала. 


Известно, что в линейном режиме работы ЛФД при подачи на него постоянной оптической мощности увеличивается фототок, следовательно при подаче постоянного значения напряжение обратного смещения $-V_{bias}$ и при наличии в цепи гасящего лавину резистора, величина падения напряжения на резисторе растет, а на ЛФД снижается (рис.\ref{fig:Quenching}. Суть атаки с выведением детектора из режима Гейгера, или <<ослеплением>> ДОФ, сводится к тому, чтобы сместить режим работы относительно напряжения пробоя ЛФД. При таком подходе даже дополнительных импульсов $V_{gate}$ становится недостаточно и диод все время находится в режиме линейной зависимости фототока от величины мощности оптического излучения, падаюшего на него.  

Тем не менее, как указано на рисунке \ref{fig:Vbreakdown}, в линейном режиме остается возможность превысить пороговое значение фототока $I_{det}$ и сформировать импульс срабатывания детектора.

Таким образом, методика выведения детектора из режима счета фотонов в линейный режим для осуществления атаки с навязыванием ключа (\todo{<<Faked-state attack>>}) легко формализуется. Экспериментальное исследование уязвимости детектора одиночных фотонов к такому типа атак реализуется в три этапа, представленных на рисунке \ref{fig:Method_2.3}:

 \begin{enumerate}
	\item Определение величины постоянной оптической мощности, достаточной для выведения детектора из режима Гейгера
	\item Подстройка оптического импульса под окно срабатывания детектора одиночных фотонов
	\item Определение зависимости вероятности срабатывания детектора от величины энергии фотонов в импульсе
\end{enumerate}

 \begin{figure}[ht] 
  \centering
  \includegraphics{Method_2.3.eps}
  \caption{Методика выведения детектора из режима Гейгера}
  \label{fig:Method_2.3}
\end{figure}

%
%  На рисунке \ref{fig:Scheme_2.3} представлена принципиальная оптическая схема для проведения исследования уязвимости детектора. 
%
% \begin{figure}[ht] 
%  \centering
%  \includegraphics{Scheme_2.3.eps}
%  \caption{Принципиальная оптическая схема эксперимента}
%  \label{fig:Scheme_2.3}
% \end{figure}

%%%%%%%%%%%%%%%%%%%%%%%%%%%%%%%%%%%%%%%%%%%%%%%%%%%%%%%%%%%%%%%%%%%%%%%%%%%%%%%%%%%%%%%%%%%%%%%%%%%%%%%%%%%%%%%%%

\section{Корреляция оптической мощности в плечах светоделителя} \label{sec:ch2/sec3}


Для того, чтобы одновременно производить воздействие оптическим излучением на исследуемый детектор одиночных фотонов и фиксировать величину этого воздействия в оптической схеме используется светоделитель 2х2 с коэффициентом деления 50:50. Один выход подключен к детектору Д2, второй к измерителю оптической мощности Д1, как показано на рисунке \ref{fig:Scheme_2.4}. Однако, обычно в волоконно-оптических элементах наблюдается небольшое отклонение основных величин, характеризующих устройство, от паспортных значений. Для того, чтобы скорректировать это отклонение, при помощи двух измерителей оптической мощности фиксируется величина оптической мощности на обоих выходах светоделителя и находится разница между двумя величинами в зависимости от напряжения, подаваемого с генератора ГЕН1 (Agilent 81110A) на лазерный диод ЛД1, применяемый для выведения детектора и режима Гейгера.   

 \begin{figure}[ht]
  \centering
  \includegraphics{Scheme_2.4.eps}
  \caption{Принципиальная оптическая схема эксперимента}
  \label{fig:Scheme_2.4}
\end{figure}


График зависимости представлен на рисунке \ref{fig:Beamsplitter}. Видно, что расхождение оставляет величину не превышающую 1 дБ. Соответственно, далее при проведении всех измерений производится корректировка на полученную величину. Таким образом учитывается неидеальность коэффициента деления используемого волоконно-оптического светоделителя 2х2. 


 \begin{figure}[ht]
  \centering
  \includegraphics{Beamsplitter.png}
  \caption{Зависимость оптической мощности на двух выходах светоделителя 2х2 от величины напряжения на ЛД1}
  \label{fig:Beamsplitter}
\end{figure}
\pagebreak

%%%%%%%%%%%%%%%%%%%%%%%%%%%%%%%%%%%%%%%%%%%%%%%%%%%%%%%%%%%%%%%%%%%%%%%%%%%%%%%%%%%%%%%%%%%%%%%%%%%%%%%%%%%%%%%%%

\section{Определение величины постоянной оптической мощности, достаточной для выведения детектора из режима Гейгера} \label{sec:ch2/sec4}

На первом этапе выведения детектора из режима счета фотонов требуется определить необходимую и достаточную величину оптической мощности для <<ослепления>> детектора. Характерной особенностью и показателем успешного завершения первого этапа является отсутствие темновых срабатываний ДОФ в линейном режиме. Таким образом, задача сводится к тому, чтобы направить в детектор оптическое излучения в области спектральной чувствительности детектора, для простоты имитации действий злоумышленника это будет длина волны 1550~нм, и зафиксировать смену растущего количества отсчетов их полным отсутствием. На рисунке \ref{fig:Scheme_2.5} представлена оптическая схема эксперимента. 


 \begin{figure}[ht]
  \centering
  \includegraphics{Scheme_2.5.eps}
  \caption{Принципиальная оптическая схема эксперимента}
  \label{fig:Scheme_2.5}
\end{figure}

В качестве источника постоянного излучения применяется лазерный модуль с распределенной обратной связью (Alcatel 1905 LMI). Излучаемая длина волны - 1550~нм. Особенностями данного модуля так же являются узкая спектральная полоса порядка 2~МГц и встроенный оптический изолятор, предотвращающий попадание переотраженного излучения обратно в лазерный диод. 
 На лицевой панели индикатор количества отсчетов показывает все нули. 


Для детектора ID210 переход из режима Гейгера в линейный режим при штатных для системы квантовой коммуникации на боковых частотах настройках осуществляется при величине постоянной оптической мощности, превыщающей 24~нВт. Далее в оптических схемах ПОА исключен в связи с тем, что для изменения выходной мощности ЛД1 использовался генератор ГЕН1 с постоянным уровнем напряжения и шагом 0,1~В, как показано на \ref{fig:Beamsplitter}. При подаче постоянного напряжения 0,7~В излучаемая оптическая мощность оставляла 35~нВт, что в рамках данного исследования принималось за мощность, необходимую для <<ослепления>> детектора


%%%%%%%%%%%%%%%%%%%%%%%%%%%%%%%%%%%%%%%%%%%%%%%%%%%%%%%%%%%%%%%%%%%%%%%%%%%%%%%%%%%%%%%%%%%%%%%%%%%%%%%%%%%%%%%%%

\section{Подстройка оптического импульса под окно детектора одиночных фотонов} \label{sec:ch2/sec5}

Вторым этапом определения возможности контроля злоумышленником детектора на основе ЛФД является подстройка оптического импульса под окно ДОФ. Эта подстройка необходима для того, чтобы контролирующий оптический импульс лазерного источника злоумышленника точно приходил в момент ожидаемого прибытия фотона от блока отправителя. В ином случае, эффективность значительно снижается. В качестве источника контролирующих импульсов используется лазерный диод с распределенной обратной связью (DFB) <<Gooch \& Housego>> модели AA1401. Он излучает на длине волны 1550~нм с узкой спектральной полосой в 1~МГц. В качестве формирователя импульсов используется генератор ГЕН2 (Highland Technology P400). Импульсы с частотой 10~МГц и длительностью 0,5~нс подаются на ЛД2. Величина 10~МГц обусловлена минимальным возможным значением мертвого времени (deadtime) ДОФ, равным 100~нс. Это значит, что при подаче синхронизационной частоты в 100~МГц с блока отправителя системы квантовой коммуникации, окно детектирования будет открываться не чаще, чем 10~МГц. При помощи ГЕН2 вносится задержка по фазе импульса, подаваемого на ЛД2, с точностью до 1~пс. Синхронизационная частота 100~МГц имитируется средствами ГЕН2. На порт детектора <<Trigger>> подается указанная частота. При увеличении оптической мощности контролирующего импульса в ДОФ в линейном режиме увеличивается фототок и при некоторой величине превышает порог срабатывания компаратора. Таким образом на дисплее детектора отображаются отсчеты, количество которых возрастает с увеличением оптической мощности. Зафиксировав на некоторой величине мощность контролирующих импульсов требуется подстройка задержки этих импульсов при помощи ГЕН2. При изменении величины задержки наблюдается рабочая точка с глобальным максимумом. На рисунке \ref{fig:Scheme_2.6} продемонстрирована оптическая схема эксперимента.

 \begin{figure}[ht]
  \centering
  \includegraphics{Scheme_2.6.eps}
  \caption{Принципиальная оптическая схема эксперимента}
  \label{fig:Scheme_2.6}
\end{figure}

При этом в одном плече установлен оптоэлектронный преобразователь Д3 (LeCroy OE555), который подключен к осциллографу ОСЦ (LeCroy 820Zi) для наблюдения оптических импульсов. Также к этому осциллографу подключен выход детектора ID210 <<Detection 1>>, где можно наблюдать электрические импульсы срабатывания, и выход <<Gate>>, где можно наблюдать окна срабатывания. Полученная осциллограмма представлена на \ref{fig:LeCroy}. 

 \begin{figure}[ht]
  \centering
  \includegraphics[scale=0.45] {LeCroy97.jpg}
  \caption{Осциллограмма с наблюдаемыми оптическими импульсами, окнами срабатываний детектора фотонов и опорной частотой}
  \label{fig:LeCroy}
\end{figure}

Нормированные на единицу и скорректированные по фазе импульсы окна срабатывания и контролирующий оптический импульс представлены на \ref{fig:optical_pulse_and_gate}. По уровню половины максимального значения видно, что длительность окна срабатывания составляет 3~нс, как и отображает дисплей ID210. Длительность оптического импульса искажена ограничением полосы опто-электронного преобразователя в 2~ГГц. 

 \begin{figure}[ht]
  \centering
  \includegraphics{images/optical pulse and gate.png}
  \caption{Осциллограммы контролирующего оптического импульса и окна срабатывания, скорректированные по фазе}
  \label{fig:optical_pulse_and_gate}
\end{figure}


\pagebreak
%%%%%%%%%%%%%%%%%%%%%%%%%%%%%%%%%%%%%%%%%%%%%%%%%%%%%%%%%%%%%%%%%%%%%%%%%%%%%%%%%%%%%%%%%%%%%%%%%%%%%%%%%%%%%%%%%


\section{Определение количества срабатываний детектора от величины мощности оптического излучения} \label{sec:ch2/sec6}

Для контроля срабатываний приемного узла системы квантовой коммуникации злоумышленнику требуется определить пороговые величины мощности (или энергии) контролирующего оптического импульса, при которых с единичной и нулевой вероятностями произойдет срабатывание ДОФ. После подстройки по фазе оптического импульса под окно срабатывания требуется с небольшим шагом увеличивать величину оптической мощности контролирующего импульса и фиксировать увеличение количества срабатывания от нуля до максимально возможного при данных настройках. Изменение с небольшим шагом производится переменным оптическим аттенюатором. Принципиальная оптическая схема отображена на \ref{fig:Scheme_2.7}   
 \begin{figure}[ht]
  \centering
  \includegraphics{Scheme_2.7.eps}
  \caption{Принципиальная оптическая схема эксперимента}
  \label{fig:Scheme_2.7}
\end{figure}

На \ref{fig:experimental_setup} изображена фотография собранного экспериментального стенда с подписями соответствующих узлов. 

 \begin{figure}[ht]
  \centering
	 %\includegraphics{blinding experimental setup}
  \caption{Фотография экспериментального стенда}
  \label{fig:experimental_setup}
\end{figure}

Таким образом, экспериментально проверено, что для <<ослепления>> детектора фотонов злоумышленнику достаточно величины в 35~нВт постоянной оптической мощности, а также необходимо посылать импульсы с частотой 10~МГц и средним уровнем мощности до 154~нВт для отсутствия срабатываний, а для формирования в приемном блоке срабатываний в каждом такте необходим средний уровень мощности оптических импульсов величиной от 258~нВт.   
 \begin{figure}[ht]
  \centering
  \includegraphics{35_nW.png}
  \caption{Зависимость количества отсчетов от величины оптической мощности контролирующего импульса}
  \label{fig:35_nW}
\end{figure}


%\pagebreak
%%%%%%%%%%%%%%%%%%%%%%%%%%%%%%%%%%%%%%%%%%%%%%%%%%%%%%%%%%%%%%%%%%%%%%%%%%%%%%%%%%%%%%%%%%%%%%%%%%%%%%%%%%%%%%%%%


%\section{Определение зависимости вероятности срабатывания детектора от величины мощности оптического излучения в квантовом канале} \label{sec:ch2/sec8}




%\pagebreak
%%%%%%%%%%%%%%%%%%%%%%%%%%%%%%%%%%%%%%%%%%%%%%%%%%%%%%%%%%%%%%%%%%%%%%%%%%%%%%%%%%%%%%%%%%%%%%%%%%%%%%%%%%%%%%%%%


\section{Определение зависимости вероятности срабатывания детектора от величины энергии фотонов в импульсе} \label{sec:ch2/sec7}
Величина количества срабатываний посредством нормировки на единицу преобразуется в вероятность срабатываний в пределах от 0 до 1. Максимальной вероятности срабатываний соответствует количество отсчетов равное $f_{pulse} = 10$~МГц. По формуле \ref{eq:equation2_8} пересчитывается величина средней оптической мощности в соответствующую ей величину энергии при заданной частоте повторения импульсов.  
\begin{equation}
 \label{eq:equation2_8}
	E = \frac{P_{opt}}{f_{pulse}}
\end{equation}
 
 Исходя из этого, получаем набор зависимостей на \ref{fig:Probability_vs_Energy}. Видно, что с увеличением оптической мощности на ЛД1 кривая зависимости вероятности срабатывания от энергии фотонов в контролирующем импульсе становится более резкой, и при больших значениях оптической мощности выведения детектора из режима счета фотонов требуется повышать мощность контролирующих оптических импульсов для формирования срабатываний с единичной вероятностью. 
 
 
 \begin{figure}[ht]
  \centering
  \includegraphics[scale=0.8]{Probability_vs_Energy.png}
  \caption{Зависимость вероятности срабатывания детектора от энергии контролирующего импульса}
  \label{fig:Probability_vs_Energy}
\end{figure}


%	\pagebreak
%%%%%%%%%%%%%%%%%%%%%%%%%%%%%%%%%%%%%%%%%%%%%%%%%%%%%%%%%%%%%%%%%%%%%%%%%%%%%%%%%%%%%%%%%%%%%%%%%%%%%%%%%%%%%%%%%
\section{Выводы по главе} \label{ch:ch2/sect8}

Во \ref{ch:ch2} главе показано, что использование коммерческих детекторов одиночных фотонов на основе лавинных фотодиодов в режиме Гейгера модели id210 с частотой стробирования 100 МГц  требует применения дополнительных средств защиты от атаки с выведением из режима Гейгера при помощи коротких оптических импульсов с энергией не менее 15,4 фДж и при постоянном уровне оптической засветки средним уровнем мощности излучения не менее 35 нВт.  

