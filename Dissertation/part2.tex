\chapter{Исследование устойчивости системы квантовой коммуникации на боковых частотах к атакам на измерительное оборудование}  \label{ch:ch2}


\section{Детектор в составе системы} \label{sec:ch2/sec1}

В зависимости от поставленных целей и решаемых задач в состав систем квантовой коммуникации включают в основном два типа детекторов одиночных фотонов, сравнительная характеристика которых дана в главе \ref{sec:ch1/sec5}. Для магистральных дистанций от 100~км, либо для обеспечения высокой скорости формирования квантовых кодирующих последовательностей применяют сверхпроводниковые ДОФ (SNSPD). Для внутригородских дистанций до 100~км с потерями в линиях связи менее 15~дБ обычно применяют более компактный, простой в использовании ДОФ на базе лавинного фотодиода (SPAD), однако его типовые характеристики на порядок уступают характеристикам сверхпроводникового ДОФ. 


В СКК на боковых частотах применяется коммерчески доступный детектор модели ID210, разработанный компанией idQuantique. Его отличительными особенностями являются:
\begin{enumerate}
	\item Поддержка высокой частоты стробирующих импульсов - до 100~МГц
	\item Возможность подачи стробирующих импульсов от внешнего устройства (External gating mode)
	\item Широкий диапазон настройки ширины окна срабатывания (gate) - от 0,5~нс до 25~нс
	\item Выставление задержки открытия окна срабатывания относительно стробирующего импульса (Trigger delay) в диапазоне до 10~нс с высоким разрешением во времени - 10~пс 
	\item Возможность выставления <<мертвого времени>> в широком диапазоне - от 0,1~мкс до 100~мкс
	\item Возможность регулировки квантовой эффективности с шагом 2,5~\% в диапазоне от 5~\% до 25~\%
	\item Полупроводниковая структура ЛФД - InGaAs/InP
	\item Относительно низкий уровень темнового счета при заданных параметрах квантовой эффективности
\end{enumerate}


В ходе исследования для обеспечения реалистичных условий атаки злоумышленника на измерительное оборудование в составе СКК устройство рассматривалось, как <<черный ящик>>, то есть оно не вскрывалось и не производились манипуляции с внутренними платами и микросхемами. Все настройки детектора выставлялись в соответствии со штатным режимом для систем квантовой коммуникации на боковых частотах модулированного излучения. Основные настройки вынесены в таблицу \ref{tab:ID210_setups}.  




\begin{table} 
	\centering
	\caption{Типовые настройки детектора ID210 в составе СКК}
	\label{tab:ID210_setups}
		\begin{tabular}{|c|c|c|}
			\hline
				№  				& Параметр    				 & Значение     \\
			\hline
				1 				& Квантовая эффективность,~\% 	 & 10 		 \\
			\hline 

				2 				& Частота стробирования,~Гц 		 & $10^8$   \\
			\hline

				3 				& Ширина окна,~нс & 5 	     \\
			\hline

				4 				& <<Мертвое>> время,~нс  & 100 		  \\
			\hline

				5 				& Частота темновых срабатываний,~Гц & 200 		  \\

			\hline
		\end{tabular}
\end{table}




%%%%%%%%%%%%%%%%%%%%%%%%%%%%%%%%%%%%%%%%%%%%%%%%%%%%%%%%%%%%%%%%%%%%%%%%%%%%%%%%%%%%%%%%%%%%%%%%%%%%%%%%%%%%%%%%%

\section{Выведение детектора из режима Гейгера} \label{sec:ch2/sec2}

Первым этапом проведения атаки с навязыванием ключа является определение возможности выведения детектора из режима счета фотонов (режима Гейгера). Известно, что вольт-амперная характеристика лавинного фотодиода имеет вид, представленный на рисунке \ref{fig:APDs_VA}.    

 \begin{figure}[ht]
  \centering
  \includegraphics{APDs_VA.png}
  \caption{Вольт-амперная характеристика ЛФД}
  \label{fig:APDs_VA}
\end{figure}


Лавинный фотодиод работает при обратном напряжении смещения близком к напряжению пробоя. Типичная величина находится в диапазоне от -40~В до -60~В. При величине обратного напряжения смещения ниже напряжения пробоя ЛФД находится в линейном режиме, где величина фототок линейно зависит от падающей оптической мощности. При увеличении величины обратного смещения выше напряжения пробоя ЛФД переходит в режим счета фотонов, или режим Гейгера. В этом режиме даже энергии единиц фотонов достаточно для формирования лавины зарядов и резкого скачка фототока. При превышении некоторого порогового значения Iдет, регулируемого уровнем срабатывания компаратора, формируется электрический импульс, который и интерпретируется, как регистрация одиночного фотона. 

 \begin{figure}[ht]
  \centering
  \includegraphics{Vbreakdown}
  \caption{Граница между режимами работы ЛФД при подаче обратного напряжения смещения}
  \label{fig:Vbreakdown}
\end{figure}

Гашение лавины в самом простом случае происходит благодаря последовательно подключенному в цепь резистору, как показано на рисунке \ref{fig:Quenching}. Существуют также схемы активного гашения лавины.  


 \begin{figure}[ht]
  \centering
  \includegraphics{Quenching}
  \caption{Принципиальная электрическая схема подключения ЛФД для регистрации одиночных фотонов}
  \label{fig:Quenching}
\end{figure}


%%%%%%%%%%%%%%%%%%%%%%%%%%%%%%%%%%%%%%%%%%%%%%%%%%%%%%%%%%%%%%%%%%%%%%%%%%%%%%%%%%%%%%%%%%%%%%%%%%%%%%%%%%%%%%%%%
\section{Оптическая схема выведения детектора из режима Гейгера} \label{sec:ch2/sec3}


 \begin{figure}[ht] 
  \centering
  \includegraphics{Scheme_2.3.eps}
  \caption{Принципиальная оптическая схема эксперимента}
  \label{fig:Scheme_2.3}
\end{figure}




%%%%%%%%%%%%%%%%%%%%%%%%%%%%%%%%%%%%%%%%%%%%%%%%%%%%%%%%%%%%%%%%%%%%%%%%%%%%%%%%%%%%%%%%%%%%%%%%%%%%%%%%%%%%%%%%%
\section{Корреляция оптической мощности в плечах светоделителя} \label{sec:ch2/sec4}


 \begin{figure}[ht]
  \centering
  \includegraphics{Scheme_2.4.eps}
  \caption{Принципиальная оптическая схема эксперимента}
  \label{fig:Scheme_2.4}
\end{figure}



%%%%%%%%%%%%%%%%%%%%%%%%%%%%%%%%%%%%%%%%%%%%%%%%%%%%%%%%%%%%%%%%%%%%%%%%%%%%%%%%%%%%%%%%%%%%%%%%%%%%%%%%%%%%%%%%%
\section{Определение оптической мощности, достаточной для выведения детектора из режима Гейгера} \label{sec:ch2/sec5}


 \begin{figure}[ht]
  \centering
  \includegraphics{Scheme_2.5.eps}
  \caption{Принципиальная оптическая схема эксперимента}
  \label{fig:Scheme_2.5}
\end{figure}


%%%%%%%%%%%%%%%%%%%%%%%%%%%%%%%%%%%%%%%%%%%%%%%%%%%%%%%%%%%%%%%%%%%%%%%%%%%%%%%%%%%%%%%%%%%%%%%%%%%%%%%%%%%%%%%%%
\section{Подстройка оптического импульса под окно детектора одиночных фотонов} \label{sec:ch2/sec6}


 \begin{figure}[ht]
  \centering
  \includegraphics{Scheme_2.6.eps}
  \caption{Принципиальная оптическая схема эксперимента}
  \label{fig:Scheme_2.6}
\end{figure}

%%%%%%%%%%%%%%%%%%%%%%%%%%%%%%%%%%%%%%%%%%%%%%%%%%%%%%%%%%%%%%%%%%%%%%%%%%%%%%%%%%%%%%%%%%%%%%%%%%%%%%%%%%%%%%%%%
\section{Определение количества срабатываний детектора от величины мощности оптического излучения} \label{sec:ch2/sec7}


 \begin{figure}[ht]
  \centering
  \includegraphics{Scheme_2.7.eps}
  \caption{Принципиальная оптическая схема эксперимента}
  \label{fig:Scheme_2.7}
\end{figure}



%%%%%%%%%%%%%%%%%%%%%%%%%%%%%%%%%%%%%%%%%%%%%%%%%%%%%%%%%%%%%%%%%%%%%%%%%%%%%%%%%%%%%%%%%%%%%%%%%%%%%%%%%%%%%%%%%
\section{Определение зависимости вероятности срабатывания детектора от величины мощности оптического излучения в квантовом канале} \label{sec:ch2/sec8}




%%%%%%%%%%%%%%%%%%%%%%%%%%%%%%%%%%%%%%%%%%%%%%%%%%%%%%%%%%%%%%%%%%%%%%%%%%%%%%%%%%%%%%%%%%%%%%%%%%%%%%%%%%%%%%%%%
\section{Определение зависимости вероятности срабатывания детектора от величины энергии фотонов в импульсе} \label{sec:ch2/sec9}


%%%%%%%%%%%%%%%%%%%%%%%%%%%%%%%%%%%%%%%%%%%%%%%%%%%%%%%%%%%%%%%%%%%%%%%%%%%%%%%%%%%%%%%%%%%%%%%%%%%%%%%%%%%%%%%%%
\section{Атака с навязыванием ключа «поддельными» состояниями} \label{sec:ch2/sec10}


%%%%%%%%%%%%%%%%%%%%%%%%%%%%%%%%%%%%%%%%%%%%%%%%%%%%%%%%%%%%%%%%%%%%%%%%%%%%%%%%%%%%%%%%%%%%%%%%%%%%%%%%%%%%%%%%%
\section{Границы применимости атаки с навязыванием ключа} \label{sec:ch2/sec11}




%%%%%%%%%%%%%%%%%%%%%%%%%%%%%%%%%%%%%%%%%%%%%%%%%%%%%%%%%%%%%%%%%%%%%%%%%%%%%%%%%%%%%%%%%%%%%%%%%%%%%%%%%%%%%%%%%
\section{Оценка возможностей злоумышленника при атаке с выведением детектора из режима Гейгера для систем квантовой коммуникации на боковых частотах} \label{sec:ch2/sec12}



%%%%%%%%%%%%%%%%%%%%%%%%%%%%%%%%%%%%%%%%%%%%%%%%%%%%%%%%%%%%%%%%%%%%%%%%%%%%%%%%%%%%%%%%%%%%%%%%%%%%%%%%%%%%%%%%%
\section{Выводы по главе} \label{ch:ch2/sect13}
