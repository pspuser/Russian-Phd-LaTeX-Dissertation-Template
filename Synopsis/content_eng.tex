
%\section*{Thesis overview}

	\newcommand{\relevance}{\underline{\textbf{\relevanceTXT}}}
%	\newcommand{\progress}{\underline{\textbf{\progressTXT}}}
	\newcommand{\goal}{\underline{\textbf{\goalTXT}}}
	\newcommand{\scientifictasks}{\underline{\textbf{\scientifictasksTXT}}}
	\newcommand{\scientificnovelty}{\underline{\textbf{\scientificnoveltyTXT}}}
	\newcommand{\importance}{\underline{\textbf{\importanceTXT}}}
%	\newcommand{\methods}{\underline{\textbf{\methodsTXT}}}
	\newcommand{\statements}{\underline{\textbf{\statementsTXT}}}
	\newcommand{\validity}{\underline{\textbf{\validityTXT}}}
	\newcommand{\approbation}{\underline{\textbf{\approbationTXT}}}
	\newcommand{\implementation}{\underline{\textbf{\implementationTXT}}}
%	\newcommand{\contribution}{\underline{\textbf{\contributionTXT}}}
	\newcommand{\refs}{\underline{\textbf{\refsTXT}}}
\chapter*{Synopsis}
\addcontentsline{toc}{chapter}{Synopsis} 
\section*{Thesis overview}

%	\section*{The goal}
%	\section*{Scientific tasks:}
%	\section*{Scientific novelty}
%	\section*{Scientific statements:}
%	\section*{Practical importance}
%	\section*{Reliability and the validity}
%	\section*{Implementation of the obtained results}
%	\section*{Approbation}
%	\section*{Publication}
%	\section*{Author contribution}
%	\section*{The structure}
%	\section*{MAIN CONTENTS OF WORK}


  {\relevance} 
  Quantum informatics is a field of knowledge that combines elements of the photonics and the theory of information. Quantum bits, or qubits, are used as the basic units of quantum information science. Qubit is a system that can be in a superposition state and used to store, calculate and transmit information. One of the promising areas are a quantum calculations and a quantum memory -- the main components of a quantum computer -- a totally new type of computing device that works on the fundamental principles of quantum mechanics. The transmission of a quantum state at various distances, or quantum teleportation, is another promising application of quantum informatics. The generation of a symmetric bit sequence by the quantum methods, or quantum key distribution (QKD), was formed as a scientific direction in the 80's of the 20th century. This direction is advanced in terms of application. In quantum communication systems (QCS), through the use of quantum states, two or more legitimate users can distribute symmetric bit sequences, which can subsequently be used as keys for encoding information, so that any eavesdropping attempts of the communication channel will be detected by the increased error rate.
  
Differences between physical implementations of QCS from ideal models used in theoretical approaches can be the basis for conducting various types of attacks on equipment used in this systems. It was previously shown that gated single-photon detectors (SPD) of several commercially available QKD systems were vulnerable to attack by the eavesdropper. It is noteworthy that exposure can be carried out even by optical methods. Eavesdropping can be made directly from a quantum channel that connects legitimate users. In practice, optical fiber is most often used as a medium for the transferring of quantum states. Single photons and their properties are usually used for that purposes.
	
This type of attack is called a “faked-states attack”. It is based on the fact that for the detection of photons, a detector based on avalanche photodiodes (APDs) operating in the Geiger mode, or photon counting mode, is used. With this approach, in the absence of additional protective measures, the attacker has the ability to use the powerful constant optical radiation to carry the detector out from the Geiger mode to the linear mode and provoke triggering due to the use of powerful short pulses. Thus, the feasibility of complete control of the single-photon registration node, became critical for QKD protocols, and as a result, leads the eavesdropper to imperceptible obtaining of the full key correlating with that of legitimate users.
	

There are several ways to counter this type of attack, but in practice most of them have not been tested in laboratories specializing in quantum hacking.  Several countermeasures that were analyzed and investigated have included counterattacks and hacking methods that were not taken into account by the developers of countermeasures.
	
The most effective countermeasure against the well-known types of attacks on detector side is the implementation of a  quite new QKD architecture that is resistant to attacks on measuring equipment (Measurement-Device-Independent, or MDI). With this approach, it is initially assumed that the single-photon registration unit is moved outside the sender and receiver blocks and is completely accessible to the eavesdropper. However, such an approach is characterized by a higher level of complexity of implementing the optical scheme and lower characteristics with respect to QKD systems with a point-to-point topology with a registration node in one of the blocks.
	

One of the promising approaches is the method of quantum communication at subcarrier-wave quantum key distribution (SCW-QKD). Its distinctive feature is the carrying of the quantum channel out to the sidebands generated as a result of modulation. This ensures high resistance to external factors and spectral efficiency, as well as good performance in the ratio of the key formation speed to the distance between the sender and receiver units. However, as part of this type of system for small and medium distances (up to 100 km), single-photon detectors based on the APD are also used, and the resistance to attacks on the measuring equipment of SCW-QKD systems has not been investigated.
	
	
% Обзор, введение в тему, обозначение места данной работы в
% мировых исследованиях и~т.\:п., можно использовать ссылки на~другие
% работы\ifnumequal{\value{bibliosel}}{1}{~\autocite{Gosele1999161}}{}
% (если их~нет, то~в~автореферате
% автоматически пропадёт раздел <<Список литературы>>). Внимание! Ссылки
% на~другие работы в разделе общей характеристики работы можно
% использовать только при использовании \verb!biblatex! (из-за технических
% ограничений \verb!bibtex8!. Это связано с тем, что одна
% и~та~же~характеристика используются и~в~тексте диссертации, и в
% автореферате. В~последнем, согласно ГОСТ, должен присутствовать список
% работ автора по~теме диссертации, а~\verb!bibtex8! не~умеет выводить в одном
% файле два списка литературы).
% При использовании \verb!biblatex! возможно использование исключительно
% в~автореферате подстрочных ссылок
% для других работ командой \verb!\autocite!, а~также цитирование
% собственных работ командой \verb!\cite!. Для этого в~файле
% \verb!Synopsis/setup.tex! необходимо присвоить положительное значение
% счётчику \verb!\setcounter{usefootcite}{1}!.
% 
% Для генерации содержимого титульного листа автореферата, диссертации
% и~презентации используются данные из файла \verb!common/data.tex!. Если,
% например, вы меняете название диссертации, то оно автоматически
% появится в~итоговых файлах после очередного запуска \LaTeX. Согласно
% ГОСТ 7.0.11-2011 <<5.1.1 Титульный лист является первой страницей
% диссертации, служит источником информации, необходимой для обработки и
% поиска документа>>. Наличие логотипа организации на~титульном листе
% упрощает обработку и~поиск, для этого разметите логотип вашей
% организации в папке images в~формате PDF (лучше найти его в векторном
% варианте, чтобы он хорошо смотрелся при печати) под именем
% \verb!logo.pdf!. Настроить размер изображения с логотипом можно
% в~соответствующих местах файлов \verb!title.tex!  отдельно для
% диссертации и автореферата. Если вам логотип не~нужен, то просто
% удалите файл с~логотипом.

% \ifsynopsis
% Этот абзац появляется только в~автореферате.
% Для формирования блоков, которые будут обрабатываться только в~автореферате,
% заведена проверка условия \verb!\!\verb!ifsynopsis!.
% Значение условия задаётся в~основном файле документа (\verb!synopsis.tex! для
% автореферата).
% \else
% Этот абзац появляется только в~диссертации.    ----- условие выполняется как else в обоих случаях (для диссертации и реферата)
% Через проверку условия \verb!\!\verb!ifsynopsis!, задаваемого в~основном файле
% документа (\verb!dissertation.tex! для диссертации), можно сделать новую
% команду, обеспечивающую появление цитаты в~диссертации, но~не~в~автореферате.
% \fi

% {\progress}
% Этот раздел должен быть отдельным структурным элементом по
% ГОСТ, но он, как правило, включается в описание актуальности
% темы. Нужен он отдельным структурынм элемементом или нет ---
% смотрите другие диссертации вашего совета, скорее всего не нужен.

{\aim} данной работы является исследование возможностей злоумышленника по получению секретного ключа с использованием атак на измерительное оборудование систем квантовой коммуникации на боковых частотах и разработка методов противодействия атакам.


Для~достижения поставленной цели необходимо было решить следующие {\tasks}:
\begin{enumerate}
  \item Исследование устойчивости детектора одиночных фотонов, применяемого в системах квантовой коммуникации на боковых частотах, к атакам с выведением из режима Гейгера (<<ослеплением>>). 

  \item Оценка возможностей злоумышленника при атаке с выведением из режима Гейгера для систем квантовой коммуникации на боковых частотах. 

  \item Разработка оптической схемы системы квантовой коммуникации, устойчивой к атакам на измерительное оборудование. 

  \item Разработка протокола квантовой рассылки ключа, устойчивого к атаке на измерительное оборудование. 

\end{enumerate}


{\novelty}
\begin{enumerate}
  \item Впервые исследована устойчивость системы квантовой коммуникации на боковых частотах к атакам злоумышленника на измерительное оборудование приёмного блока. 
  \item Впервые предложена и применена контрмера против атаки злоумышленника с выведением детектора из режима Гейгера и навязыванием легитимным пользователям ключа. 
  \item Было выполнено оригинальное исследование и разработана оптическая схема системы квантовой коммуникации на боковых частотах с вынесением устройства детектирования одиночных фотонов из приемного блока в <<недоверенный>> узел, подконтрольный злоумышленнику 
\end{enumerate}

{\influence} 

Разработанные методы и подход позволили однозначно определить уязвимость коммерчески доступных детекторов компании id Quantique модели id210 к выведению из режима Гейгера оптическими средствами. В связи с чем доказана необходимость применения дополнительных мер защиты от атак на измерительный узел. Предложена схема, позволяющая производить активный мониторинг попыток <<ослепить>> детектор, благодаря использованию особенностей систем квантовой коммуникации на боковых частотах. Результаты внедрены в производство ООО "Кванттелеком". 

%  {\methods} \todo{TO DO} \ldots

%%%%%	{\defpositions}
%%%%%	\begin{enumerate}
%%%%%  \item Использование коммерческих детекторов одиночных фотонов на основе лавинных фотодиодов в режиме Гейгера модели id210 с частотой стробирования 100 МГц  требует применения дополнительных средств защиты от атаки с выведением из режима Гейгера при помощи коротких оптических импульсов с энергией не менее 15,4 фДж и при постоянном уровне оптической засветки средним уровнем мощности излучения не менее 35 нВт.  
%%%%%  \item Измерение величины оптического излучения на несущей частоте, отраженного от оптического фильтра, при помощи мониторного фотодиода в приемном блоке системы квантовой коммуникации на боковых частотах в диапазоне от 7 нВт до 2,93 мкВт с применением дополнительных мер в виде пассивного оптического аттенюатора номиналом 10 дБ для его защиты позволяет противостоять атаке с выведением детектора одиночных фотонов из режима Гейгера и навязыванием ключа нелегитимным пользователем. 
%%%%%  \item Метод квантовой коммуникации на боковых частотах позволяет реализовывать протокол, устойчивый к контролю нелегитимным пользователем измерительного оборудования. 
%%%%%  \item В результате интерференции квантового фазомодулированного сигнала на боковых частотах на симметричном светоделителе в схеме квантовой рассылки ключа с узлом регистрации, независящим от легитимного пользователя, происходит спектральное разделение квантового сигнала и сигнала на центральной длине волны с их независимой регистрацией в разных плечах светоделителя. 
%%%%%%%  \item Показана возможность двукратного увеличения дальности квантовой рассылки ключа на боковых частотах посредством применения недоверенной системы регистрации квантовых состояний. 
%%%%%%%%\end{enumerate}
% В папке Documents можно ознакомиться в решением совета из Томского ГУ
% в~файле \verb+Def_positions.pdf+, где обоснованно даются рекомендации
% по~формулировкам защищаемых положений.


{\defpositions}
\begin{enumerate}

  \item Детектор одиночных фотонов на основе лавинного фотодиода с частотой стробирования 100 МГц  может быть выведен из режима Гейгера при постоянном уровне оптической засветки средним уровнем мощности излучения не менее 35 нВт и регистрировать срабатывания посредством воздействия на него оптических импульсов с энергией более 15,4 фДж независимо от частоты следования этих импульсов
  \item Преобразование частоты контролирующего сигнала посредством фазовой модуляции на радиочастотной моде позволяет обнаружить атаку с поддельными состояниями и навязыванием срабатываний модулю получателя системы квантовой коммуникации.
  \item Применение слабых когерентных многомодовых состояний с фазовым кодированием позволяет реализовывать протокол, устойчивый к контролю нелегитимным пользователем измерительного оборудования, в том числе с вынесением за пределы приемного модуля. 
  \item В результате интерференции квантовых сигналов на боковых частотах, сформированных в результате  фазовой модуляции двух независимых абонентов системы квантовой коммуникации на симметричном светоделителе в составе узла регистрации, независимого от легитимных пользователей, происходит спектральное разделение квантового сигнала и сигнала на центральной длине волны с их независимой регистрацией в разных плечах светоделителя. 
  %  \item Показана возможность двукратного увеличения дальности квантовой рассылки ключа на боковых частотах посредством применения недоверенной системы регистрации квантовых состояний. 
 
\end{enumerate}

{\reliability} полученных результатов обеспечивается применением утверждённых методик проведений экспериментальных исследований и аттестованного оборудование. Математическое моделирование и обработка данных, полученных в результате экспериментов, осуществлялось с использованием пакетов прикладных программ MathCad и Excel. Результаты находятся в соответствии с результатами, полученными другими авторами.


{\probation}
Основные результаты работы докладывались~на:
%перечисление основных конференций, симпозиумов и~т.\:п. 
\begin{enumerate}
	\item ICQOQI 2019, Минск, Беларусь, 13 - 17 мая 2019
	\item XLVIII научная и учебно-методическая конференция Университета~ИТМО, Санкт-Петербург, Россия, 29 января - 1 февраля 2019
	\item QCrypt 2018, Шанхай, Китай, 27 - 31 августа 2018
	\item 18th International Conference on Laser Optics ICLO 2018, Санкт-Петербург, Россия, 4 - 8 июня 2018
	\item VII Всероссийский конгресс молодых ученых, Санкт-Петербург, Россия, 17 - 20 апреля 2018
	\item XLVII научная и учебно-методическая конференция Университета ИТМО, Санкт-Петербург, Россия, 30 января - 2 февраля 2018
\end{enumerate}

% {\contribution} Автор принимал активное участие \todo{TO DO} \ldots

%\publications\ Основные результаты по теме диссертации изложены в ХХ печатных изданиях~\cite{Sokolov,Gaidaenko,Lermontov,Management},
%Х из которых изданы в журналах, рекомендованных ВАК~\cite{Sokolov,Gaidaenko},
%ХХ --- в тезисах докладов~\cite{Lermontov,Management}.

\ifnumequal{\value{bibliosel}}{0}{% Встроенная реализация с загрузкой файла через движок bibtex8
    \publications\ Основные результаты по теме диссертации изложены в XX печатных изданиях,
    X из которых изданы в журналах, рекомендованных ВАК,
    X "--- в тезисах докладов.%
}{% Реализация пакетом biblatex через движок biber
%Сделана отдельная секция, чтобы не отображались в списке цитированных материалов
    \begin{refsection}[vak,wos,scopus,papers,conf]% Подсчет и нумерация авторских работ. Засчитываются только те, которые были прописаны внутри \nocite{}.
        %Чтобы сменить порядок разделов в сгрупированном списке литературы необходимо перетасовать следующие три строчки, а также команды в разделе \newcommand*{\insertbiblioauthorgrouped} в файле biblio/biblatex.tex
        \printbibliography[heading=countauthorvak, env=countauthorvak, keyword=biblioauthorvak, section=1]%
        \printbibliography[heading=countauthorwos,env=countauthorwos, keyword=biblioauthorwos, section=1]%
        \printbibliography[heading=countauthorscopus,env=countauthorscopus, keyword=biblioauthorscopus, section=1]%
	\printbibliography[heading=countauthorconf, env=countauthorconf, keyword=biblioauthorconf, section=1]%
        \printbibliography[heading=countauthorothers, env=countauthorothers, keyword=biblioauthorothers, section=1]%
        \printbibliography[heading=countauthor, env=countauthor, keyword=biblioauthor, section=1]%
        \nocite{%Порядок перечисления в этом блоке определяет порядок вывода в списке публикаций автора
                JOT,Chistyakov_2016,%
		Chistiakov:19,%
		scbib1,%
                confbib1,confbib2,%
                bib1,bib2,%
        }%
	\publications\ Основные результаты по теме диссертации изложены в 13 печатных изданиях. 10 из которых изданы в журналах, рекомендованных ВАК, 3 в тезисах докладов. 
	
	\section*{Работы автора по теме диссертации}
{Статьи в журналах, рекомендованных ВАК: }
\begin{enumerate}\addtolength{\itemsep}{-0.5\baselineskip}
\renewcommand{\labelenumi}{[\theenumi]}
\item Vladimir Chistiakov, Anqi Huang, Vladimir Egorov, and Vadim Makarov, Controlling single-photon detector ID210 with bright light, Opt. Express 27, 32253-32262 (2019)
\\
\item Чистяков В.В., Гайдаш А.А., Козубов А.В., Глейм А.В. Исследование интерференции слабых когерентных многомодовых состояний для задач квантовой коммуникации с недоверенным приемным узлом // Научно-технический вестник информационных технологий, механики и оптики. 2019. Т. 19. № 6. doi: 10.17586/2226-1494-2019-19-6
\\
\item    Gleim A.V., Egorov V.I., Nazarov Y.V., Smirnov S.V., Chistyakov V.V., Bannik O.I., Anisimov A.A., Kynev S.M., Ivanova A.E., Collins R.J., Kozlov S.A., Buller G. Secure polarization-independent subcarrier quantum key distribution in optical fiber channel using BB84 protocol with a strong reference//Optics express, IET - 2016, Vol. 24, No. 3, pp. 2619-2633
\\
\item  Глейм А.В., Егоров В.И., Чистяков В.В., Смирнов С.В., Банник О.И., Булдаков Н.В., Гайдаш А.А., Козубов А.В., Васильев А.Б., Кынев С.М., Хоружников С.Э., Козлов С.А., Васильев В.Н. Квантовая коммуникация на боковых частотах со скоростью 1 Мбит/с в городской сети // Оптический журнал -2017. - Т. 84. - № 6. - С. 3-9
\\
\item  Chistyakov V.V., Kynev S.M, Smirnov S.V., Nazarov Y.V., Gleim A.V. Achieving high visibility in subcarrier wave quantum key distribution system // Journal of Physics: Conference Series, IET - 2016, Vol. 735, No. 1, pp. 012085
\\
\item V. V. Chistyakov, A. V. Gleim, V. I. Egorov, Yu. V. Nazarov. Implementation of multiplexing in a subcarrier-wave quantum cryptography system // Journal of Physics: Conference Series - 2014  vol. 541,  pp. 012078
\\
\item   Kynev S.M., Chistyakov V.V., Smirnov S.V., Volkova K.P., Egorov V.I., Gleim A.V. Free-space subcarrier wave quantum communication // Journal of Physics: Conference Series - 2017, Vol. 917, No. 5, pp. 052003
\\

\item    Gleim A.V., Nazarov Y.V., Egorov V.I., Smirnov S.V., Bannik O.I., Chistyakov V.V., Kynev S.M., Anisimov A.A., Kozlov S.A., Vasil'ev V.N. Subcarrier Wave Quantum Key Distribution in Telecommunication Network with Bitrate 800 kbit/s//EPJ Web of Conferences, IET - 2015, Vol. 103, pp. 10005
\\
\item    Gleim A.V., Egorov V.I., Nazarov Y.V., Smirnov S.V., Chistyakov V.V., Bannik O.I., Anisimov A.A., Kynev S.M., Collins R.J., Kozlov S.A., Buller G.S. Polarization insensitive 100 MHz clock subcarrier quantum key distribution over a 45 dB loss optical fiber channel // Conference on Lasers and Electro-Optics, CLEO 2015, IET - 2015, pp. 7182997
\\
\item Gaidash A.A., Kozubov A.V., Chistyakov V.V., Miroshnichenko G.P., Egorov V.I., Gleim A.V. Security conditions for sub-carrier wave quantum key distribution protocol in errorless channel // Journal of Physics: Conference Series - 2017, Vol. 917, No. 6, pp. 062014
\\
%\item  Gleim A.V., Chistyakov V.V., Bannik O.I., Egorov V.I., Buldakov N.V., Vasilev A.B., Gaidash A.A., Kozubov A.V., Smirnov S.V., Kynev S.M., Khoruzhnikov S.E., Kozlov S.A., Vasil'ev V.N. Sideband quantum communication at 1 Mbit/s on a metropolitan area network // Journal of Optical Technology - 2017, Vol. 84, No. 6, pp. 362-36
\\

\end{enumerate}
\noindent{ Другие публикации: }
\begin{enumerate}\addtolength{\itemsep}{-0.5\baselineskip}
\renewcommand{\labelenumi}{[\theenumi]}
\setcounter{enumi}{9}
\item   Чистяков В.В., Кынев С.М., Смирнов С.В., Назаров Ю.В., Глейм А.В. Обеспечение высокой видности в системе квантовой криптографии на боковых частотах // Сборник трудов IX международной конференции молодых ученых и специалистов «Оптика – 2015», с. 658-660
\\
\item Глейм А.В., Назаров Ю.В., Егоров В.И., Чистяков В.В, Смирнов С.В., Банник О.И., Кынев С.М., Иванова А.Е., Дубровская В.Д., Тарасов М.Г., Булдаков Н.В., Кузьмина Т.Б., Чивилихин С.А., Анисимов А.А., Рощупкин С.В., Рогачёв К.С., Хоружников С.Э., Козлов С.А., Васильев В.Н. Создание квантовой сети университета ИТМО //Сборник трудов VIII международной конференции «Фундаментальные проблемы оптики – 2014». Санкт-Петербург, 20-24 октября ,2014, С.3-4,  541 с. 
\\
\item А.В. Глейм, В.И.Егоров, А.А. Анисимов, Ю.В. Назаров, С.М. Кынев, А.В. Рупасов, В.В. Чистяков, А.А.Гайдаш, М.А. Смирнов, С.А. Чивилихин, С.А. Козлов Квантовая рассылка криптографического ключа по оптическому волокну телекоммуникационного стандарта на расстояние 200 км со скоростью 0.18 кбит/с // Cборник трудов III Всероссийская конференция по фотонике и информационной оптике Москва, НИЯУ МИФИ, 2014 с. 17-19
\\


\end{enumerate}

	
%	\setcounter{citeauthorscwostot}{\value{citeauthorscopus}} % вместе setcounter и addtocounter добавляют пробел между словами. По-этому они так раскиданы.
%        в~\arabic{citeauthor}~печатных изданиях,
%	\addtocounter{citeauthorscwostot}{\value{citeauthorwos}}
%	\arabic{citeauthorvak} из которых изданы в журналах, рекомендованных ВАК\sloppy
%	\ifnum \value{citeauthorscwostot}>0
%	, \arabic{citeauthorscwostot} "--- в~периодических научных журналах, индексируемых Web of Science и Scopus\sloppy
%	\fi
%	\ifnum \value{citeauthorconf}>0
%	, \arabic{citeauthorconf} "--- в~тезисах докладов.
%	\else
%	.
%	\fi
%    \end{refsection}
%    \begin{refsection}[vak,wos,scopus,papers,conf]%Блок, позволяющий отобрать из всех работ автора наиболее значимые, и только их вывести в автореферате, но считать в блоке выше общее число работ
%        \printbibliography[heading=countauthorvak, env=countauthorvak, keyword=biblioauthorvak, section=2]%
%        \printbibliography[heading=countauthorwos, env=countauthorwos, keyword=biblioauthorwos, section=2]%
 %       \printbibliography[heading=countauthorscopus, env=countauthorscopus, keyword=biblioauthorscopus, section=2]%
  %      \printbibliography[heading=countauthorothers, env=countauthorothers, keyword=biblioauthorothers, section=2]%
  %      \printbibliography[heading=countauthorconf, env=countauthorconf, keyword=biblioauthorconf, section=2]%
   %     \printbibliography[heading=countauthor, env=countauthor, keyword=biblioauthor, section=2]%
        
    %    \nocite{Chistyakov_2016}%vak
    %    \nocite{Gleim:15}
    %    \nocite{Gleim:16}
    %    \nocite{Gleĭm:17}
        
     %   \nocite{bib1}%other
     %   \nocite{confbib1}%conf
   \end{refsection}
}
% При использовании пакета \verb!biblatex! для автоматического подсчёта
% количества публикаций автора по теме диссертации, необходимо
% их~здесь перечислить с использованием команды \verb!\nocite!.
 % Характеристика работы по структуре во введении и в автореферате не отличается (ГОСТ Р 7.0.11, пункты 5.3.1 и 9.2.1), потому её загружаем из одного и того же внешнего файла, предварительно задав форму выделения некоторым параметрам

%Диссертационная работа была выполнена при поддержке грантов ...

%\underline{\textbf{Объем и структура работы.}} Диссертация состоит из~введения,
%четырех глав, заключения и~приложения. Полный объем диссертации
%\textbf{ХХХ}~страниц текста с~\textbf{ХХ}~рисунками и~5~таблицами. Список
%литературы содержит \textbf{ХХX}~наименование.

 \section*{Main contents of work}
 The \underline{\textbf{introduction}} substantiate the relevance of research carried out as part of this dissertation is substantiated, the goal is formulated, the tasks are set, the scientific novelty and practical significance of the work presented are stated.

 In \underline{\textbf{the first chapter}} with a review character, the main currently known methods for detecting single photons for use in quantum communication systems are considered. The analysis and comparison of various types of photon detectors are carried out, the choice of the single photon detector (SPD) studied in the work based on avalanche breakdown is justified. Possible attacks by a potential attacker on measuring equipment are shown, as well as known countermeasures against this type of attack.

%  картинку можно добавить так:
% \begin{figure}[ht]
%   \centering
%   \includegraphics [scale=0.27] {latex}
%   \caption{Подпись к картинке.}
 %  \label{fig:latex}
% \end{figure}

% Формулы в строку без номера добавляются так:
% \[
%   \lambda_{T_s} = K_x\frac{d{x}}{d{T_s}}, \qquad
%   \lambda_{q_s} = K_x\frac{d{x}}{d{q_s}},
% \]

 \underline{\textbf{The second chapter}} is devoted to the study of a detector based on avalanche breakdown in a photodiode, which is used as measuring device for registering the result of interference of quantum states in a subcarrier-wave QKD system. This type of SPD is usually used for mid range distances of up to 100 ~ km with losses in communication lines less than 15~dB.

It's main advantages are following:
\begin{enumerate}
	\item High frequency trigger pulses - up to 100~MHz.
	\item The ability to supply strobe pulses from an external device (External gating mode).
	\item Wide range of setting the width of the operation window (gate) - from 0.5~ns to 25~ns.
	\item Setting the delay of opening the response window relative to the gate pulse (Trigger delay) in the range up to 10~ns with high resolution in time - 10~ps.
	\item Ability to set <<dead time>> in a wide range - from 0.1~μs to 100~μs.
	\item The ability to adjust quantum efficiency in increments of 2.5~\% in the range from 5~\% to 25~\%.
	\item The semiconductor structure of an APD - InGaAs/InP.
	\item Relatively low level of dark counting with given parameters of quantum efficiency.
\end{enumerate}

In the course of the study, to ensure realistic conditions for the eavesdropper to attack the measuring device as part of the SCW QKD system, it was considered as a <<black box>> and it was not opened and no manipulations were made with internal boards and microcircuits. All detector settings were set in accordance with the standard mode for SCW QKD system.

In order to successfully carry out a FSA, an eavesdropper needs to manipulate the detector
 in a way of forcing the clicks and their absence at the right time. It is assumed that the type of measuring device used is known to the eavesdropper, but there is no direct access to it. In that case the model is limited by the possibility of influencing the detector only by optical methods directly from the quantum channel.

It is known that APD in the linear mode, when a constant optical power is supplied to it, the photocurrent increases. Therefore, when a constant value is applied, the reverse bias voltage $-V_{bias}$ and in the presence of a resistor extinguishing the avalanche in the circuit, the voltage drop across the resistor increases, and on APD is reduced. The essence of the attack lies in switching the detector from the Geiger mode, or <<blinding>> it. Thus, it's operating mode is shifted relative to the breakdown voltage of the APD. With this approach, even additional pulses of $V_{gate}$ become insufficient and the diode is always in the mode of linear dependence of the photocurrent on the incident optical power.  

Nevertheless, in linear mode, it remains possible to exceed the threshold value of the photocurrent $I_{det}$ and generate a detector impulse.

Thus, the technique of removing the detector from the photon-counting mode to the linear mode for carrying out the FSA is easily formalized. An experimental study of the vulnerability of the SPD to this type of attack is carried out in three stages, shown in the figure \ref{fig:Method_2.3}:
%
 \begin{figure}[ht] 
  \centering
  \includegraphics[scale=0.5]{Method_2.3.eps}
  \caption{Blinding methods}
  \label{fig:Method_2.3}
\end{figure}
%
 \begin{enumerate}
	\item Determining the constant optical power sufficient to switch the detector from the Geiger mode.
	\item Optical pulse adjustment for the SPD gate
	\item Dependence of the click probability on the photon energy in the trigger pulse value.
\end{enumerate}
%
As a result of the experimental study detector click probabilities were obtained (fig. \ref{fig:Probability_vs_Energy}). 

\begin{figure}[ht]
  \centering
  \includegraphics[scale=0.5]{Probability_vs_Energy.png}
  \caption{Detector click probability as a function of trigger pulse energy}
  \label{fig:Probability_vs_Energy}
\end{figure}


Thus, it has been shown that the use of SPD based on APD in the Geiger mode (id210) with a gating frequency of 100~MHz requires the use of additional means of protection against attack with the switching from the Geiger mode using short optical pulses with an energy of at least 15.4~fJ and at a constant level of optical illumination with an average radiation power level of at least 35~nW. 

The model of the FSA on SCW~QKD systems is proposed in  \underline{\textbf{the third chapter}}. The applicability limits of this attack are estimated taking into account the characteristics and equipment parameters in the receiver unit. To determine the eavesdropper attempts to carry out the attack described above, it is proposed to measure the intensity of the central optical mode. The central mode is reflected by a narrow-band filter based on the Bragg grating. In order to be able to measure it, a fiber optic circulator is installed in the circuit. All radiation, which, after phase modulators in the receiving unit, enters the first port of the circulator is sent to the second port to the optical filter. The reflected central mode is sent to the third port of the circulator. In order to measure the central one, which, as shown in the section, should be of the order of 700 ~ nW, it is proposed to use a monitor photodiode. As. If the eavesdropper has the opportunity to select the intensity on the central mode, it is obvious that the monitor photodiode installed at the output from the 3rd port can also be controlled by optical methods. In order to avoid this, it is proposed to use a fiber optic mirror, for example a widely available Faraday mirror, and a fixed passive optical attenuator. The monitor photodiode itself is proposed to be installed in the 4th port of the circulator. Thus, the radiation with the central mode arrives at the 3rd port, passes the attenuator in the forward direction, is reflected from the fiber mirror, passes the attenuator in the opposite direction, and arrives at the 4th port where the monitor photodiode is installed.       
 \begin{figure}[ht]
  \centering
  \includegraphics[scale=0.25]{scw-setup_Countermeasure.pdf}
  \caption{Model of an eavesdropper for FSA implementation}
  \label{fig:countermeasure}
\end{figure}


The figure \ref{fig:Watchdog_photodiode} shows two dependencies: the lower limit of the optical power required to switch the photon detector from the Geiger mode and is the minimum required for a successful FSA by the eavesdropper; the upper limit is the optical power value, which is based on the parameters of the receiving module of the quantum communication system and the upper limit on the optical power of the control pulses. The minimum level in accordance with the calculation is 0.7 ~ μW, and the maximum is of the order of 300 ~ μW (within the framework of this study), but in fact it is limited by the threshold value of the APD inside the DOF, at which the current becomes too large and the diode burns out, incapacitating the whole device. Typical quantities are units and tens of milliwatts.
 
Based on the dependencies we can estimate the attenuation in a fixed attenuator, which will be sufficient to protect the monitor photodiode from exposure to light, and additional measures will not be required to expand the dynamic range of sensitivity to the input optical power in the direction of increase, since at optical powers below few nanowatts the measurement error becomes quite high.

Based on this, the attenuation value of 10~dB is sufficient, since the reflected central optical mode at the output from the 3rd port of the circulator passes through the attenuating element for the first time, and then after reflection from the mirror for the second time. The resulting attenuation by 20~dB, or 100 times, will reduce the minimum power level at the center frequency, which is enough to remove the detector from the photon counting mode, to a value of the order of 7~nW, which satisfies the requirements described above. The maximum level (within the framework of this researc) will decrease to about 3~μW.


 \begin{figure}[ht]
  \centering
  \includegraphics[scale=0.5]{images/Watchdog_photodiode.png}
  \caption{Optical power on monitoring photodiode}
  \label{fig:Watchdog_photodiode}
\end{figure}


The third chapter shows that measuring the amount of optical radiation at the carrier frequency reflected from the optical filter using a monitor photodiode in the receiving unit of the quantum communication system at side frequencies in the range from 7~nW to 2.93~μW using additional countermeasures in the form of a passive optical attenuator with a losses nominal value of 10 dB to protect it allows you to withstand an attack by switching SPD from Geiger mode and FSA.

% Можно сослаться на свои работы в автореферате. Для этого в файле
% \verb!Synopsis/setup.tex! необходимо присвоить положительное значение
% счётчику \verb!\setcounter{usefootcite}{1}!. В таком случае ссылки на
% работы других авторов будут подстрочными.
% \ifnumgreater{\value{usefootcite}}{0}{
% Изложенные в третьей главе результаты опубликованы в~\cite{vakbib1, vakbib2}.
% }{}
% Использование подстрочных ссылок внутри таблиц может вызывать проблемы.

 \underline{\textbf{The fourth chapter}} describes the coherent states that are widely used in QCS, because coherent lasers are widely used as a radiation source.
 
  A distinctive feature of SCW QKD systems is the generation of multimode coherent states at different optical modes, depending on the frequency of the modulating signal, as shown in the figure \ref{fig:multimodes}. 

 \begin{figure}[ht]
  \centering
  \includegraphics[scale=0.4]{Modes_rus.pdf}
  \caption{Sidebands generation scheme}
  \label{fig:multimodes}
\end{figure}


Define these states as follows. Input (unmodulated) state at sender's sides (Alice) and (Bob) (denoted, as $A$ or $B$) are denoted, as $|\sqrt{\mu_0}\rangle_0\otimes|\mathrm{vac}\rangle_{SB}$, where $|\mathrm{vac}\rangle_{SB}$ is the sideband's vacuum state and $|\sqrt{\mu_0}\rangle_0$ is a carrier's coherent state with an amplitude (mean photon number per pulse $\mu_0$). The carrier is formed by coherent light beam on a frequency $\omega$. Carrier's phase is supposed to be reference and other phase shifts denotes with respect to it. Электро-оптический фазовый модулятор (с частотой колебаний микроволнового поля $\Omega$ и её фазой $\varphi_A$ или $\varphi_B$) перераспределяет энергию между взаимодействующими модами (поле на выходе модулятора приобретает боковые частоты $\omega_k=\omega+k\Omega$, ограничим рассматриваемый нами случай $2S$ боковыми частотами и пусть целое число $k$ мод ограничено пределами $-S\le k\le S$), так, что состояние поля на выходе модулятора - это многомодовое когерентное состояние: 
%
\begin{equation}\label{phi}
|\psi_0(\varphi_j)\rangle = \bigotimes_{k=-S}^S|{\alpha_k(\varphi_j)}\rangle_k,
\end{equation}
%
где $j$ это и $A$, и $B$ (определяющее Алису или Боба), а амплитуды имеют следующий вид: 
%
\begin{equation}\label{alpha}
\alpha_k(\varphi_j)=\sqrt{\mu_0}d^S_{0k}(\beta)e^{i\varphi_jk},
\end{equation}
%
и $d^S_{nk}(\beta)$ это d-функция Вигнера, взятая из квантовой теории углового момента %\cite{varshalovich1988quantum}
, $\beta$ определяется индексом модуляции $m$, который без учета дисперсии в среде модулятора можно выразить: 
%
\begin{equation}\label{betam}
\cos{({\beta})}=1-\frac{1}{2}{\left(\frac{m}{S+0.5}\right)^2},
\end{equation}
где $S$ количество взаимодействующий мод, принимаемое очень большим.

После подстройки относительной фазы оптических сигналов в двух плечах, наблюдается интерференция этих состояний на втором светоделителе ($BS2$). Описание состояний на выходах светоделителя дано в уравнении~\ref{states}. Далее предполагается, что относительная фазовая отстройка оптических импульсов равна $\phi_0\approx0$. Если разность фаз между радиочастотными модулирующими сигналами Алисы и Боба равна нулю ($\Delta\varphi=0$) весь спектр идет в одно выходное плечо светоделителя $BS2$, в ином случае, четные моды спектра, включая центральную, идут в то же плечо, а нечетные идут во второе плечо. В случае $\Delta\varphi=0$, требуется спектральное разделение центральной частоты и боковых в приёмном блоке, так как кодирование квантовых состояний происходит на боковых частотах. Следует учитывать, что при использовании малого среднего числа фотонов в импульсе, значительный вклад в многомодовых состояниях вносят только первая пара боковых частот. Случай, при котором $\Delta\varphi=\pi$, является нетривиальным. Многомодовое состояние разделяется на $BS2$ и центральная мода (и все четные) идут в первое плечо, а все нечетные (первые боковые частоты вносят наибольший вклад в результирующий сигнал) идут во второе плечо, как показано на рисунке \ref{fig:Interference_result}. Таким образом, можно отказаться от спектрального разделения при помощи оптического фильтра в одном из выходных плечей приёмного узла. Показателем хорошей подстройки по фазе оптических импульсов является постоянный высокий уровень на центральной частоте в первом плече (на $D2$).  


\begin{figure}[ht]
 \centering
  \includegraphics[scale=0.5]{Interference_result.png}
  \caption{Принципиальная схема наблюдения результата интерференции когерентных состояний}
  \label{fig:Interference_result}
\end{figure}

 
В четвертой главе показано, что метод квантовой коммуникации на боковых частотах позволяет реализовывать протокол, устойчивый к контролю нелегитимным пользователем измерительного оборудования. 

Основываясь на полученных результатах можно сформулировать протокол для системы квантовой коммуникации, использующей многомодовые когерентные состояния, с недоверенным измерительным узлом (рис. \ref{fig:Protocol}). Для простоты приведем пример с использованием только двух фазовых состояний по аналогии с протоколом B92. Алиса случайным образом выбирает одно фазовое состояние из двух возможных <<$0$>> или <<$\pi$>>. Боб независимо от Алисы тоже случайным образом выбирает одно из двух фазовых состояний. В результате интерференции многомодовых когерентных состояний можно наблюдать 4 различных варианта, в зависимости от разности фаз, выбранных Алисой и Бобом.  При этом на недоверенном узле регистрации будет происходить спектральное разделение, так что на $D1$ и $D2$ будут поступать боковые частоты. Злоумышленник при этом имеет полный доступ к недоверенному измерительному узлу и может фиксировать себе оглашенный результат срабатываний одного из детекторов. Положим, что нас интересует только случай, при котором разность фаз между $A$ и $B$ равна $\pi$. Преимущество этого случая в том, что за счет спектрального разделения в результате интерференции в плечо с детектором $D1$ всегда будут идти только боковые и никогда не будет идти центральная мода, а значит можно исключить из оптической схемы этого плеча оптический фильтр, согласование которого с фильтром в другом плече -- достаточно трудная инженерная задача. Таким образом, при просеивании остаются только те биты, которые соответствуют отсчетам на $D1$, при этом для корреляции между битами Алисы и битами Боба последний должен сделать смену бита на противоположный (так называемый <<bitflip>>).  

\begin{figure}[ht]
 \centering
  \includegraphics[scale=0.5]{Protocol.png}
  \caption{Протокол}
  \label{fig:Protocol}
\end{figure}
 
  В \underline{\textbf{пятой главе}} показано, что в результате интерференции квантового фазомодулированного сигнала на боковых частотах на симметричном светоделителе в схеме квантовой рассылки ключа с узлом регистрации, независящим от легитимного пользователя, происходит спектральное разделение квантового сигнала и сигнала на центральной длине волны с их независимой регистрацией в разных плечах светоделителя. 
  
  Для проверки концепции была собрана оптическая схема, представленная на рисунке \ref{fig:RF_sin}. Для простоты использовался только один источник излучения, сигнал на выходе которого делился светоделителем 50:50 $BS1$. Такой подход позволяет имитировать случай, когда у Алисы и Боба хорошо скоррелированные источники излучения. В данном эксперименте применялся лазер с распределенной обратной связью Л1 (TTX1994 Neophotonics), отличительными особенностями которого являются очень узкая ширина спектральной линии порядка 100~кГц и широкий диапазон перестройки длин волн (от 1530~нм до 1565,5~нм) с высокой степенью точности (20~пм). Выходная мощность при этом составляла 10~мВт. Однако, это значение зависит от глубины динамического диапазона перестраиваемого оптического аттенюатора ПОА и суммарных потерь, вносимых элементами Алисы или Боба, так как результирующая мощность сигнала на боковых частотах должна не превышать величину 2,56~пВт, соответствующую среднему числу фотонов в импульсе 0,2 при частоте смены состояний 100~Мгц.   

Из-за чувствительности резонатора внутри лазера с распределенной обратной связью к обратному относительно выходного оптическому излучению, то есть переотражениям и рассеиванию сигналов, в оптической схеме применяется волоконный изолятор И на основе эффекта Фарадея с величиной изоляции порядка 50~дБ. Для перевода системы в однофотонный режим применяется ПОА с достаточной малым шагом перестройки 0,05 дБ и большим динамическим диапазоном величиной 60~дБ.  

  

  \begin{figure}[ht]
  \centering
  \includegraphics[scale=0.4]{Scheme_colored_rus.png}
  \caption{Принципиальная схема экспериментального стенда}
  \label{fig:RF_sin}
\end{figure}

  
 
  
На выходе из источника сигнал проходил по двум плечам формируемого таким образом интерферометра через светоделитель, благодаря чему имитировалось, что Алиса и Боб - это раздельные узлы, в которых информация кодировалась с помощью фазовых модуляторов. Выходы первого светоделителя подключались ко входам двух независимых 10~ГГц электро-оптических модуляторов на основе $LiNbO_3$ (со встроенным поляризатором) $PM1$ and $PM2$. Поляризаторы снижали чувствительность модуляторов к поляризации входного излучения, для чего использовался контроллер поляризации КП, и таким образом, повышалась видность картины интерференции. Электрические входы $PM1$ и $PM2$ подключались к выходам с цифро-аналоговых преобразователей ЦАП с модулирующим радиочастотным синусоидальным сигналом (с частотой 4,8~ГГц). Амплитуды управляющих сигналов подбирались таким образом, чтобы мощность сигнала на боковых частотах была равно у Алисы и у Боба

Индекс модуляции, который показывает долю энергии на боковых частотах по отношению к центральной в результате модуляции должен быть 5~\%. В таком случае наблюдается оптимальное соотношение сигнала на боковых частотах к шуму проходящей через оптический фильтр центральной частоты.  Так что средняя величина оптической мощности была равна 2.56~пВт, что соответствует $\mu=0.2$.  


В данном исследовании  использовались только два кодирующих фазовых состояния на Алисе и Бобе ($\varphi_{A,B}\in\{0,\pi\}$. Разница фаз ($\Delta\varphi$) между Алисой и Бобом формировалась с помощью изменения IQ-таблиц ЦАП при помощь ПЛИС и ПО сосбтвенной разработки. В случае измерения зависимости количества срабатываний детекторов от разности фаз ($\Delta\varphi$) фаза сдвигалась последовательно с шагом $\varphi_{step}\ = 10^{\circ}$. Наблюдаемые состояния на выходах $PM1$ и $PM2$ могут быть описаны в соотвествии с уравнением~\ref{phi}.

После того, как состояния приготовлены, мы посылаем их в квантовый канал. Для компенсации оптической разности хода в двух плечах интерферометра, а так же для точной подстройки оптических фаз сигналов использовались $PM3$ и $PM4$. Эта подстройка осуществлялась при помощи изменения постоянного напряжения, подаваемого на модуляторы и формируемого двумя независимыми электрическими выходами генератора сигналов $GEN1$ и $GEN2$. Выходы втрой пары фазовых модуляторов подключались к паре входных портов светоделителя $BS2$ 2x2 с коэффициентом деления 50:50.

Измерения проводились в два этапа: первый этап в классическом свете, второй этап - в режиме счета фотонов на сверхпроводниковом детекторе одиночных фотонов (Сконтел), у которого встроены два независимых приёмника $D1$, $D4$. Квантовая эффективность обоих была равна 10~\%, а уровень темновых шумов не превышал 50~Гц для каждого. Измеренная величина суммарных шумов, включающих в себя темновые срабатывания и засветку от центральной частоты в силу ограниченной экстинкции оптического фильтра, составляла $\gamma_{темн}=1.5$ кГц.

Показаны зависимости интенсивности на боковых частотах в результате интерференции от разности фаз модулирующих сигналов в классическом режиме и в режиме счета фотонов. Оценены основные параметры, характеризующие систему квантовой коммуникации: квантовый коэффициент ошибок по битам (QBER) и среднее значение скорости формирования просеянного ключа $K$ (одинаковые биты между легитимными пользователями, но коррелирующие с возможным результатом у злоумышленника, что требует проведения процедуры усиления секретности) 



 \underline{\textbf{Заключение}} содержит список основных результатов, полученных в работе. 

% !_____!
  %% Согласно ГОСТ Р 7.0.11-2011:
%% 5.3.3 В заключении диссертации излагают итоги выполненного исследования, рекомендации, перспективы дальнейшей разработки темы.
%% 9.2.3 В заключении автореферата диссертации излагают итоги данного исследования, рекомендации и перспективы дальнейшей разработки темы.
\begin{enumerate}
  \item На основе экспериментального анализа детектора, работающего в режиме Гейгера, показано, что требуются дополнительные средства защиты от атаки с выведением из режима Гейгера при помощи коротких оптических импульсов с энергией не менее 15,4 фДж и при постоянном уровне оптической засветки средним уровнем мощности излучения не менее 35 нВт. 
  \item Численные исследования показали, что измерение величины оптического излучения на несущей частоте, отраженного от оптического фильтра, при помощи мониторного фотодиода в приемном блоке системы квантовой коммуникации на боковых частотах в диапазоне от 7 нВт до 2,93 мкВт с применением дополнительных мер в виде пассивного оптического аттенюатора номиналом 10 дБ для его защиты позволяет противостоять атаке с выведением детектора одиночных фотонов из режима Гейгера и навязыванием ключа нелегитимным пользователем. 
  \item Метод квантовой коммуникации на боковых частотах позволяет реализовывать протокол, устойчивый к контролю нелегитимным пользователем измерительного оборудования.
  \item Для выполнения поставленных задач был создан экспериментальный стенд и в результате интерференции квантового фазомодулированного сигнала на боковых частотах на симметричном светоделителе в схеме квантовой рассылки ключа с узлом регистрации, независящим от легитимного пользователя, наблюдается спектральное разделение квантового сигнала и сигнала на центральной длине волны с их независимой регистрацией в разных плечах светоделителя. 
\end{enumerate}

%	\newcommand{\publications}{\underline{\textbf{\publicationsTXT}}}

{The main results of the dissertation are listed in following publications, included in Scopus and Web of Science:}
\begin{enumerate}\addtolength{\itemsep}{-0.5\baselineskip}
\renewcommand{\labelenumi}{[\theenumi]}
\item Vladimir Chistiakov, Anqi Huang, Vladimir Egorov, and Vadim Makarov, Controlling single-photon detector ID210 with bright light, Opt. Express 27, 32253-32262 (2019)
\\
\item Чистяков В.В., Гайдаш А.А., Козубов А.В., Глейм А.В. Исследование интерференции слабых когерентных многомодовых состояний для задач квантовой коммуникации с недоверенным приемным узлом // Научно-технический вестник информационных технологий, механики и оптики. 2019. Т. 19. № 6. doi: 10.17586/2226-1494-2019-19-6
\\
\item    Gleim A.V., Egorov V.I., Nazarov Y.V., Smirnov S.V., Chistyakov V.V., Bannik O.I., Anisimov A.A., Kynev S.M., Ivanova A.E., Collins R.J., Kozlov S.A., Buller G. Secure polarization-independent subcarrier quantum key distribution in optical fiber channel using BB84 protocol with a strong reference//Optics express, IET - 2016, Vol. 24, No. 3, pp. 2619-2633
\\
\item  Глейм А.В., Егоров В.И., Чистяков В.В., Смирнов С.В., Банник О.И., Булдаков Н.В., Гайдаш А.А., Козубов А.В., Васильев А.Б., Кынев С.М., Хоружников С.Э., Козлов С.А., Васильев В.Н. Квантовая коммуникация на боковых частотах со скоростью 1 Мбит/с в городской сети // Оптический журнал -2017. - Т. 84. - № 6. - С. 3-9
\\
\item  Chistyakov V.V., Kynev S.M, Smirnov S.V., Nazarov Y.V., Gleim A.V. Achieving high visibility in subcarrier wave quantum key distribution system // Journal of Physics: Conference Series, IET - 2016, Vol. 735, No. 1, pp. 012085
\\
\item V. V. Chistyakov, A. V. Gleim, V. I. Egorov, Yu. V. Nazarov. Implementation of multiplexing in a subcarrier-wave quantum cryptography system // Journal of Physics: Conference Series - 2014  vol. 541,  pp. 012078
\\
\item   Kynev S.M., Chistyakov V.V., Smirnov S.V., Volkova K.P., Egorov V.I., Gleim A.V. Free-space subcarrier wave quantum communication // Journal of Physics: Conference Series - 2017, Vol. 917, No. 5, pp. 052003
\\

\item    Gleim A.V., Nazarov Y.V., Egorov V.I., Smirnov S.V., Bannik O.I., Chistyakov V.V., Kynev S.M., Anisimov A.A., Kozlov S.A., Vasil'ev V.N. Subcarrier Wave Quantum Key Distribution in Telecommunication Network with Bitrate 800 kbit/s//EPJ Web of Conferences, IET - 2015, Vol. 103, pp. 10005
\\
\item    Gleim A.V., Egorov V.I., Nazarov Y.V., Smirnov S.V., Chistyakov V.V., Bannik O.I., Anisimov A.A., Kynev S.M., Collins R.J., Kozlov S.A., Buller G.S. Polarization insensitive 100 MHz clock subcarrier quantum key distribution over a 45 dB loss optical fiber channel // Conference on Lasers and Electro-Optics, CLEO 2015, IET - 2015, pp. 7182997
\\
\item Gaidash A.A., Kozubov A.V., Chistyakov V.V., Miroshnichenko G.P., Egorov V.I., Gleim A.V. Security conditions for sub-carrier wave quantum key distribution protocol in errorless channel // Journal of Physics: Conference Series - 2017, Vol. 917, No. 6, pp. 062014
\\
%\item  Gleim A.V., Chistyakov V.V., Bannik O.I., Egorov V.I., Buldakov N.V., Vasilev A.B., Gaidash A.A., Kozubov A.V., Smirnov S.V., Kynev S.M., Khoruzhnikov S.E., Kozlov S.A., Vasil'ev V.N. Sideband quantum communication at 1 Mbit/s on a metropolitan area network // Journal of Optical Technology - 2017, Vol. 84, No. 6, pp. 362-36
\\

\end{enumerate}
\noindent{Other publications:}
\begin{enumerate}\addtolength{\itemsep}{-0.5\baselineskip}
\renewcommand{\labelenumi}{[\theenumi]}
\setcounter{enumi}{9}
\item   Чистяков В.В., Кынев С.М., Смирнов С.В., Назаров Ю.В., Глейм А.В. Обеспечение высокой видности в системе квантовой криптографии на боковых частотах // Сборник трудов IX международной конференции молодых ученых и специалистов «Оптика – 2015», с. 658-660
\\
\item Глейм А.В., Назаров Ю.В., Егоров В.И., Чистяков В.В, Смирнов С.В., Банник О.И., Кынев С.М., Иванова А.Е., Дубровская В.Д., Тарасов М.Г., Булдаков Н.В., Кузьмина Т.Б., Чивилихин С.А., Анисимов А.А., Рощупкин С.В., Рогачёв К.С., Хоружников С.Э., Козлов С.А., Васильев В.Н. Создание квантовой сети университета ИТМО //Сборник трудов VIII международной конференции «Фундаментальные проблемы оптики – 2014». Санкт-Петербург, 20-24 октября ,2014, С.3-4,  541 с. 
\\
\item А.В. Глейм, В.И.Егоров, А.А. Анисимов, Ю.В. Назаров, С.М. Кынев, А.В. Рупасов, В.В. Чистяков, А.А.Гайдаш, М.А. Смирнов, С.А. Чивилихин, С.А. Козлов Квантовая рассылка криптографического ключа по оптическому волокну телекоммуникационного стандарта на расстояние 200 км со скоростью 0.18 кбит/с // Cборник трудов III Всероссийская конференция по фотонике и информационной оптике Москва, НИЯУ МИФИ, 2014 с. 17-19
\\


\end{enumerate}


%При использовании пакета \verb!biblatex! список публикаций автора по теме
%диссертации формируется в разделе <<\publications>>\ файла
%\verb!../common/characteristic.tex!  при помощи команды \verb!\nocite!

\ifdefmacro{\microtypesetup}{\microtypesetup{protrusion=false}}{} % не рекомендуется применять пакет микротипографики к автоматически генерируемому списку литературы
\ifnumequal{\value{bibliosel}}{0}{% Встроенная реализация с загрузкой файла через движок bibtex8
  \renewcommand{\bibname}{\large \authorbibtitle}
  \nocite{*}
  \insertbiblioauthor           % Подключаем Bib-базы
  %\insertbiblioother   % !!! bibtex не умеет работать с несколькими библиографиями !!!
}{% Реализация пакетом biblatex через движок biber
  \ifnumgreater{\value{usefootcite}}{0}{
%  \nocite{*} % Невидимая цитата всех работ, позволит вывести все работы автора
  \insertbiblioauthorcited      % Вывод процитированных в автореферате работ автора
  }{
  \insertbiblioauthor           % Вывод всех работ автора
%  \insertbiblioauthorgrouped    % Вывод всех работ автора, сгруппированных по источникам
%  \insertbiblioauthorimportant  % Вывод наиболее значимых работ автора (определяется в файле characteristic во второй section)
  \insertbiblioother            % Вывод списка литературы, на которую ссылались в тексте автореферата
  }
}
\ifdefmacro{\microtypesetup}{\microtypesetup{protrusion=true}}{}
