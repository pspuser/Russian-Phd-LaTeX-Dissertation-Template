\chapter*{Реферат}
\addcontentsline{toc}{chapter}{Реферат} 
\section*{Общая характеристика работы}

	\newcommand{\actuality}{\underline{\textbf{\actualityTXT}}}
	\newcommand{\progress}{\underline{\textbf{\progressTXT}}}
	\newcommand{\aim}{\underline{{\textbf\aimTXT}}}
	\newcommand{\tasks}{\underline{\textbf{\tasksTXT}}}
	\newcommand{\novelty}{\underline{\textbf{\noveltyTXT}}}
	\newcommand{\influence}{\underline{\textbf{\influenceTXT}}}
	\newcommand{\methods}{\underline{\textbf{\methodsTXT}}}
	\newcommand{\defpositions}{\underline{\textbf{\defpositionsTXT}}}
	\newcommand{\reliability}{\underline{\textbf{\reliabilityTXT}}}
	\newcommand{\probation}{\underline{\textbf{\probationTXT}}}
	\newcommand{\contribution}{\underline{\textbf{\contributionTXT}}}
	\newcommand{\publications}{\underline{\textbf{\publicationsTXT}}}
	

{\actuality} \todo{поменять с 1 режим черновика на 0 в setup.tex для соблюдения ГОСТ}

Обзор, введение в тему, обозначение места данной работы в
мировых исследованиях и~т.\:п., можно использовать ссылки на~другие
работы\ifnumequal{\value{bibliosel}}{1}{~\autocite{Gosele1999161}}{}
(если их~нет, то~в~автореферате
автоматически пропадёт раздел <<Список литературы>>). Внимание! Ссылки
на~другие работы в разделе общей характеристики работы можно
использовать только при использовании \verb!biblatex! (из-за технических
ограничений \verb!bibtex8!. Это связано с тем, что одна
и~та~же~характеристика используются и~в~тексте диссертации, и в
автореферате. В~последнем, согласно ГОСТ, должен присутствовать список
работ автора по~теме диссертации, а~\verb!bibtex8! не~умеет выводить в одном
файле два списка литературы).
При использовании \verb!biblatex! возможно использование исключительно
в~автореферате подстрочных ссылок
для других работ командой \verb!\autocite!, а~также цитирование
собственных работ командой \verb!\cite!. Для этого в~файле
\verb!Synopsis/setup.tex! необходимо присвоить положительное значение
счётчику \verb!\setcounter{usefootcite}{1}!.

Для генерации содержимого титульного листа автореферата, диссертации
и~презентации используются данные из файла \verb!common/data.tex!. Если,
например, вы меняете название диссертации, то оно автоматически
появится в~итоговых файлах после очередного запуска \LaTeX. Согласно
ГОСТ 7.0.11-2011 <<5.1.1 Титульный лист является первой страницей
диссертации, служит источником информации, необходимой для обработки и
поиска документа>>. Наличие логотипа организации на~титульном листе
упрощает обработку и~поиск, для этого разметите логотип вашей
организации в папке images в~формате PDF (лучше найти его в векторном
варианте, чтобы он хорошо смотрелся при печати) под именем
\verb!logo.pdf!. Настроить размер изображения с логотипом можно
в~соответствующих местах файлов \verb!title.tex!  отдельно для
диссертации и автореферата. Если вам логотип не~нужен, то просто
удалите файл с~логотипом.

% \ifsynopsis
% Этот абзац появляется только в~автореферате.
% Для формирования блоков, которые будут обрабатываться только в~автореферате,
% заведена проверка условия \verb!\!\verb!ifsynopsis!.
% Значение условия задаётся в~основном файле документа (\verb!synopsis.tex! для
% автореферата).
% \else
% Этот абзац появляется только в~диссертации.
% Через проверку условия \verb!\!\verb!ifsynopsis!, задаваемого в~основном файле
% документа (\verb!dissertation.tex! для диссертации), можно сделать новую
% команду, обеспечивающую появление цитаты в~диссертации, но~не~в~автореферате.
% \fi

% {\progress}
% Этот раздел должен быть отдельным структурным элементом по
% ГОСТ, но он, как правило, включается в описание актуальности
% темы. Нужен он отдельным структурынм элемементом или нет ---
% смотрите другие диссертации вашего совета, скорее всего не нужен.

{\aim} данной работы является оценка возможностей злоумышленника по получению секретного ключа с использованием атак на измерительное оборудование систем квантовой коммуникации на боковых частотах и разработка методов противодействия на основе результатов оценки.


Для~достижения поставленной цели необходимо было решить следующие {\tasks}:
\begin{enumerate}
  \item Исследование устойчивости детектора одиночных фотонов, применяемого в системах квантовой коммуникации на боковых частотах, к атакам с выведением из режима Гейгера (<<ослеплением>>). 

  \item Оценка возможностей злоумышленника при атаке с выведением из режима Гейгера для систем квантовой коммуникации на боковых частотах. 

  \item Разработка оптической схемы системы квантовой коммуникации, устойчивой к атакам на измерительное оборудование. 

  \item Разработка протокола квантовой рассылки ключа, устойчивого к атаке на 						измерительное оборудование. 

\end{enumerate}


{\novelty}
\begin{enumerate}
  \item Впервые исследована устойчивость системы квантовой коммуникации на боковых частотах к атакам злоумышленника на измерительное оборудование приёмного блока. 
  \item Впервые предложена и применена контрмера против атаки злоумышленника с выведением детектора из режима Гейгера и навязыванием легитимным пользователями ключа. 
  \item Было выполнено оригинальное исследование и разработана оптическая схема с вынесением устройства детектирования одиночных фотонов из приемного блока в <<недоверенный>> узел, подконтрольный злоумышленнику 
\end{enumerate}

{\influence} \ldots

{\methods} \ldots

{\defpositions}
\begin{enumerate}
  \item Использование коммерческих детекторов одиночных фотонов на основе лавинных фотодиодов в режиме Гейгера модели id210 с частотой стробирования 100 МГц  требует применения дополнительных средств защиты от атаки с выведением из режима Гейгера.   
  \item \VCc{ КОНТРМЕРА }
  \item Метод квантовой коммуникации на боковых частотах позволяет реализовывать протокол, устойчивый к контролю нелегитимным пользователем измерительного оборудования. 
  \item В результате интерференции квантового фазомодулированного сигнала на боковых частотах на симметричном светоделителе в схеме квантовой рассылки ключа с узлом регистрации, независящим от легитимного пользователя, происходит спектральное разделение квантового сигнала и сигнала на центральной длине волны с их независимой регистрацией в разных плечах светоделителя. 
  \item Показана возможность двукратного увеличения дальности квантовой рассылки ключа на боковых частотах посредством применения недоверенной системы регистрации квантовых состояний. 
\end{enumerate}
% В папке Documents можно ознакомиться в решением совета из Томского ГУ
% в~файле \verb+Def_positions.pdf+, где обоснованно даются рекомендации
% по~формулировкам защищаемых положений.

{\reliability} полученных результатов обеспечивается применением утверждённых методик проведений экспериментальных исследований и аттестованного оборудование. Математическое моделирование и обработка данных, полученных в результате экспериментов, осуществлялось с использованием пакетов прикладных программ MathCad и Excel. Результаты находятся в соответствии с результатами, полученными другими авторами.


{\probation}
Основные результаты работы докладывались~на:
перечисление основных конференций, симпозиумов и~т.\:п.

{\contribution} Автор принимал активное участие \ldots

%\publications\ Основные результаты по теме диссертации изложены в ХХ печатных изданиях~\cite{Sokolov,Gaidaenko,Lermontov,Management},
%Х из которых изданы в журналах, рекомендованных ВАК~\cite{Sokolov,Gaidaenko},
%ХХ --- в тезисах докладов~\cite{Lermontov,Management}.

\ifnumequal{\value{bibliosel}}{0}{% Встроенная реализация с загрузкой файла через движок bibtex8
    \publications\ Основные результаты по теме диссертации изложены в XX печатных изданиях,
    X из которых изданы в журналах, рекомендованных ВАК,
    X "--- в тезисах докладов.%
}{% Реализация пакетом biblatex через движок biber
%Сделана отдельная секция, чтобы не отображались в списке цитированных материалов
    \begin{refsection}[vak,wos,scopus,papers,conf]% Подсчет и нумерация авторских работ. Засчитываются только те, которые были прописаны внутри \nocite{}.
        %Чтобы сменить порядок разделов в сгрупированном списке литературы необходимо перетасовать следующие три строчки, а также команды в разделе \newcommand*{\insertbiblioauthorgrouped} в файле biblio/biblatex.tex
        \printbibliography[heading=countauthorvak, env=countauthorvak, keyword=biblioauthorvak, section=1]%
        \printbibliography[heading=countauthorwos,env=countauthorwos, keyword=biblioauthorwos, section=1]%
        \printbibliography[heading=countauthorscopus,env=countauthorscopus, keyword=biblioauthorscopus, section=1]%
	\printbibliography[heading=countauthorconf, env=countauthorconf, keyword=biblioauthorconf, section=1]%
        \printbibliography[heading=countauthorothers, env=countauthorothers, keyword=biblioauthorothers, section=1]%
        \printbibliography[heading=countauthor, env=countauthor, keyword=biblioauthor, section=1]%
        \nocite{%Порядок перечисления в этом блоке определяет порядок вывода в списке публикаций автора
                vakbib1,vakbib2,%
		wosbib1,%
		scbib1,%
                confbib1,confbib2,%
                bib1,bib2,%
        }%
	\publications\ Основные результаты по теме диссертации изложены
	\setcounter{citeauthorscwostot}{\value{citeauthorscopus}} % вместе setcounter и addtocounter добавляют пробел между словами. По-этому они так раскиданы.
        в~\arabic{citeauthor}~печатных изданиях,
	\addtocounter{citeauthorscwostot}{\value{citeauthorwos}}
	\arabic{citeauthorvak} из которых изданы в журналах, рекомендованных ВАК\sloppy
	\ifnum \value{citeauthorscwostot}>0
	, \arabic{citeauthorscwostot} "--- в~периодических научных журналах, индексируемых Web of Science и Scopus\sloppy
	\fi
	\ifnum \value{citeauthorconf}>0
	, \arabic{citeauthorconf} "--- в~тезисах докладов.
	\else
	.
	\fi
    \end{refsection}
    \begin{refsection}[vak,wos,scopus,papers,conf]%Блок, позволяющий отобрать из всех работ автора наиболее значимые, и только их вывести в автореферате, но считать в блоке выше общее число работ
        \printbibliography[heading=countauthorvak, env=countauthorvak, keyword=biblioauthorvak, section=2]%
        \printbibliography[heading=countauthorwos, env=countauthorwos, keyword=biblioauthorwos, section=2]%
        \printbibliography[heading=countauthorscopus, env=countauthorscopus, keyword=biblioauthorscopus, section=2]%
        \printbibliography[heading=countauthorothers, env=countauthorothers, keyword=biblioauthorothers, section=2]%
        \printbibliography[heading=countauthorconf, env=countauthorconf, keyword=biblioauthorconf, section=2]%
        \printbibliography[heading=countauthor, env=countauthor, keyword=biblioauthor, section=2]%
        \nocite{vakbib2}%vak
        \nocite{bib1}%other
        \nocite{confbib1}%conf
    \end{refsection}
}
При использовании пакета \verb!biblatex! для автоматического подсчёта
количества публикаций автора по теме диссертации, необходимо
их~здесь перечислить с использованием команды \verb!\nocite!.
 % Характеристика работы по структуре во введении и в автореферате не отличается (ГОСТ Р 7.0.11, пункты 5.3.1 и 9.2.1), потому её загружаем из одного и того же внешнего файла, предварительно задав форму выделения некоторым параметрам

%Диссертационная работа была выполнена при поддержке грантов ...

%\underline{\textbf{Объем и структура работы.}} Диссертация состоит из~введения,
%четырех глав, заключения и~приложения. Полный объем диссертации
%\textbf{ХХХ}~страниц текста с~\textbf{ХХ}~рисунками и~5~таблицами. Список
%литературы содержит \textbf{ХХX}~наименование.

%	Полный объём диссертации составляет
%	\formbytotal{TotPages}{страниц}{у}{ы}{}, включая
%	\formbytotal{totalcount@figure}{рисун}{ок}{ка}{ков} и
%	\formbytotal{totalcount@table}{таблиц}{у}{ы}{}.   Список литературы содержит
%	\formbytotal{citenum}{наименован}{ие}{ия}{ий}.

 \section*{Содержание работы}
 Во \underline{\textbf{введении}} обосновывается актуальность
 исследований, проводимых в~рамках данной диссертационной работы,
  формулируется цель, ставятся задачи работы, излагается научная новизна
 и практическая значимость представляемой работы. 

 В \underline{\textbf{первой главе}}, имеющей обзорный характер, описаны методы и протоколы квантового распределения ключа. Показаны практические приложения, в том числе и решение проблемы <<последней мили>> посредством использования атмосферного канала для передачи квантовых состояний. Рассмотрены основные известные на данный момент способы регистрации одиночных фотонов для применения в системах квантовой коммуникации. Проведены анализ и сравнение сравнение различных типов детекторов фотонов, обоснован выбор исследуемого в работе детектора одиночных фотонов (ДОФ) на основе лавинного пробоя. Показаны возможные атаки потенциального злоумышленника на измерительное оборудование, а также приведены известные контрмеры против такого типа атак.   

%  картинку можно добавить так:
% \begin{figure}[ht]
%   \centering
%   \includegraphics [scale=0.27] {latex}
%   \caption{Подпись к картинке.}
 %  \label{fig:latex}
% \end{figure}

% Формулы в строку без номера добавляются так:
% \[
%   \lambda_{T_s} = K_x\frac{d{x}}{d{T_s}}, \qquad
%   \lambda_{q_s} = K_x\frac{d{x}}{d{q_s}},
% \]

 \underline{\textbf{Вторая глава}} посвящена исследованию детектора на основе лавинного пробоя в фотодиоде, который применяется в качестве измерительного оборудования для регистрации результата интерференции квантовых состояний в системе квантовой коммуникации на боковых частотах (ККБЧ) фазомодулированного излучения. Такой тип ДОФ обычно применяется для внутригородских дистанций до 100~км с потерями в линиях связи менее 15~дБ. 

Его отличительными особенностями являются:
\begin{enumerate}
	\item Поддержка высокой частоты стробирующих импульсов - до 100~МГц
	\item Возможность подачи стробирующих импульсов от внешнего устройства (External gating mode)
	\item Широкий диапазон настройки ширины окна срабатывания (gate) - от 0,5~нс до 25~нс
	\item Выставление задержки открытия окна срабатывания относительно стробирующего импульса (Trigger delay) в диапазоне до 10~нс с высоким разрешением во времени - 10~пс 
	\item Возможность выставления <<мертвого времени>> в широком диапазоне - от 0,1~мкс до 100~мкс
	\item Возможность регулировки квантовой эффективности с шагом 2,5~\% в диапазоне от 5~\% до 25~\%
	\item Полупроводниковая структура ЛФД - InGaAs/InP
	\item Относительно низкий уровень темнового счета при заданных параметрах квантовой эффективности
\end{enumerate}

В ходе исследования для обеспечения реалистичных условий атаки злоумышленника на измерительное оборудование в составе системы ККБЧ устройство рассматривалось, как <<черный ящик>>, то есть оно не вскрывалось и не производились манипуляции с внутренними платами и микросхемами. Все настройки детектора выставлялись в соответствии со штатным режимом для систем квантовой коммуникации на боковых частотах модулированного излучения. 

Для успешного осуществления атаки с навязыванием ключа злоумышленнику требуется манипулировать детектором, то есть форсировать срабатывания и их отсутствие в нужные моменты времени (такты), при этом предполагается, что тип применяемого измерительного оборудования известен, но непосредственный доступ к нему отсутствует. В таких рамках модель атаки ограничивается возможностью воздействия на детектор только оптическими методами непосредственно из квантового канала. 


Известно, что в линейном режиме работы ЛФД при подачи на него постоянной оптической мощности увеличивается фототок, следовательно при подаче постоянного значения напряжение обратного смещения $-V_{bias}$ и при наличии в цепи гасящего лавину резистора, величина падения напряжения на резисторе растет, а на ЛФД снижается. Суть атаки с выведением детектора из режима Гейгера, или <<ослеплением>> ДОФ, сводится к тому, чтобы сместить режим работы относительно напряжения пробоя ЛФД. При таком подходе даже дополнительных импульсов $V_{gate}$ становится недостаточно и диод все время находится в режиме линейной зависимости фототока от величины мощности оптического излучения, падаюшего на него.  

Тем не менее, в линейном режиме остается возможность превысить пороговое значение фототока $I_{det}$ и сформировать импульс срабатывания детектора.

Таким образом, методика выведения детектора из режима счета фотонов в линейный режим для осуществления атаки с навязыванием ключа (<<Faked-state attack>>) легко формализуется. Экспериментальное исследование уязвимости детектора одиночных фотонов к такому типа атак реализуется в три этапа, представленных на рисунке \ref{fig:Method_2.3}:
%
 \begin{figure}[ht] 
  \centering
  \includegraphics[scale=0.5]{Method_2.3.eps}
  \caption{Методика выведения детектора из режима Гейгера}
  \label{fig:Method_2.3}
\end{figure}
%
 \begin{enumerate}
	\item Определение величины постоянной оптической мощности, достаточной для выведения детектора из режима Гейгера
	\item Подстройка оптического импульса под окно срабатывания детектора одиночных фотонов
	\item Определение зависимости вероятности срабатывания детектора от величины энергии фотонов в импульсе
\end{enumerate}
%
В результате экспериментального исследования получен ряд зависимостей вероятности формирования отсчета от величины оптической мощности, применяемой для выведения детектора из режима счета фотонов, для различных оптических контролирующих импульсов (рис. \ref{fig:Probability_vs_Energy}). 

\begin{figure}[ht]
  \centering
  \includegraphics[scale=0.5]{Probability_vs_Energy.png}
  \caption{Зависимость вероятности срабатывания детектора от энергии контролирующего импульса}
  \label{fig:Probability_vs_Energy}
\end{figure}


Таким образом показано, что использование коммерческих детекторов одиночных фотонов на основе лавинных фотодиодов в режиме Гейгера модели id210 с частотой стробирования 100 МГц  требует применения дополнительных средств защиты от атаки с выведением из режима Гейгера при помощи коротких оптических импульсов с энергией не менее 15,4 фДж и при постоянном уровне оптической засветки средним уровнем мощности излучения не менее 35 нВт.  

 \underline{\textbf{Третья глава}} посвящена исследованию предложенной автором работы атаки с навязыванием ключа на системы ККБЧ. Оценены границы применимости данной атаки с учетом характеристик и параметром оборудования в составе блока получателя. Для определения попыток злоумышленника провести описанную выше атаку предлагается измерять интенсивность центральной оптической моды. Центральная мода отражается узкополосным фильтром на основе брэгговской решетки. Для того, чтобы её можно было измерить в схеме устанавливается волоконно-оптический циркулятор. Все излучение, которое после фазовых модуляторов в приемном блоке поступает на первый порт циркулятора направляется во второй порт на оптический фильтр. Отраженная центральная мода направляется в третий порт циркулятора. Для того, чтобы измерять центральную, которая, как показано в разделе, должна быть порядка 700~нВт, предлагается использовать мониторный фотодиод. Так как. злоумышленника есть возможность подбирать интенсивность на центральной моде, то очевидным является то, что мониторный фотодиод, установленный на выходе с 3-го порта, можно также контролировать оптическими методами. Для того, чтобы избежать этого, предлагается использовать волоконно-оптическое зеркало, например широко доступное зеркало Фарадея, и пассивный оптический аттенюатор. Сам мониторный фотодиод предлагается устанавливать в 4-ом порту циркулятора. Таким образом, излучение с центральной модой поступает на 3-ий порт, проходит аттенюатор в прямом направлении, отражается от волоконного зеркала, проходит аттенюатор в обратном направлении и поступает на 4-й порт, где установлен мониторный фотодиод.         
 \begin{figure}[ht]
  \centering
  \includegraphics[scale=0.25]{scw-setup_Countermeasure.pdf}
  \caption{Принципиальная оптическая схема предлагаемой контрмеры против атаки с <<поддельными>> состояниями}
  \label{fig:countermeasure}
\end{figure}
 
 На рисунке \ref{fig:Watchdog_photodiode} представлены две зависимости: нижний предел -- величина оптической мощности, которая требуется для выведения детектора фотонов из режима Гейгера и является минимальной необходимой для проведения успешной атаки злоумышленником; верхняя граница - величина оптической мощности, которая получена исходя из параметров приёмного модуля системы квантовой коммуникации и ограничения сверху на величину оптической мощности контролирующих импульсов. Минимальный уровень в соотвествии с расчетом составляет 0.7~мкВт, а максимальный - величину порядка 300~мкВт (в рамках данного исследования), однако на деле ограничен пороговой величиной ЛФД внутри ДОФ, при достижении которой ток становится слишком большим и диод сгорает, выводя из строя всё устройство. Типичной величиной являются единицы и десятки милливатт. 


С учётом данной зависимости можно оценить величину ослабления в фиксированном аттенюаторе, которой будет достаточно для защиты мониторного фотодиода от засветки, и при этом не потребуются дополнительные меры по расширению динамического диапазона чувствительности к входной оптической мощности в сторону увеличения, так как при оптических мощностях ниже единиц нановатт погрешность измерений становится достаточно высокой.   

Исходя из этого, величины аттенюации в 10~дБ достаточно, так как отраженная центральная на выходе с 3-го порта циркулятора прохоит через ослабляющий элемент в первый раз, а затем после отражения от зеркала во второй раз. Результирующее ослабление на 20~дБ, или в 100 раз, снизит минимальный уровень мощности на центральной частоте, которого достаточно для выведения детектора из режима счета фотонов, до величины порядка 7~нВт, что удовлетворяет требованиям, описанным выше. Максимальный уровень (в рамках данного исследования) снизится при этом примерно до 3~мкВт. 

 \begin{figure}[ht]
  \centering
  \includegraphics[scale=0.5]{images/Watchdog_photodiode.png}
  \caption{Динамика и границы оптической мощности на мониторном фотодиоде}
  \label{fig:Watchdog_photodiode}
\end{figure}


 В третьей главе показано, что измерение величины оптического излучения на несущей частоте, отраженного от оптического фильтра, при помощи мониторного фотодиода в приемном блоке системы квантовой коммуникации на боковых частотах в диапазоне от 7 нВт до 2,93 мкВт с применением дополнительных мер в виде пассивного оптического аттенюатора номиналом 10 дБ для его защиты позволяет противостоять атаке с выведением детектора одиночных фотонов из режима Гейгера и навязыванием ключа нелегитимным пользователем. 
 
 При этом ввиду особенности формирования квантовых состояний в системах на боковых частотах на приёмной стороне происходит спектральное разделение неинформативной несущей частоты и боковых частот, используемых для осуществления распределения ключа в качестве квантового сигнала. Боковые частоты попадают в спектр пропускания оптического фильтра, а несущая в спектр отражения, где осуществляется активный мониторинг её величины. Однако, полоса чувствительности детекторов фотонов на основе структур InGaAs/InP значительно шире (900-1700~нм). Таким образом, одним из вариантов контратаки злоумышленника с целью обойти предлагаемую контрмеру является изменение длины волны источника оптического излучения. Благодаря этому, в полосе отражения ОФ, и как следствие, на мониторном фотодиоде в приёмном модуле СКК сигнал будет отсутствовать, а <<контролирующие>> импульсы и оптические импульсы постоянного уровня оптической мощности для выведения ДОФ из режима счета фотонов позволят эффективно воздействовать на регистрирующее устройство блока получателя.

Решением является использование спектрально-селективного устройства на входе в блок получателя СКК, ограничивающего диапазон частот, поступающих в этот блок. Полоса пропускания должна быть достаточно узкой - не многим более величины 2 $\Omega$ (в нашем случае, 10~ГГц). В качестве такого устройства может быть использован волоконно-оптический фильтр на основе брэгговских решеток с термостабилизацией (такого же типа, как в ОФ в блоке получателя СКК). Однако, узкий диапазон частот характерен для работы в режиме отражения. Таким образом, в качестве спектрально-селективного устройства должна применяться связка волоконного оптического фильтра ОФ1 с волоконно-оптическим циркулятором, как показано на рис. \ref{fig:scw-setup_ref}. Вносимые потери фильтра и портов циркулятора составили 1.6 дБ. Преимуществом такого подхода является тот факт, что если злоумышленник будет отправлять импульсы непосредственно на частоте боковой, то в результате фазовой модуляции на ФМ2 в спектре появятся дополнительные компоненты, величина которых будет пропорциональная индексу модуляции (20:1), так что часть сигнала от злоумышленника, так или иначе будет попадать в полосу отражения ОФ2 и регистрироваться мониторным диодом). 

\begin{figure}
	\centering \includegraphics{scw-setup.pdf}
	\caption{Система квантовой коммуникации на боковых частотах модулированного излучения. Устройства, окрашенные красным (серым в ч/б печати) цветом предлагаются в качестве контрмеры. Л - источник лазерного излучения; ФМ - фазовый модулятор; АТТ - аттенюатор; Ц- циркулятор; ОФ - оптический фильтр; ДОФ - детектор одиночных фотонов; ЗФ - зеркало Фарадея; ФД - мониторный фотодиод. Вставки показывают спектр сигнала в различных точках оптической схемы.}
	\label{fig:scw-setup_ref}
\end{figure}


Таким образом, в 3 главе диссертации показано, что преобразование частоты контролирующего сигнала посредством фазовой модуляции на радиочастотной моде позволяет обнаружить атаку с поддельными состояниями и навязыванием срабатываний модулю получателя системы квантовой коммуникации.

% Можно сослаться на свои работы в автореферате. Для этого в файле
% \verb!Synopsis/setup.tex! необходимо присвоить положительное значение
% счётчику \verb!\setcounter{usefootcite}{1}!. В таком случае ссылки на
% работы других авторов будут подстрочными.
% \ifnumgreater{\value{usefootcite}}{0}{
% Изложенные в третьей главе результаты опубликованы в~\cite{vakbib1, vakbib2}.
% }{}
% Использование подстрочных ссылок внутри таблиц может вызывать проблемы.

 В \underline{\textbf{четвертой главе}} приведено описание когерентных состояний, которые нашли широкое применение в системах квантовой коммуникации, ввиду того, что в качестве источника излучения используются когерентные лазеры. 
 
 Отличительно особенностью систем квантовой коммуникации на боковых частотах модулированного излучения является генерация многомодовых когерентных состояний на разных оптических модах, зависящих от частоты модулирующего сигнала, как показано на рисунке \ref{fig:multimodes}. 

 \begin{figure}[ht]
  \centering
  \includegraphics[scale=0.4]{Modes_rus.pdf}
  \caption{Принципиальная схема генерации боковых частот}
  \label{fig:multimodes}
\end{figure}


Приготовленные состояния определяются следующим образом: входное (немодулированное) состояние на стороне модулятора отправителя (Алиса) и получателя (Боб) (далее именуемые, как $A$ or $B$) определяется, как $|\sqrt{\mu_0}\rangle_0\otimes|\mathrm{vac}\rangle_{SB}$, где $|\mathrm{vac}\rangle_{SB}$ это вакуумное состояние на боковых и $|\sqrt{\mu_0}\rangle_0$ это когерентное состояние несущей частоты с амплитудой, определенной средним числом фотонов $\mu_0$ в окне пропускания. формируемая когерентным монохроматическим излучением с оптической частотой $\omega$. Фаза несущей волны принимается как опорная и все остальные фазы считаются по отношению к ней. Электро-оптический фазовый модулятор (с частотой колебаний микроволнового поля $\Omega$ и её фазой $\varphi_A$ или $\varphi_B$) перераспределяет энергию между взаимодействующими модами (поле на выходе модулятора приобретает боковые частоты $\omega_k=\omega+k\Omega$, ограничим рассматриваемый нами случай $2S$ боковыми частотами и пусть целое число $k$ мод ограничено пределами $-S\le k\le S$), так, что состояние поля на выходе модулятора - это многомодовое когерентное состояние: 
%
\begin{equation}\label{phi}
|\psi_0(\varphi_j)\rangle = \bigotimes_{k=-S}^S|{\alpha_k(\varphi_j)}\rangle_k,
\end{equation}
%
где $j$ это и $A$, и $B$ (определяющее Алису или Боба), а амплитуды имеют следующий вид: 
%
\begin{equation}\label{alpha}
\alpha_k(\varphi_j)=\sqrt{\mu_0}d^S_{0k}(\beta)e^{i\varphi_jk},
\end{equation}
%
и $d^S_{nk}(\beta)$ это d-функция Вигнера, взятая из квантовой теории углового момента %\cite{varshalovich1988quantum}
, $\beta$ определяется индексом модуляции $m$, который без учета дисперсии в среде модулятора можно выразить: 
%
\begin{equation}\label{betam}
\cos{({\beta})}=1-\frac{1}{2}{\left(\frac{m}{S+0.5}\right)^2},
\end{equation}
где $S$ количество взаимодействующий мод, принимаемое очень большим.

После подстройки относительной фазы оптических сигналов в двух плечах, наблюдается интерференция этих состояний на втором светоделителе ($BS2$). Описание состояний на выходах светоделителя дано в уравнении~\ref{states}. Далее предполагается, что относительная фазовая отстройка оптических импульсов равна $\phi_0\approx0$. Если разность фаз между радиочастотными модулирующими сигналами Алисы и Боба равна нулю ($\Delta\varphi=0$) весь спектр идет в одно выходное плечо светоделителя $BS2$, в ином случае, четные моды спектра, включая центральную, идут в то же плечо, а нечетные идут во второе плечо. В случае $\Delta\varphi=0$, требуется спектральное разделение центральной частоты и боковых в приёмном блоке, так как кодирование квантовых состояний происходит на боковых частотах. Следует учитывать, что при использовании малого среднего числа фотонов в импульсе, значительный вклад в многомодовых состояниях вносят только первая пара боковых частот. Случай, при котором $\Delta\varphi=\pi$, является нетривиальным. Многомодовое состояние разделяется на $BS2$ и центральная мода (и все четные) идут в первое плечо, а все нечетные (первые боковые частоты вносят наибольший вклад в результирующий сигнал) идут во второе плечо, как показано на рисунке \ref{fig:Interference_result}. Таким образом, можно отказаться от спектрального разделения при помощи оптического фильтра в одном из выходных плечей приёмного узла. Показателем хорошей подстройки по фазе оптических импульсов является постоянный высокий уровень на центральной частоте в первом плече (на $D2$).  


\begin{figure}[ht]
 \centering
  \includegraphics[scale=0.5]{Interference_result.png}
  \caption{Принципиальная схема наблюдения результата интерференции когерентных состояний}
  \label{fig:Interference_result}
\end{figure}

 
В четвертой главе показано, что метод квантовой коммуникации на боковых частотах позволяет реализовывать протокол, устойчивый к контролю нелегитимным пользователем измерительного оборудования. 

Основываясь на полученных результатах можно сформулировать протокол для системы квантовой коммуникации, использующей многомодовые когерентные состояния, с недоверенным измерительным узлом (рис. \ref{fig:Protocol}). Для простоты приведем пример с использованием только двух фазовых состояний по аналогии с протоколом B92. Алиса случайным образом выбирает одно фазовое состояние из двух возможных <<$0$>> или <<$\pi$>>. Боб независимо от Алисы тоже случайным образом выбирает одно из двух фазовых состояний. В результате интерференции многомодовых когерентных состояний можно наблюдать 4 различных варианта, в зависимости от разности фаз, выбранных Алисой и Бобом.  При этом на недоверенном узле регистрации будет происходить спектральное разделение, так что на $D1$ и $D2$ будут поступать боковые частоты. Злоумышленник при этом имеет полный доступ к недоверенному измерительному узлу и может фиксировать себе оглашенный результат срабатываний одного из детекторов. Положим, что нас интересует только случай, при котором разность фаз между $A$ и $B$ равна $\pi$. Преимущество этого случая в том, что за счет спектрального разделения в результате интерференции в плечо с детектором $D1$ всегда будут идти только боковые и никогда не будет идти центральная мода, а значит можно исключить из оптической схемы этого плеча оптический фильтр, согласование которого с фильтром в другом плече -- достаточно трудная инженерная задача. Таким образом, при просеивании остаются только те биты, которые соответствуют отсчетам на $D1$, при этом для корреляции между битами Алисы и битами Боба последний должен сделать смену бита на противоположный (так называемый <<bitflip>>).  

\begin{figure}[ht]
 \centering
  \includegraphics[scale=0.5]{Protocol.png}
  \caption{Протокол}
  \label{fig:Protocol}
\end{figure}
 
  В \underline{\textbf{пятой главе}} показано, что в результате интерференции квантового фазомодулированного сигнала на боковых частотах на симметричном светоделителе в схеме квантовой рассылки ключа с узлом регистрации, независящим от легитимного пользователя, происходит спектральное разделение квантового сигнала и сигнала на центральной длине волны с их независимой регистрацией в разных плечах светоделителя. 
  
  Для проверки концепции была собрана оптическая схема, представленная на рисунке \ref{fig:RF_sin}. Для простоты использовался только один источник излучения, сигнал на выходе которого делился светоделителем 50:50 $BS1$. Такой подход позволяет имитировать случай, когда у Алисы и Боба хорошо скоррелированные источники излучения. В данном эксперименте применялся лазер с распределенной обратной связью Л1 (TTX1994 Neophotonics), отличительными особенностями которого являются очень узкая ширина спектральной линии порядка 100~кГц и широкий диапазон перестройки длин волн (от 1530~нм до 1565,5~нм) с высокой степенью точности (20~пм). Выходная мощность при этом составляла 10~мВт. Однако, это значение зависит от глубины динамического диапазона перестраиваемого оптического аттенюатора ПОА и суммарных потерь, вносимых элементами Алисы или Боба, так как результирующая мощность сигнала на боковых частотах должна не превышать величину 2,56~пВт, соответствующую среднему числу фотонов в импульсе 0,2 при частоте смены состояний 100~Мгц.   

Из-за чувствительности резонатора внутри лазера с распределенной обратной связью к обратному относительно выходного оптическому излучению, то есть переотражениям и рассеиванию сигналов, в оптической схеме применяется волоконный изолятор И на основе эффекта Фарадея с величиной изоляции порядка 50~дБ. Для перевода системы в однофотонный режим применяется ПОА с достаточной малым шагом перестройки 0,05 дБ и большим динамическим диапазоном величиной 60~дБ.  

  

  \begin{figure}[ht]
  \centering
  \includegraphics[scale=0.4]{Scheme_colored_rus.png}
  \caption{Принципиальная схема экспериментального стенда}
  \label{fig:RF_sin}
\end{figure}

  
 
  
На выходе из источника сигнал проходил по двум плечам формируемого таким образом интерферометра через светоделитель, благодаря чему имитировалось, что Алиса и Боб - это раздельные узлы, в которых информация кодировалась с помощью фазовых модуляторов. Выходы первого светоделителя подключались ко входам двух независимых 10~ГГц электро-оптических модуляторов на основе $LiNbO_3$ (со встроенным поляризатором) $PM1$ and $PM2$. Поляризаторы снижали чувствительность модуляторов к поляризации входного излучения, для чего использовался контроллер поляризации КП, и таким образом, повышалась видность картины интерференции. Электрические входы $PM1$ и $PM2$ подключались к выходам с цифро-аналоговых преобразователей ЦАП с модулирующим радиочастотным синусоидальным сигналом (с частотой 4,8~ГГц). Амплитуды управляющих сигналов подбирались таким образом, чтобы мощность сигнала на боковых частотах была равно у Алисы и у Боба

Индекс модуляции, который показывает долю энергии на боковых частотах по отношению к центральной в результате модуляции должен быть 5~\%. В таком случае наблюдается оптимальное соотношение сигнала на боковых частотах к шуму проходящей через оптический фильтр центральной частоты.  Так что средняя величина оптической мощности была равна 2.56~пВт, что соответствует $\mu=0.2$.  


В данном исследовании  использовались только два кодирующих фазовых состояния на Алисе и Бобе ($\varphi_{A,B}\in\{0,\pi\}$. Разница фаз ($\Delta\varphi$) между Алисой и Бобом формировалась с помощью изменения IQ-таблиц ЦАП при помощь ПЛИС и ПО сосбтвенной разработки. В случае измерения зависимости количества срабатываний детекторов от разности фаз ($\Delta\varphi$) фаза сдвигалась последовательно с шагом $\varphi_{step}\ = 10^{\circ}$. Наблюдаемые состояния на выходах $PM1$ и $PM2$ могут быть описаны в соотвествии с уравнением~\ref{phi}.

После того, как состояния приготовлены, мы посылаем их в квантовый канал. Для компенсации оптической разности хода в двух плечах интерферометра, а так же для точной подстройки оптических фаз сигналов использовались $PM3$ и $PM4$. Эта подстройка осуществлялась при помощи изменения постоянного напряжения, подаваемого на модуляторы и формируемого двумя независимыми электрическими выходами генератора сигналов $GEN1$ и $GEN2$. Выходы втрой пары фазовых модуляторов подключались к паре входных портов светоделителя $BS2$ 2x2 с коэффициентом деления 50:50.

Измерения проводились в два этапа: первый этап в классическом свете, второй этап - в режиме счета фотонов на сверхпроводниковом детекторе одиночных фотонов (Сконтел), у которого встроены два независимых приёмника $D1$, $D4$. Квантовая эффективность обоих была равна 10~\%, а уровень темновых шумов не превышал 50~Гц для каждого. Измеренная величина суммарных шумов, включающих в себя темновые срабатывания и засветку от центральной частоты в силу ограниченной экстинкции оптического фильтра, составляла $\gamma_{темн}=1.5$ кГц.

Показаны зависимости интенсивности на боковых частотах в результате интерференции от разности фаз модулирующих сигналов в классическом режиме (рис. \ref{fig:Experimental_TF_classical_ref}) и в режиме счета фотонов (рис. \ref{fig:det_rate_TF_RUS_ref}). Оценены основные параметры, характеризующие систему квантовой коммуникации: квантовый коэффициент ошибок по битам (QBER) и среднее значение скорости формирования просеянного ключа $K$ (одинаковые биты между легитимными пользователями, но коррелирующие с возможным результатом у злоумышленника, что требует проведения процедуры усиления секретности) 


 \begin{figure}[ht]
  \centering
  \includegraphics[scale=0.7]{Experimental_TF_classical.png}
  \caption{Зависимость интенсивности на боковых частотах в результате интерференции от разности фаз модулирующих сигналов}
  \label{fig:Experimental_TF_classical_ref}
\end{figure}


 \begin{figure}[ht]
  \centering
  \includegraphics[scale=0.2]{det_rate_rus.png}
  \caption{Зависимость количества отсчетов в результате интерференции от разности фаз модулирующих сигналов}
  \label{fig:det_rate_TF_RUS_ref}
\end{figure}


 В \underline{\textbf{шестой главе}} показано, что для когерентных многомодовых квантовых состояний атака злоумышленника с разделением многофотонных состояний не приводит к раскрытию ключевой информации, так как при измерении числа фотонов в импульсе (проекции на Фоковский базис), сокращенное состояние не содержит информацию о фазе когерентного состояния. В главе приведена оценка скорости формирования просеянных ключей для системы с недоверенным узлом, в которой не учитывается доля информации, получаемая злоумышленником из квантового канала и после исправления ошибок.
 
 
  Также приведена оценка асимптотической скорости формирования ключа с учетом возможности получения ключевой информации из двух независимых каналов в системе с недоверенным приёмным узлом. 
 
 Оценка границы Холево для многомодовых когерентных состояний дана в \ref{phi} в соответствии с \cite{kozubov2019finite}:
\begin{equation}
    \chi=h\left(\frac{1-e^{-\mu_0(1-d^S_{00}(2\beta))}}{2}\right)\approx h\left(\frac{1-e^{-2\mu}}{2}\right).
\end{equation}

Она показывает максимальное количество информации, которую Ева может получить из состояний в одном канале. Однако, в схеме с недоверенным узлом используются два независимых канала. Следовательно, злоумышленник может осуществлять независимые измерения в двух квантовых каналах и получать следующее количество информации:
\begin{equation}
    \tilde{\chi}=2(1-\chi)\chi+\chi^2.
\end{equation}

 Проведен расчет параметров системы с характеристиками из публикаций о системе квантовой коммуникации. В соотвествии с расчетом показано на рис. \ref{fig:fig2}, что одним из преимуществ протокола с недоверенным узлом является превышение величины скорости формирования ключа над линейной границей в определенном диапазоне дистанций.


Асимптотическая скорость генерации кодирующей битовой последовательности $K$ будет следующим:
\begin{equation}
     K=FR(1-\xi h(Q)-\tilde{\chi}),
\end{equation}
где $\xi$ - эффективность процедуры исправления ошибок. 




\begin{figure}
	\includegraphics[width=1\linewidth]{TFQKD.png}
	\caption{Расчет параметров системы квантовой коммуникации на боковых частотах модулированного излучения с недоверенным узлом.}
	\label{fig:fig2}
\end{figure}


Для предполагаемых значений скорость формирования ключа превышает линейный предел скорости \cite{pirandola2017fundamental} где $\eta_c \gtrsim$ 40 дБ (200 км). Вдобавок, предлагаемый в работе протокол также способен достичь длины линии передачи $\eta_c\approx$ 83 дБ (460 км).


 \underline{\textbf{Заключение}} содержит список основных результатов, полученных в работе. 

% !_____!
  %% Согласно ГОСТ Р 7.0.11-2011:
%% 5.3.3 В заключении диссертации излагают итоги выполненного исследования, рекомендации, перспективы дальнейшей разработки темы.
%% 9.2.3 В заключении автореферата диссертации излагают итоги данного исследования, рекомендации и перспективы дальнейшей разработки темы.
\begin{enumerate}
  \item На основе экспериментального анализа детектора, работающего в режиме Гейгера, показано, что требуются дополнительные средства защиты от атаки с выведением из режима Гейгера при помощи коротких оптических импульсов с энергией не менее 15,4 фДж и при постоянном уровне оптической засветки средним уровнем мощности излучения не менее 35 нВт. 
  \item Численные исследования показали, что измерение величины оптического излучения на несущей частоте, отраженного от оптического фильтра, при помощи мониторного фотодиода в приемном блоке системы квантовой коммуникации на боковых частотах в диапазоне от 7 нВт до 2,93 мкВт с применением дополнительных мер в виде пассивного оптического аттенюатора номиналом 10 дБ для его защиты позволяет противостоять атаке с выведением детектора одиночных фотонов из режима Гейгера и навязыванием ключа нелегитимным пользователем. 
  \item Метод квантовой коммуникации на боковых частотах позволяет реализовывать протокол, устойчивый к контролю нелегитимным пользователем измерительного оборудования.
  \item Для выполнения поставленных задач был создан экспериментальный стенд и в результате интерференции квантового фазомодулированного сигнала на боковых частотах на симметричном светоделителе в схеме квантовой рассылки ключа с узлом регистрации, независящим от легитимного пользователя, наблюдается спектральное разделение квантового сигнала и сигнала на центральной длине волны с их независимой регистрацией в разных плечах светоделителя. 
\end{enumerate}

\section*{Работы автора по теме диссертации}
{Статьи в журналах, рецензируемые в Web of Science или Scopus: }
\begin{enumerate}\addtolength{\itemsep}{-0.5\baselineskip}
\renewcommand{\labelenumi}{[\theenumi]}
\item Vladimir Chistiakov, Anqi Huang, Vladimir Egorov, and Vadim Makarov, Controlling single-photon detector ID210 with bright light, Opt. Express 27, 32253-32262 (2019)
\\
\item Чистяков В.В., Гайдаш А.А., Козубов А.В., Глейм А.В. Исследование интерференции слабых когерентных многомодовых состояний для задач квантовой коммуникации с недоверенным приемным узлом // Научно-технический вестник информационных технологий, механики и оптики. 2019. Т. 19. № 6. doi: 10.17586/2226-1494-2019-19-6
\\
\item    Gleim A.V., Egorov V.I., Nazarov Y.V., Smirnov S.V., Chistyakov V.V., Bannik O.I., Anisimov A.A., Kynev S.M., Ivanova A.E., Collins R.J., Kozlov S.A., Buller G. Secure polarization-independent subcarrier quantum key distribution in optical fiber channel using BB84 protocol with a strong reference//Optics express, IET - 2016, Vol. 24, No. 3, pp. 2619-2633
\\
%\item  Глейм А.В., Егоров В.И., Чистяков В.В., Смирнов С.В., Банник О.И., Булдаков Н.В., Гайдаш А.А., Козубов А.В., Васильев А.Б., Кынев С.М., Хоружников С.Э., Козлов С.А., Васильев В.Н. Квантовая коммуникация на боковых частотах со скоростью 1 Мбит/с в городской сети // Оптический журнал -2017. - Т. 84. - № 6. - С. 3-9
\\
\item  Chistyakov V.V., Kynev S.M, Smirnov S.V., Nazarov Y.V., Gleim A.V. Achieving high visibility in subcarrier wave quantum key distribution system // Journal of Physics: Conference Series, IET - 2016, Vol. 735, No. 1, pp. 012085
\\
\item V. V. Chistyakov, A. V. Gleim, V. I. Egorov, Yu. V. Nazarov. Implementation of multiplexing in a subcarrier-wave quantum cryptography system // Journal of Physics: Conference Series - 2014  vol. 541,  pp. 012078
\\
\item   Kynev S.M., Chistyakov V.V., Smirnov S.V., Volkova K.P., Egorov V.I., Gleim A.V. Free-space subcarrier wave quantum communication // Journal of Physics: Conference Series - 2017, Vol. 917, No. 5, pp. 052003
\\

\item    Gleim A.V., Nazarov Y.V., Egorov V.I., Smirnov S.V., Bannik O.I., Chistyakov V.V., Kynev S.M., Anisimov A.A., Kozlov S.A., Vasil'ev V.N. Subcarrier Wave Quantum Key Distribution in Telecommunication Network with Bitrate 800 kbit/s//EPJ Web of Conferences, IET - 2015, Vol. 103, pp. 10005
\\
\item    Gleim A.V., Egorov V.I., Nazarov Y.V., Smirnov S.V., Chistyakov V.V., Bannik O.I., Anisimov A.A., Kynev S.M., Collins R.J., Kozlov S.A., Buller G.S. Polarization insensitive 100 MHz clock subcarrier quantum key distribution over a 45 dB loss optical fiber channel // Conference on Lasers and Electro-Optics, CLEO 2015, IET - 2015, pp. 7182997
\\
\item Gaidash A.A., Kozubov A.V., Chistyakov V.V., Miroshnichenko G.P., Egorov V.I., Gleim A.V. Security conditions for sub-carrier wave quantum key distribution protocol in errorless channel // Journal of Physics: Conference Series - 2017, Vol. 917, No. 6, pp. 062014
\\
\item  Gleim A.V., Chistyakov V.V., Bannik O.I., Egorov V.I., Buldakov N.V., Vasilev A.B., Gaidash A.A., Kozubov A.V., Smirnov S.V., Kynev S.M., Khoruzhnikov S.E., Kozlov S.A., Vasil'ev V.N. Sideband quantum communication at 1 Mbit/s on a metropolitan area network // Journal of Optical Technology - 2017, Vol. 84, No. 6, pp. 362-36
\\
\item Latypov I.Z., Акатьев Д.О., Fadeev M.A., Chistyakov V.V., Khalturinskii A.K., Kynev S.M., Egorov V.I., Gleim A.V. Atmosphere channel for “last mile problem” in quantum communication // EPJ Web of Conferences, 2019, Vol. 220, pp. 01006

\end{enumerate}

\noindent{ Другие публикации: }
\begin{enumerate}\addtolength{\itemsep}{-0.5\baselineskip}
\renewcommand{\labelenumi}{[\theenumi]}
\setcounter{enumi}{9}
\item   Чистяков В.В., Кынев С.М., Смирнов С.В., Назаров Ю.В., Глейм А.В. Обеспечение высокой видности в системе квантовой криптографии на боковых частотах // Сборник трудов IX международной конференции молодых ученых и специалистов «Оптика – 2015», с. 658-660
\\
\item Глейм А.В., Назаров Ю.В., Егоров В.И., Чистяков В.В, Смирнов С.В., Банник О.И., Кынев С.М., Иванова А.Е., Дубровская В.Д., Тарасов М.Г., Булдаков Н.В., Кузьмина Т.Б., Чивилихин С.А., Анисимов А.А., Рощупкин С.В., Рогачёв К.С., Хоружников С.Э., Козлов С.А., Васильев В.Н. Создание квантовой сети университета ИТМО //Сборник трудов VIII международной конференции «Фундаментальные проблемы оптики – 2014». Санкт-Петербург, 20-24 октября ,2014, С.3-4,  541 с. 
\\
\item А.В. Глейм, В.И.Егоров, А.А. Анисимов, Ю.В. Назаров, С.М. Кынев, А.В. Рупасов, В.В. Чистяков, А.А.Гайдаш, М.А. Смирнов, С.А. Чивилихин, С.А. Козлов Квантовая рассылка криптографического ключа по оптическому волокну телекоммуникационного стандарта на расстояние 200 км со скоростью 0.18 кбит/с // Cборник трудов III Всероссийская конференция по фотонике и информационной оптике Москва, НИЯУ МИФИ, 2014 с. 17-19
\\


\end{enumerate}


%	\newcommand{\publications}{\underline{\textbf{\publicationsTXT}}}

%При использовании пакета \verb!biblatex! список публикаций автора по теме
%диссертации формируется в разделе <<\publications>>\ файла
%\verb!../common/characteristic.tex!  при помощи команды \verb!\nocite!

\ifdefmacro{\microtypesetup}{\microtypesetup{protrusion=false}}{} % не рекомендуется применять пакет микротипографики к автоматически генерируемому списку литературы
\ifnumequal{\value{bibliosel}}{0}{% Встроенная реализация с загрузкой файла через движок bibtex8
  \renewcommand{\bibname}{\large \authorbibtitle}
  \nocite{*}
  \insertbiblioauthor           % Подключаем Bib-базы
  %\insertbiblioother   % !!! bibtex не умеет работать с несколькими библиографиями !!!
}{% Реализация пакетом biblatex через движок biber
  \ifnumgreater{\value{usefootcite}}{0}{
%  \nocite{*} % Невидимая цитата всех работ, позволит вывести все работы автора
  \insertbiblioauthorcited      % Вывод процитированных в автореферате работ автора
  }{
  \insertbiblioauthor           % Вывод всех работ автора
%  \insertbiblioauthorgrouped    % Вывод всех работ автора, сгруппированных по источникам
%  \insertbiblioauthorimportant  % Вывод наиболее значимых работ автора (определяется в файле characteristic во второй section)
  \insertbiblioother            % Вывод списка литературы, на которую ссылались в тексте автореферата
  }
}
\ifdefmacro{\microtypesetup}{\microtypesetup{protrusion=true}}{}
