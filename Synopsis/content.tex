
\section*{Общая характеристика работы}

\newcommand{\actuality}{\underline{\textbf{\actualityTXT}}}
\newcommand{\progress}{\underline{\textbf{\progressTXT}}}
\newcommand{\aim}{\underline{{\textbf\aimTXT}}}
\newcommand{\tasks}{\underline{\textbf{\tasksTXT}}}
\newcommand{\novelty}{\underline{\textbf{\noveltyTXT}}}
\newcommand{\influence}{\underline{\textbf{\influenceTXT}}}
\newcommand{\methods}{\underline{\textbf{\methodsTXT}}}
\newcommand{\defpositions}{\underline{\textbf{\defpositionsTXT}}}
\newcommand{\reliability}{\underline{\textbf{\reliabilityTXT}}}
\newcommand{\probation}{\underline{\textbf{\probationTXT}}}
\newcommand{\contribution}{\underline{\textbf{\contributionTXT}}}
\newcommand{\publications}{\underline{\textbf{\publicationsTXT}}}


{\actuality} \todo{поменять с 1 режим черновика на 0 в setup.tex для соблюдения ГОСТ}

Обзор, введение в тему, обозначение места данной работы в
мировых исследованиях и~т.\:п., можно использовать ссылки на~другие
работы\ifnumequal{\value{bibliosel}}{1}{~\autocite{Gosele1999161}}{}
(если их~нет, то~в~автореферате
автоматически пропадёт раздел <<Список литературы>>). Внимание! Ссылки
на~другие работы в разделе общей характеристики работы можно
использовать только при использовании \verb!biblatex! (из-за технических
ограничений \verb!bibtex8!. Это связано с тем, что одна
и~та~же~характеристика используются и~в~тексте диссертации, и в
автореферате. В~последнем, согласно ГОСТ, должен присутствовать список
работ автора по~теме диссертации, а~\verb!bibtex8! не~умеет выводить в одном
файле два списка литературы).
При использовании \verb!biblatex! возможно использование исключительно
в~автореферате подстрочных ссылок
для других работ командой \verb!\autocite!, а~также цитирование
собственных работ командой \verb!\cite!. Для этого в~файле
\verb!Synopsis/setup.tex! необходимо присвоить положительное значение
счётчику \verb!\setcounter{usefootcite}{1}!.

Для генерации содержимого титульного листа автореферата, диссертации
и~презентации используются данные из файла \verb!common/data.tex!. Если,
например, вы меняете название диссертации, то оно автоматически
появится в~итоговых файлах после очередного запуска \LaTeX. Согласно
ГОСТ 7.0.11-2011 <<5.1.1 Титульный лист является первой страницей
диссертации, служит источником информации, необходимой для обработки и
поиска документа>>. Наличие логотипа организации на~титульном листе
упрощает обработку и~поиск, для этого разметите логотип вашей
организации в папке images в~формате PDF (лучше найти его в векторном
варианте, чтобы он хорошо смотрелся при печати) под именем
\verb!logo.pdf!. Настроить размер изображения с логотипом можно
в~соответствующих местах файлов \verb!title.tex!  отдельно для
диссертации и автореферата. Если вам логотип не~нужен, то просто
удалите файл с~логотипом.

% \ifsynopsis
% Этот абзац появляется только в~автореферате.
% Для формирования блоков, которые будут обрабатываться только в~автореферате,
% заведена проверка условия \verb!\!\verb!ifsynopsis!.
% Значение условия задаётся в~основном файле документа (\verb!synopsis.tex! для
% автореферата).
% \else
% Этот абзац появляется только в~диссертации.
% Через проверку условия \verb!\!\verb!ifsynopsis!, задаваемого в~основном файле
% документа (\verb!dissertation.tex! для диссертации), можно сделать новую
% команду, обеспечивающую появление цитаты в~диссертации, но~не~в~автореферате.
% \fi

% {\progress}
% Этот раздел должен быть отдельным структурным элементом по
% ГОСТ, но он, как правило, включается в описание актуальности
% темы. Нужен он отдельным структурынм элемементом или нет ---
% смотрите другие диссертации вашего совета, скорее всего не нужен.

{\aim} данной работы является оценка возможностей злоумышленника по получению секретного ключа с использованием атак на измерительное оборудование систем квантовой коммуникации на боковых частотах и разработка методов противодействия на основе результатов оценки.


Для~достижения поставленной цели необходимо было решить следующие {\tasks}:
\begin{enumerate}
  \item Исследование устойчивости детектора одиночных фотонов, применяемого в системах квантовой коммуникации на боковых частотах, к атакам с выведением из режима Гейгера (<<ослеплением>>). 

  \item Оценка возможностей злоумышленника при атаке с выведением из режима Гейгера для систем квантовой коммуникации на боковых частотах. 

  \item Разработка оптической схемы системы квантовой коммуникации, устойчивой к атакам на измерительное оборудование. 

  \item Разработка протокола квантовой рассылки ключа, устойчивого к атаке на 						измерительное оборудование. 

\end{enumerate}


{\novelty}
\begin{enumerate}
  \item Впервые исследована устойчивость системы квантовой коммуникации на боковых частотах к атакам злоумышленника на измерительное оборудование приёмного блока. 
  \item Впервые предложена и применена контрмера против атаки злоумышленника с выведением детектора из режима Гейгера и навязыванием легитимным пользователями ключа. 
  \item Было выполнено оригинальное исследование и разработана оптическая схема с вынесением устройства детектирования одиночных фотонов из приемного блока в <<недоверенный>> узел, подконтрольный злоумышленнику 
\end{enumerate}

{\influence} \ldots

{\methods} \ldots

{\defpositions}
\begin{enumerate}
  \item Использование коммерческих детекторов одиночных фотонов на основе лавинных фотодиодов в режиме Гейгера модели id210 с частотой стробирования 100 МГц  требует применения дополнительных средств защиты от атаки с выведением из режима Гейгера.   
  \item \VCc{ КОНТРМЕРА }
  \item Метод квантовой коммуникации на боковых частотах позволяет реализовывать протокол, устойчивый к контролю нелегитимным пользователем измерительного оборудования. 
  \item В результате интерференции квантового фазомодулированного сигнала на боковых частотах на симметричном светоделителе в схеме квантовой рассылки ключа с узлом регистрации, независящим от легитимного пользователя, происходит спектральное разделение квантового сигнала и сигнала на центральной длине волны с их независимой регистрацией в разных плечах светоделителя. 
  \item Показана возможность двукратного увеличения дальности квантовой рассылки ключа на боковых частотах посредством применения недоверенной системы регистрации квантовых состояний. 
\end{enumerate}
% В папке Documents можно ознакомиться в решением совета из Томского ГУ
% в~файле \verb+Def_positions.pdf+, где обоснованно даются рекомендации
% по~формулировкам защищаемых положений.

{\reliability} полученных результатов обеспечивается применением утверждённых методик проведений экспериментальных исследований и аттестованного оборудование. Математическое моделирование и обработка данных, полученных в результате экспериментов, осуществлялось с использованием пакетов прикладных программ MathCad и Excel. Результаты находятся в соответствии с результатами, полученными другими авторами.


{\probation}
Основные результаты работы докладывались~на:
перечисление основных конференций, симпозиумов и~т.\:п.

{\contribution} Автор принимал активное участие \ldots

%\publications\ Основные результаты по теме диссертации изложены в ХХ печатных изданиях~\cite{Sokolov,Gaidaenko,Lermontov,Management},
%Х из которых изданы в журналах, рекомендованных ВАК~\cite{Sokolov,Gaidaenko},
%ХХ --- в тезисах докладов~\cite{Lermontov,Management}.

\ifnumequal{\value{bibliosel}}{0}{% Встроенная реализация с загрузкой файла через движок bibtex8
    \publications\ Основные результаты по теме диссертации изложены в XX печатных изданиях,
    X из которых изданы в журналах, рекомендованных ВАК,
    X "--- в тезисах докладов.%
}{% Реализация пакетом biblatex через движок biber
%Сделана отдельная секция, чтобы не отображались в списке цитированных материалов
    \begin{refsection}[vak,wos,scopus,papers,conf]% Подсчет и нумерация авторских работ. Засчитываются только те, которые были прописаны внутри \nocite{}.
        %Чтобы сменить порядок разделов в сгрупированном списке литературы необходимо перетасовать следующие три строчки, а также команды в разделе \newcommand*{\insertbiblioauthorgrouped} в файле biblio/biblatex.tex
        \printbibliography[heading=countauthorvak, env=countauthorvak, keyword=biblioauthorvak, section=1]%
        \printbibliography[heading=countauthorwos,env=countauthorwos, keyword=biblioauthorwos, section=1]%
        \printbibliography[heading=countauthorscopus,env=countauthorscopus, keyword=biblioauthorscopus, section=1]%
	\printbibliography[heading=countauthorconf, env=countauthorconf, keyword=biblioauthorconf, section=1]%
        \printbibliography[heading=countauthorothers, env=countauthorothers, keyword=biblioauthorothers, section=1]%
        \printbibliography[heading=countauthor, env=countauthor, keyword=biblioauthor, section=1]%
        \nocite{%Порядок перечисления в этом блоке определяет порядок вывода в списке публикаций автора
                vakbib1,vakbib2,%
		wosbib1,%
		scbib1,%
                confbib1,confbib2,%
                bib1,bib2,%
        }%
	\publications\ Основные результаты по теме диссертации изложены
	\setcounter{citeauthorscwostot}{\value{citeauthorscopus}} % вместе setcounter и addtocounter добавляют пробел между словами. По-этому они так раскиданы.
        в~\arabic{citeauthor}~печатных изданиях,
	\addtocounter{citeauthorscwostot}{\value{citeauthorwos}}
	\arabic{citeauthorvak} из которых изданы в журналах, рекомендованных ВАК\sloppy
	\ifnum \value{citeauthorscwostot}>0
	, \arabic{citeauthorscwostot} "--- в~периодических научных журналах, индексируемых Web of Science и Scopus\sloppy
	\fi
	\ifnum \value{citeauthorconf}>0
	, \arabic{citeauthorconf} "--- в~тезисах докладов.
	\else
	.
	\fi
    \end{refsection}
    \begin{refsection}[vak,wos,scopus,papers,conf]%Блок, позволяющий отобрать из всех работ автора наиболее значимые, и только их вывести в автореферате, но считать в блоке выше общее число работ
        \printbibliography[heading=countauthorvak, env=countauthorvak, keyword=biblioauthorvak, section=2]%
        \printbibliography[heading=countauthorwos, env=countauthorwos, keyword=biblioauthorwos, section=2]%
        \printbibliography[heading=countauthorscopus, env=countauthorscopus, keyword=biblioauthorscopus, section=2]%
        \printbibliography[heading=countauthorothers, env=countauthorothers, keyword=biblioauthorothers, section=2]%
        \printbibliography[heading=countauthorconf, env=countauthorconf, keyword=biblioauthorconf, section=2]%
        \printbibliography[heading=countauthor, env=countauthor, keyword=biblioauthor, section=2]%
        \nocite{vakbib2}%vak
        \nocite{bib1}%other
        \nocite{confbib1}%conf
    \end{refsection}
}
При использовании пакета \verb!biblatex! для автоматического подсчёта
количества публикаций автора по теме диссертации, необходимо
их~здесь перечислить с использованием команды \verb!\nocite!.
 % Характеристика работы по структуре во введении и в автореферате не отличается (ГОСТ Р 7.0.11, пункты 5.3.1 и 9.2.1), потому её загружаем из одного и того же внешнего файла, предварительно задав форму выделения некоторым параметрам

%Диссертационная работа была выполнена при поддержке грантов ...

%\underline{\textbf{Объем и структура работы.}} Диссертация состоит из~введения,
%четырех глав, заключения и~приложения. Полный объем диссертации
%\textbf{ХХХ}~страниц текста с~\textbf{ХХ}~рисунками и~5~таблицами. Список
%литературы содержит \textbf{ХХX}~наименование.

 \section*{Содержание работы}
 Во \underline{\textbf{введении}} обосновывается актуальность
 исследований, проводимых в~рамках данной диссертационной работы,
  формулируется цель, ставятся задачи работы, излагается научная новизна
 и практическая значимость представляемой работы. 

 В \underline{\textbf{первой главе}}, имеющей обзорный характер, рассмотрены основные известные на данный момент способы регистрации одиночных фотонов для применения в системах квантовой коммуникации. Проведены анализ и сравнение сравнение различных типов детекторов фотонов, обоснован выбор исследуемого в работе детектора одиночных фотонов (ДОФ) на основе лавинного пробоя. Показаны возможные атаки потенциального злоумышленника на измерительное оборудование, а также приведены известные контрмеры против такого типа атак.   

%  картинку можно добавить так:
% \begin{figure}[ht]
%   \centering
%   \includegraphics [scale=0.27] {latex}
%   \caption{Подпись к картинке.}
 %  \label{fig:latex}
% \end{figure}

% Формулы в строку без номера добавляются так:
% \[
%   \lambda_{T_s} = K_x\frac{d{x}}{d{T_s}}, \qquad
%   \lambda_{q_s} = K_x\frac{d{x}}{d{q_s}},
% \]

 \underline{\textbf{Вторая глава}} посвящена исследованию детектора на основе лавинного пробоя в фотодиоде, который применяется в качестве измерительного оборудования для регистрации результата интерференции квантовых состояний в системе квантовой коммуникации на боковых частотах (ККБЧ) фазомодулированного излучения. Такой тип ДОФ обычно применяется для внутригородских дистанций до 100~км с потерями в линиях связи менее 15~дБ. 

Его отличительными особенностями являются:
\begin{enumerate}
	\item Поддержка высокой частоты стробирующих импульсов - до 100~МГц
	\item Возможность подачи стробирующих импульсов от внешнего устройства (External gating mode)
	\item Широкий диапазон настройки ширины окна срабатывания (gate) - от 0,5~нс до 25~нс
	\item Выставление задержки открытия окна срабатывания относительно стробирующего импульса (Trigger delay) в диапазоне до 10~нс с высоким разрешением во времени - 10~пс 
	\item Возможность выставления <<мертвого времени>> в широком диапазоне - от 0,1~мкс до 100~мкс
	\item Возможность регулировки квантовой эффективности с шагом 2,5~\% в диапазоне от 5~\% до 25~\%
	\item Полупроводниковая структура ЛФД - InGaAs/InP
	\item Относительно низкий уровень темнового счета при заданных параметрах квантовой эффективности
\end{enumerate}

В ходе исследования для обеспечения реалистичных условий атаки злоумышленника на измерительное оборудование в составе системы ККБЧ устройство рассматривалось, как <<черный ящик>>, то есть оно не вскрывалось и не производились манипуляции с внутренними платами и микросхемами. Все настройки детектора выставлялись в соответствии со штатным режимом для систем квантовой коммуникации на боковых частотах модулированного излучения. 

Для успешного осуществления атаки с навязыванием ключа злоумышленнику требуется манипулировать детектором, то есть форсировать срабатывания и их отсутствие в нужные моменты времени, при этом предполагается, что тип применяемого измерительного оборудования известен, но непосредственный доступ к нему отсутствует. В таких рамках модель атаки ограничивается возможностью воздействия на детектор только оптическими методами непосредственно из квантового канала. 


Известно, что в линейном режиме работы ЛФД при подачи на него постоянной оптической мощности увеличивается фототок, следовательно при подаче постоянного значения напряжение обратного смещения $-V_{bias}$ и при наличии в цепи гасящего лавину резистора, величина падения напряжения на резисторе растет, а на ЛФД снижается. Суть атаки с выведением детектора из режима Гейгера, или <<ослеплением>> ДОФ, сводится к тому, чтобы сместить режим работы относительно напряжения пробоя ЛФД. При таком подходе даже дополнительных импульсов $V_{gate}$ становится недостаточно и диод все время находится в режиме линейной зависимости фототока от величины мощности оптического излучения, падаюшего на него.  

Тем не менее, в линейном режиме остается возможность превысить пороговое значение фототока $I_{det}$ и сформировать импульс срабатывания детектора.

Таким образом, методика выведения детектора из режима счета фотонов в линейный режим для осуществления атаки с навязыванием ключа (<<Faked-state attack>>) легко формализуется. Экспериментальное исследование уязвимости детектора одиночных фотонов к такому типа атак реализуется в три этапа, представленных на рисунке \ref{fig:Method_2.3}:
%
 \begin{figure}[ht] 
  \centering
  \includegraphics[scale=0.5]{Method_2.3.eps}
  \caption{Методика выведения детектора из режима Гейгера}
  \label{fig:Method_2.3}
\end{figure}
%
 \begin{enumerate}
	\item Определение величины постоянной оптической мощности, достаточной для выведения детектора из режима Гейгера
	\item Подстройка оптического импульса под окно срабатывания детектора одиночных фотонов
	\item Определение зависимости вероятности срабатывания детектора от величины энергии фотонов в импульсе
\end{enumerate}
%
В результате экспериментального исследования получен ряд зависимостей вероятности формирования отсчета от величины оптической мощности, применяемой для выведения детектора из режима счета фотонов, для различных оптических контролирующих импульсов (рис. \ref{fig:Probability_vs_Energy}). 

\begin{figure}[ht]
  \centering
  \includegraphics[scale=0.5]{Probability_vs_Energy.png}
  \caption{Зависимость вероятности срабатывания детектора от энергии контролирующего импульса}
  \label{fig:Probability_vs_Energy}
\end{figure}


Таким образом показано, что использование коммерческих детекторов одиночных фотонов на основе лавинных фотодиодов в режиме Гейгера модели id210 с частотой стробирования 100 МГц  требует применения дополнительных средств защиты от атаки с выведением из режима Гейгера при помощи коротких оптических импульсов с энергией не менее 15,4 фДж и при постоянном уровне оптической засветки средним уровнем мощности излучения не менее 35 нВт.  

 \underline{\textbf{Третья глава}} посвящена исследованию
 
 главе показано, что измерение величины оптического излучения на несущей частоте, отраженного от оптического фильтра, при помощи мониторного фотодиода в приемном блоке системы квантовой коммуникации на боковых частотах в диапазоне от 7 нВт до 2,93 мкВт с применением дополнительных мер в виде пассивного оптического аттенюатора номиналом 10 дБ для его защиты позволяет противостоять атаке с выведением детектора одиночных фотонов из режима Гейгера и навязыванием ключа нелегитимным пользователем. 

% Можно сослаться на свои работы в автореферате. Для этого в файле
% \verb!Synopsis/setup.tex! необходимо присвоить положительное значение
% счётчику \verb!\setcounter{usefootcite}{1}!. В таком случае ссылки на
% работы других авторов будут подстрочными.
% \ifnumgreater{\value{usefootcite}}{0}{
% Изложенные в третьей главе результаты опубликованы в~\cite{vakbib1, vakbib2}.
% }{}
% Использование подстрочных ссылок внутри таблиц может вызывать проблемы.

 В \underline{\textbf{четвертой главе}} приведено описание
 
 главе показано, что метод квантовой коммуникации на боковых частотах позволяет реализовывать протокол, устойчивый к контролю нелегитимным пользователем измерительного оборудования. 
 
  В \underline{\textbf{пятой главе}} приведено описание
  
 главе показано, что в результате интерференции квантового фазомодулированного сигнала на боковых частотах на симметричном светоделителе в схеме квантовой рассылки ключа с узлом регистрации, независящим от легитимного пользователя, происходит спектральное разделение квантового сигнала и сигнала на центральной длине волны с их независимой регистрацией в разных плечах светоделителя. 

 \underline{\textbf{Заключение}} содержит список основных результатов, полученных в работе. 

% !_____!
%  %% Согласно ГОСТ Р 7.0.11-2011:
%% 5.3.3 В заключении диссертации излагают итоги выполненного исследования, рекомендации, перспективы дальнейшей разработки темы.
%% 9.2.3 В заключении автореферата диссертации излагают итоги данного исследования, рекомендации и перспективы дальнейшей разработки темы.
\begin{enumerate}
  \item На основе экспериментального анализа детектора, работающего в режиме Гейгера, показано, что требуются дополнительные средства защиты от атаки с выведением из режима Гейгера при помощи коротких оптических импульсов с энергией не менее 15,4 фДж и при постоянном уровне оптической засветки средним уровнем мощности излучения не менее 35 нВт. 
  \item Численные исследования показали, что измерение величины оптического излучения на несущей частоте, отраженного от оптического фильтра, при помощи мониторного фотодиода в приемном блоке системы квантовой коммуникации на боковых частотах в диапазоне от 7 нВт до 2,93 мкВт с применением дополнительных мер в виде пассивного оптического аттенюатора номиналом 10 дБ для его защиты позволяет противостоять атаке с выведением детектора одиночных фотонов из режима Гейгера и навязыванием ключа нелегитимным пользователем. 
  \item Метод квантовой коммуникации на боковых частотах позволяет реализовывать протокол, устойчивый к контролю нелегитимным пользователем измерительного оборудования.
  \item Для выполнения поставленных задач был создан экспериментальный стенд и в результате интерференции квантового фазомодулированного сигнала на боковых частотах на симметричном светоделителе в схеме квантовой рассылки ключа с узлом регистрации, независящим от легитимного пользователя, наблюдается спектральное разделение квантового сигнала и сигнала на центральной длине волны с их независимой регистрацией в разных плечах светоделителя. 
\end{enumerate}

%При использовании пакета \verb!biblatex! список публикаций автора по теме
%диссертации формируется в разделе <<\publications>>\ файла
%\verb!../common/characteristic.tex!  при помощи команды \verb!\nocite!

\ifdefmacro{\microtypesetup}{\microtypesetup{protrusion=false}}{} % не рекомендуется применять пакет микротипографики к автоматически генерируемому списку литературы
\ifnumequal{\value{bibliosel}}{0}{% Встроенная реализация с загрузкой файла через движок bibtex8
  \renewcommand{\bibname}{\large \authorbibtitle}
  \nocite{*}
  \insertbiblioauthor           % Подключаем Bib-базы
  %\insertbiblioother   % !!! bibtex не умеет работать с несколькими библиографиями !!!
}{% Реализация пакетом biblatex через движок biber
  \ifnumgreater{\value{usefootcite}}{0}{
%  \nocite{*} % Невидимая цитата всех работ, позволит вывести все работы автора
  \insertbiblioauthorcited      % Вывод процитированных в автореферате работ автора
  }{
  \insertbiblioauthor           % Вывод всех работ автора
%  \insertbiblioauthorgrouped    % Вывод всех работ автора, сгруппированных по источникам
%  \insertbiblioauthorimportant  % Вывод наиболее значимых работ автора (определяется в файле characteristic во второй section)
  \insertbiblioother            % Вывод списка литературы, на которую ссылались в тексте автореферата
  }
}
\ifdefmacro{\microtypesetup}{\microtypesetup{protrusion=true}}{}
